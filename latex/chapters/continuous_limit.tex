The only thing left to prove now is that as $L \rightarrow \infty$, \cref{prob:min-q} does indeed converge towards \cref{prob:cont-least-action}.

\begin{figure}
	\makebox[\textwidth][c]{
		\centering
		\begin{tikzpicture}[font=\scriptsize, align=center]
				\node[draw, above left, align=center] (a) {
					ResNet \ref{prob:min-v-f}\\
					$\begin{cases}
						\text{Min~} & \frac{\nu \cdot L}{2} \sum_{k=1}^{L} \norm{v_k}_\cV^2
						+ l_R(\Phi_L(X), Y) \\
						\text{s.t.~} & v_1, \ldots, v_L \in \cV,\ \Phi_L = (I + v_L) \circ \ldots \circ (I + v_1)
					\end{cases}$};
				\node[right=of a, align=center,outer sep=-2em] (z1) {\large$\stackrel{L \rightarrow \infty}{\longrightarrow}$\\
					\scriptsize\cref{theo:problem-convergence}};
				\node[draw, right= of z1, align=center] (b) {
					Limit of ResNet \ref{prob:resnet-limit}\\
					$\begin{cases}
						\text{Min~}& \frac{\nu}{2} \int_{0}^{1} \norm{v}_\mathcal{V}^2 \mathrm{d}t
						+ l(\phi_v(X, 1), Y)\\
						\text{s.t.~}& v \in C([0, 1], \mathcal{V}),\ \Phi(x, 0) = x,\\
						&\dot{\Phi}^v(x, t) = \mathbf{\Gamma}(\Phi^v(x, t), q(t)) \bGamma(q(t), q(t))^{-1} \dot{q}(t)
					\end{cases}$};
				\node[below= of a,outer sep=-2em] (z2) {\large{$\big\Updownarrow$} \scriptsize\cref{theo:v-q-problem-equivalence}};
				\node[draw, below= of z2, align=center] (c) {
					Discrete Least Action \ref{prob:min-q}\\
					$\begin{cases}
						\text{Min~} & \frac{\nu}{2} \sum_{k=1}^{L} \left(\frac{q_{k+1} - q_k}{\Delta t}\right)^\mathrm{T} \bGamma(q_k, q_k)^{-1} \left(\frac{q_{k+1} - q_k}{\Delta t}\right) \Delta t\\
						&+ l(q_{L+1}, Y) \\
						\text{s.t.~} & q_1 = X,\ q_2, \ldots, q_{L+1} \in \cX^N,\ t = \frac{1}{L}\\
						& v_k(x) = \bGamma(x, q_k)^\mathrm{T}\bGamma(q_k, q_k)^{-1} (q_{k+1} - q_k)
					\end{cases}$};
				\node[below= of b,outer sep=-2em] (z3) {\large{$\big\Updownarrow$} \scriptsize\cref{theo:v-q-continuous-problem-equivalence}};
				\node[right=of c, align=center,outer sep=-2em] (z4) {\large$\stackrel{L \rightarrow \infty}{\longrightarrow}$\\
					\scriptsize\cref{theo:problem-convergence}};
				\node[draw, below= of z3, right=of z4, align=center] (d) {
					Least Action \ref{prob:cont-least-action}\\
					$\begin{cases}
					\text{Min~} & \frac{\nu}{2} \int_{0}^{1} \dot{q}(t)^\mathrm{T} \mathbf{\Gamma}(q(t), q(t))^{-1}  \dot{q}(t) \mathrm{d}t + l(q(1), Y)\\
					\text{s.t.~} & q \in C^1([0,1], \cX^N),\ q(0) = X
					\end{cases}$};
				\node[below= of d,outer sep=-2em] (z5) {\large{$\big\Updownarrow$} \scriptsize\cref{theo:geodesic-shooting}};
				\node[draw, below= of z5, align=center] (e) {
					Geodesic Shooting \ref{prob:geodesic-shooting}\\
					$\begin{cases}
						\text{Min~} & \frac{\nu}{2} p(0)^\T \bGamma(X, X)p(0) + l(q(1), Y)\\
						\text{s.t.~} & 	p = \bGamma(q, q)^{-1}\dot{q},\ q(0) = X,\ \dot{q} = \bGamma(q, q) p,\\
						&\dot{p} = -\grad_q \left(\frac{1}{2} p^\mathrm{T} \bGamma(q, q) p\right)
					\end{cases}$};
				\node[below= of c,outer sep=-2em] (z6) {\large{$\big\Updownarrow$} \scriptsize\cref{theo:discrete-shooting-min-q-equivalence}};
				\node[draw, below= of z6, align=center] (f) {
					Discrete Geodesic Shooting \ref{prob:discrete-geodesic-shooting}\\
					$\begin{cases}
					\text{Min~} & \frac{\nu}{2} \sum_{k=1}^L p_k^\T \bGamma(q_k, q_k) p_k \Delta t + l(q_{L+1}, Y)\\
					\text{s.t.~} & p_k = \bGamma(q_k, q_k)^{-1} \frac{q_{k+1} - q_k}{\Delta t},\ q_1 = X \\
					& q_{k+1} = q_k + \Delta t \bGamma(q_k, q_k) p_k,\\
					& p_{k+1} = p_k + \frac{\Delta t}{2} \grad_{q_k} \left(p_k^\T \bGamma(q_k, q_k) p_k\right)
					\end{cases}$};
				\node[right=of f, align=center,outer sep=-2em] (z7) {\large$\stackrel{L \rightarrow \infty}{\longrightarrow}$\\
					\scriptsize\cref{theo:problem-convergence}};
		\end{tikzpicture}
	}
	\caption{Summary of problems, their correspondence, their continuous limits and the respective theorems. Discrete problems are on the left, continuous on the right.}
	\label{fig:convergence}
\end{figure}

This is equation 1.12 from \cite{owhadi20}.
See how this fits here
\begin{equation}
	\label{eq:phi-v-differential-equation}
	\dot{\Phi}^v(x, t) = \mathbf{\Gamma}(\Phi^v(x, t), q) p
\end{equation}

\begin{problem}
	\label{prob:resnet-limit}
	\begin{cases}
		\text{Minimize~}& \frac{\nu}{2} \int_{0}^{1} \norm{v}_\mathcal{V}^2 \mathrm{d}t
		+ l(\phi_v(X, 1), Y)\\
		\text{such that~}& v \in C([0, 1], \mathcal{V})\\
	\end{cases}
\end{problem}

\begin{theorem}
	\label{theo:v-q-continuous-problem-equivalence}
	$v$ minimizes \cref{prob:resnet-limit} if and only if $v$ fulfills
	\begin{equation}
			\dot{\Phi}^v(x, t) = \mathbf{\Gamma}(\Phi^v(x, t), q_t) \bGamma(q_t, q_t)^{-1} \dot{q}_t
	\end{equation}
	such that $\Phi(x, 0) = x$ for all $x \in X$ and $q$ minimizes \cref{prob:cont-least-action}.
\end{theorem}
The proof of this theorem is similar to that of \cref{theo:v-q-problem-equivalence}.
\begin{proof}
	TBD... Seems like we can just reconstruct the proof of the other theorem with some extra steps.
\end{proof}


\subsubsection{Existence of Minimizers}

We already saw the equivalence of \cref{prob:cont-least-action,prob:geodesic-shooting,prob:resnet-limit}, in the sense that their minimizers bijectively correspond to each other -- if they exist.
Now we will show that there do indeed exist such minimizers and also that the problems' minimal values are identical.
\begin{theorem}[Existence for the continuous problems]
		There exist minimal points for \cref{prob:cont-least-action,prob:geodesic-shooting,prob:resnet-limit} and their minimal values are identical.
\end{theorem}

\Todo{Check with paper again. In the paper the equation for p might be wrong...}
\begin{equation}
	\label{eq:discrete-hamiltonian-system}
	\begin{split}
		q_{k+1} &= q_k + \Delta t \bGamma(q_k, q_k) p_k\\
		p_{k+1} &= p_k + \frac{\Delta t}{2} \grad_q{k} \left(p_k^\T \bGamma(q_k, q_k) p_k\right)\ .
	\end{split}
\end{equation}

\begin{problem}
	\label{prob:discrete-least-action}
	\begin{cases}
		\text{Minimize~} & \frac{\nu}{2} \sum_{k=1}^L p_k^\T \bGamma(q_k, q_k) p_k \Delta t + l(q_{L+1}, Y)\\
		\text{such that~} & p_k = \bGamma(q_k, q_k)^{-1} \frac{q_{k+1} - q_k}{\Delta t},\ q_1 = X \\
		&\text{~and~} (q_k, p_k) \text{~follow the discrete Hamiltonian equations \ref{eq:discrete-hamiltonian-system}}\ .
	\end{cases}
\end{problem}

\begin{theorem}[Existence of solutions to the discrete problems]
	There exist minimal points for \cref{prob:discrete-least-action,prob:min-v-f,prob:min-q} and their minimal values are identical.
\end{theorem}


\subsubsection{Convergence}

The following theorem is quite a big one.
It states convergence results for all three discrete problems we have dealt with so far.
\begin{theorem}[Convergence theorem]
	\label{theo:problem-convergence}
	content...
\end{theorem}