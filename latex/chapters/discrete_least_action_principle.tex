Our main goal will now be to show that the minimization problem \ref{prob:min-v-f} can in fact be considered as a discrete solver of a mechanical system.
In order to do that, we first reformulate the problem by introducing additional variables $q_{i,j}$ for $2 \leq i \leq L+1$, $j \in [N]$:
\begin{align}
	q_{1, j} &\coloneqq X_j \, \\
	q_{i, j} &\coloneqq (\phi_{i-1} \circ \ldots \circ \phi_1) (X_j) \ .
\end{align}
Writing $q_i$ for the vector with entries $q_{i,j}$, we get $q_1 = X$ and $q_i = \phi_i(q_{i-1})$.
Consider the following minimization problem:
\begin{problem}
	\label{prob:min-q}
	\begin{cases}
		\text{Minimize~} & \nu \cdot \frac{1}{2} \sum_{k=1}^{L} \left(\frac{q_{k+1} - q_k}{\Delta t}\right)^\mathrm{T} \bGamma(q_k, q_k)^{-1} \left(\frac{q_{k+1} - q_k}{\Delta t}\right) \Delta t+ l(q_{L+1}, Y) \\
		\text{such that~} & q_1 = X,\ q_2, \ldots, q_{L+1} \in \cX^N  \text{~and~} \Delta t = \frac{1}{L} \ .
	\end{cases}
\end{problem}
We will now prove that this problem is in fact equivalent to \cref{prob:min-v-f} and also that we can obtain a closed-form expression for the $v_k$ as a function of the $q_k$.
Remember from \cref{sec:preliminaries} that $\bGamma(x, q_k)$ denotes the vector with entries $\Gamma(x, q_{k,i})$.

\begin{theorem}
	\label{theo:v-q-problem-equivalence}
	$v_1, \ldots, v_L \in \cV$ minimize \cref{prob:min-v-f} if and only if $q_1, \ldots, q_{L+1} \in \cX^N$ minimize \cref{prob:min-q} and $v_k(x) = \bGamma(x, q_k)^\mathrm{T}\bGamma(q_k, q_k)^{-1} (q_{k+1} - q_k)$ for all $k \in [L]$.
\end{theorem}
\begin{proof}
	Let $v_1, \ldots, v_L$ minimize \cref{prob:min-v-f}.
	Using $q_k$ as defined above we have $v_k(q_k) = q_{k+1} - q_k$ for all $k \in [L]$.
	We can rewrite \cref{prob:min-v-f} as
	\begin{problem}
		\label{prob:min-q-v}
		\begin{cases}
			\text{Minimize~} & \nu \cdot \frac{L}{2} \sum_{k=1}^{L} \norm{v_k}_\cV^2
			+ l(q_{L+1}, Y) \\
			\text{such that~} & v_1, \ldots, v_L \in \cV,\ \forall k \in [L]: v_k(q_k) = q_{k+1} - q_k, \\
			& q_1 = X \text{~and~} q_2, \ldots, q_{L+1} \in \cX^N \ .
		\end{cases}
	\end{problem}
	Here we just added the $q_k$ as additional variables but then constrained them to $q_{k+1} = q_k + v_k(q_k)$, with $q_1 = X$.
	Recursively we get $q_{L+1} = \Phi_L(X)$ which means the target function remains the same as in \cref{prob:min-v-f}.
	Also, the domains of the $v_k$ do not change.
	This means that the minimal points are the same as in \cref{prob:min-v-f}.
	
	We will now derive closed form expressions for the $v_k$ as functions of the $q_k$.
	For that, let $q_k \in \cX^N$, $2 \leq k \leq N$ be arbitrary but fixed .
	This means $l(q_{L+1}, Y)$ is constant and $v_k \in V_q(k) \coloneqq \{v \in \cV~|~ v_k(q_k) = q_{k+1} - q_k\}$.
%	These sets remain convex: Let $\lambda \in [0, 1]$ and $v_{k, 1}, v_{k, 2} \in V_{k, q}$, then we get
%	\begin{align}
%		(\lambda v_{k, 1} + (1 - \lambda) v_{k, 2})(q_k) &= (\lambda v_{k, 1})(q_k) + ((1 - \lambda) v_{k, 2})(q_k)\\
%		&=\lambda(q_{k+1} - q_k) + (1 - \lambda)(q_{k+1} - q_k)\\
%		&= q_{k+1} - q_k \ .
%	\end{align}
	We are left with the following problem:
	\begin{problem}
		\begin{cases}
			\text{Minimize} &\sum_{k=1}^L \norm{v_k}_\cV^2\\
			\text{such that} & \forall k \in [L]: v_k \in V_q(k) \ ,
		\end{cases}
	\end{problem}
	where each summand is non-negative.
	One can easily verify that minimization problems of this type have a global minimum at any point where each of the summands is minimal.
	The boundedness of the summands ensures that at least one such minimum exists.

	Thus, we have to find the minima of $\min\{\norm{v_k}_\cV^2\ |\ v_k \in \cV,\ v_k(q_k) = q_{k+1} - q_k\} = l_{\text{OR}}(q_k, q_{k+1} - q_k)$.
	Here, $l_{\text{OR}}$ is the optimal recovery loss defined in \cref{eq:optimal-recovery-loss}.
	From that section we get the following representations for the minimal $v_k$ and the target value:
	\begin{align}
		v_k & = \bGamma(\cdot, q_k)^\mathrm{T}\bGamma(q_k, q_k)^{-1} (q_{k+1} - q_k) \ ,\\
		\norm{v_k}_\cH^2 &= (q_{k+1} - q_k)^\mathrm{T} \bGamma(q_k, q_k)^{-1} (q_{k+1} - q_k) \ .
	\end{align}
	In these expressions $\bGamma(q_k, q_k)$ is the block operator matrix with entries $\Gamma(q_{k,i}, q_{k, j})$ and $\bGamma(x, q_k)$ the vector $(\Gamma(x, q_{k, i}))_{i \in [N]}$.
	Having now computed optimal $v_k$ -- or rather their squared $\cH$-norms -- as a function of $q_2, \ldots, q_{L+1}$, we can reformulate \cref{prob:min-q-v} without the $v_k$ as variables.
	Define $\Delta t \coloneqq \frac{1}{L}$.
	We get
	\begin{problem}
		\begin{cases}
			\text{Minimize~} & \nu \cdot \frac{1}{2} \sum_{k=1}^{L}  
			\left(\frac{q_{k+1} - q_k}{\Delta t}\right)^\mathrm{T} \bGamma(q_k, q_k)^{-1}
			\left(\frac{q_{k+1} - q_k}{\Delta t}\right) \cdot \Delta t
			+ l(q_{L+1}, Y) \\
			\text{such that~} & q_1 = X \text{~and~} q_2, \ldots, q_{L+1} \in \cX^N \ .
		\end{cases}
	\end{problem}
	As all steps and transformations hold true in both ways, this concludes the proof.
\end{proof}

When comparing the target function of \cref{prob:min-q} with the results found in discrete mechanics literature \cite[~Chapter VI.6.2]{hairer06}, in particular with those for the discretized stationary action principle, it turns out that there are strong similarities.
We will not explore these similarities in depth for the discrete case but rather take a closer look at the continuous counterpart in the next section.