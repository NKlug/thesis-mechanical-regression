\begin{polyabstract}{Abstract} 
	Despite the extraordinary success of neural networks, they are still not fully understood from a mathematical point of view.
	In "Do Ideas Have Shape? Plato's Theory of Forms as the Continuous Limit of Neural Networks", Houman Owhadi analyzes a subclass of neural networks which are called residual neural networks (ResNets) and shows their convergence towards a continuous mechanical system.
	This system resembles algorithms from image registration and computational anatomy.
	
	In this thesis, parts of Owhadi's results are presented and discussed.
	It is shown that ResNets relate to a discretized stationary action principle which can be formulated as a discrete geodesic shooting problem.
	Because of the connection to physics, they are called \emph{mechanical regression}.
	All three discrete problems have continuous counterparts and converge towards them.
	This convergence is in the sense that as the number of ResNet layers tends towards infinity or the step size towards zero, the minimal values converge and the adherence points of sequences of minimizers are solutions to the continuous problems.
	
	From the continuous geodesic shooting problem, an algorithm for the supervised learning problem can be derived.
	Parts of this algorithm are implemented and numerical experiments are conducted.
	The results closely resemble those presented by Owhadi.	
\end{polyabstract}

\pagebreak
\begin{polyabstract}{Zusammenfassung}
	TBD
\end{polyabstract}
