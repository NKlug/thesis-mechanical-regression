In order to show that similarity of \cref{prob:min-q} to a mechanical system, consider the first term of its objective function (the balancing parameter $\nu$ has been deliberately omitted):
\begin{equation}
	\label{eq:discrete-lagrangian}
	\frac{1}{2} \sum_{k=1}^{L} \left(\frac{q_{k+1} - q_k}{\Delta t}\right)^\mathrm{T} \mathbf{\Gamma}(q_k, q_k)^{-1} \frac{q_{k+1} - q_k}{\Delta t} \Delta t \ .
\end{equation}
With some imagination this very much looks like an approximation to the integral of some continuous function with a step function using $L$ intervals of width $\Delta t$.
This would require the sequence $q_k$ to be a discrete equidistant sampling of a function $q: A \rightarrow \cX^N$, where $A$ is an interval in $\R$.
Without loss of generality we choose $A = [0, 1]$ and let $q$ be such that $q_k = q(k \Delta t)$.
This way, $\frac{q_{k+1} - q_k}{\Delta t}$ can be interpreted as a forward differential quotient approximating the derivative of $q$ on at the point $k \Delta t = \frac{k}{L}$.
Using physical nomenclature we call the domain of $q$ \emph{time} and write $\dot{q}$ for the (time-)derivative $\frac{\mathrm{d}}{\mathrm{d}t}q(t)$.
In mechanical applications, $q$ is the trajectory of particles and $\dot{q}$ their velocities.
We get the approximation
\begin{equation}
	\left(\frac{q_{k+1} - q_k}{\Delta t}\right)^\mathrm{T} \mathbf{\Gamma}(q_k, q_k)^{-1} \frac{q_{k+1} - q_k}{\Delta t} 
	\approx \dot{q}\left(\frac{k}{L}\right)^\mathrm{T} \mathbf{\Gamma}\left(q\left(\frac{k}{L}\right), q\left(\frac{k}{L}\right)\right)^{-1}\dot{q}\left(\frac{k}{L}\right) \ .
\end{equation}
This leads us to the obvious question if the solutions of \cref{prob:min-q} are also approximating the solutions of a similar continuous problem.
As it turns out, they do.
This "convergence" will be the main result of this section and is presented at the end in \cref{theo:problem-convergence}.
For now, we will work under the assumption that it holds true and continue evolving the theory, being cautious not to introduce proof cycles.

Consider the \emph{action}
\begin{equation}
\label{eq:action}
	\cA(q) \coloneqq \int_{0}^{1} \fL(t, q(t), \dot{q}(t)) \mathrm{d}t \ ,
\end{equation}
with the \emph{Lagrangian}
\begin{equation}
\label{eq:lagrangian}
	\begin{split}
		\fL(t, x_1, x_2): [0, 1] \times \cX^N \times \cX^N \rightarrow \R\\ \fL(t, x, y) \coloneqq \frac{1}{2}  x_2^\mathrm{T} \mathbf{\Gamma}(x_1, x_1)^{-1} x_2 \ .
	\end{split}
\end{equation}
These expressions have their origins in Lagrangian mechanics (from a physical point of view) and calculus of variations (the mathematical approach) and have a strong physical interpretation.
But first, a caveat concerning the Lagrangian:
It is not exclusively defined when the third argument is the time derivative of the second, but more generally.
Using this general definition, we can then construct a function $(t, q) \rightarrow \fL(t, q(t), \dot{q}(t))$, where $\dot{q} = \frac{\mathrm{d}}{\mathrm{d}t}q(t)$.
This was done in the definition of the action.

In a physical scenario, the Lagrangian is often the difference between the kinetic and the potential energy of (the particles of) the system, that is $L = T - V$.
Usually $T$ does not depend on the second and $V$ not on the third argument of the function, which correspond to the velocities and the locations of the particles.
This theory is especially helpful in problems where constraining forces act.

In our case, the Lagrangian is a quadratic form.
In a physical interpretation it most closely resembles that of a system of free moving particles that don't interact with each other (which also is a quadratic form, though not dependent on the location).
Here, the particles are coupled by a location dependent matrix, which can be regarded as a mass matrix \cite{marsden10}.
However, these "masses" would be location dependent.

With the definition of the action, we can write a continuous version of \cref{prob:min-q}:
\begin{problem}
\label{prob:cont-least-action}
	\begin{cases}
		\text{Minimize~} & \nu \cA(q) + l(q(1), Y)\\
		\text{such that~} & q \in C^1([0,1], \cX^N),\ q(0) = X \ .
	\end{cases}
\end{problem}
As usual, $C^1([0,1], \cX^N)$ is the set of continuously differentiable functions $q: [0, 1] \rightarrow \cX^N$.

Usually, $q$ describes the trajectory of particles.
In our case, each training sample constitutes a particle, therefore $q$ -- a vector -- describes the trajectories of not one, but $N$ particles.rich 
Now within the framework of theoretical physics, we can explore if the theory of Lagrangian mechanics can help us in the analysis the problem.

One core principle in Lagrangian Mechanics is \emph{Hamilton's Stationary Action Principle}, also referred to as \emph{the Principle of Least Action} or just Hamilton's Principle.
It postulates that\vspace{.5em}
\newline
\noindent{\emph{"The motion of the system from time $t_1$ to time $t_2$ is such that the [action] has a stationary value for the actual path of the motion." \cite{goldstein01}}}

This means that in the real world, the trajectories particles follow are a stationary point of the action.
A stationary point is one for which the \emph{variation} of the action is $0$.
Note that "Principle of Least Action" is therefore a misnomer in general and should rather be "Principle of Stationary Action", as the action does not necessarily have to be minimal.
Regardless, within the scope of \cref{prob:cont-least-action} it is appropriate to use the term "Least Action" because we are in fact seeking a minimum.
An introduction to the theory derived from the postulate can be found in various books on theoretical mechanics \cite{goldstein01, marsden10, feynman63} and in calculus of variations \cite{kielhofer18}.

A core result in calculus of variations are the Euler-Lagrange equations.
It can be shown that every extremal point $q$ of the actions satisfies them.
The Euler-Lagrange equations (EL) are given by
\begin{equation}
\frac{\mathrm{d}}{\mathrm{d}t} \frac{\partial \fL}{\partial \dot{q}_{i, j}} - \frac{\partial \fL}{\partial q_{i, j}} = 0
\end{equation}
and hold true for each component $q_{i, j}$ of the vector $q$ and its time-derivative $\dot{q}$.
The partial derivatives with respect to $q$ and $\dot{q}$ are to be read as the partial derivatives of the Lagrangian $\fL$ with respect to the second and third argument at the point $x_1 = q(t)$ and $x_2 = \dot{q}(t)$.

Now we will try to apply this theory to \cref{prob:cont-least-action}.
Here, we also seek to minimize the action, but in combination with the loss term which depends on the endpoint of the trajectory, $q(1)$.
This can be solved easily, as we will see in the following lemma.
\begin{lemma}
	A minimizer $q$ of \cref{prob:cont-least-action} follows the Euler-Lagrange equations.
\end{lemma}
\begin{proof}
	Let $q(1) \coloneqq a \in \cX^N$ be arbitrary but fixed.
	Then $l(q(1), Y)$ is constant and the problem is reduced to the minimization of the first term -- the action.
	By \cite[~Proposition 1.4.1]{kielhofer18} it follows that $q$ satisfied the EL equations.
	As this holds true for all endpoints $a$, it especially hold true for optimal $q$'s endpoint.
\end{proof}

Within the scope of \cref{prob:cont-least-action} we do not rely on the Stationary Action Principle to be postulated.
By the problem's definition it is \emph{required} that the actual trajectory $q$ minimizes the action.

The vector $q$ describes the trajectories of $N$ particles.
Setting $m \coloneqq \mathrm{dim}(\cX)$, this would result in $mN$ EL equations, one for each $q_{i, j}$, $i \in [N], j \in [m]$.
For simplicity we will combine them into one and write $\grad_{\dot{q}} \fL$ for the vector of partial derivatives with respect to $\dot{q}_{i, j}$.
$\grad_q \fL$ is to be interpreted similarly.
Together with the block operator notation we can write the EL equations in a concise formulation.
Because of $\grad_{\dot{q}} \fL = \bGamma(q(t), q(t))^{-1} \dot{q}(t)$. It reads
\begin{equation}
\label{eq:concrete-lagrangian}
	\frac{\mathrm{d}}{\mathrm{d}t} \left(\mathbf{\Gamma}(q(t), q(t))^{-1} \dot{q}(t) \right)
	= \grad_q \left(\frac{1}{2} \dot{q}(t)^\mathrm{T} \mathbf{\Gamma}(q(t), q(t))^{-1} \dot{q}(t)\right) \ .
\end{equation}

$q(t)$ is an $(N\times1)$ vector in $\cX^N$.
Each of its elements can be written as a $(m\times1)$ vector.
So technically speaking $q$ is $(N \times m)$, or equivalently (\cref{cor:matrix-ring-equivalence}) a a $(mN \times 1)$ vector.
For the notation $\grad_q \fL$, the interpretation as $(mN \times 1)$ would be more appropriate.
However, for more flexibility we will not strictly define it like that and interpret such expressions with generosity.