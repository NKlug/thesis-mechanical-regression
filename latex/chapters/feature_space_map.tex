Next, we define feature spaces and feature maps.
They are of special interest in machine learning and are especially relevant for Support Vector Machines (SVMs) \cite{steinwart08}, where they enable learning non-linear decision functions.
\citet{steinwart08} give an extended introduction to kernels, feature maps and feature space for the scalar case.
Here, we will follow \cite{owhadi20} and define these terms for the more general case of operator-valued kernels.
We will mainly present results which are relevant for later sections.

\begin{definition}
	\label{def:feature-map-space}
	A Hilbert space $\cF$ and a function $\psi: \cX \rightarrow L(\cY, \cF)$ are \emph{feature space} and \emph{feature map} for the Kernel $K$ if for all $x_1, x_2 \in \cX, y_1, y_2 \in \cY$:
	\begin{equation}
		\left< y_1, K(x_1, x_2) y_2\right>_\cY = \left< \psi(x_1) y_1, \psi(x_2) y_2\right>_\cF \ .
	\end{equation}
\end{definition}

The Kernel $K$ does in itself induce a feature map and space.
The following example is an extension of \cite[Lemma~4.19]{steinwart08}, where the canonical feature map of $K$ is defined for the scalar case.
\begin{example}[Canonical feature map for operator-valued kernel]
	Let $a \in \cX$ be arbitrary but fixed and define $\psi(x) = \left(y \rightarrow K(\cdot, x)y\right)$.
	Then $\psi$ is a function $\cX \rightarrow \left(\cY \rightarrow \cH\right)$, as $K(\cdot, x) y \in \cH$ by definition.
	Furthermore we have for all $x_1, x_2 \in \cX$ and $y_1, y_2 \in \cY$:
	\begin{align}
		\left<\psi(x_1) y_1, \psi(x_2)y_2 \right>_\cH &= 
		\left< K(\cdot, x_1) y_1, K(\cdot, x_2) y_2\right>_\cH\\
		& = \left< K(x_2, x_1) y_1, y_2 \right>_\cY \\
		& = \left< y_1, K(x_2, x_1)^\T y_2 \right>_\cY\\
		& = \left< y_1, K(x_1, x_2) y_2\right>_\cY \ .		
	\end{align}
	This makes $\psi$ a feature map and $\cH$ a feature space for $K$. \qed
\end{example}

Let $\psi^\T: \cX \rightarrow L(\cF, \cY)$ be the adjoint of $\psi$ in the sense that for all $x \in \cX, y\in \cY, \alpha \in \cF$
\begin{equation}
 \left< \psi(x)y, \alpha\right>_\cF = \left<y, \psi^\T(x) \alpha \right>_\cY \ .
\end{equation}
With the adjoint we can derive a concise equation for the kernel $K$:
\begin{align}
	\left< y_1, K(x_1, x_2) y_2\right>_\cY = \left< \psi(x_1) y_1, \psi(x_2) y_2\right>_\cF = \left<y_1, \psi^\T(x_1)\psi(x_2) y_2\right>_\cY \ .
\end{align}
As this holds true for all $x_1, x_2 \in \cX, y_1, y_2 \in \cY$, we conclude
\begin{equation}
\label{eq:kernel-feature-map}
K(x_1, x_2) = \psi^\T(x_1) \psi(x_2) \ .
\end{equation}
Note that composing $\psi^T(x) \circ \psi(x)$ does indeed result in a function $\cY \rightarrow \cY$, which matches the target type of $K$.

Using $\psi^\T$ and $\alpha \in \cF$, we can also define functions $\cX \rightarrow \cY$:
\begin{align}
		\psi^\T\alpha: ~&\cX \rightarrow \cY \\
		(\psi^\T\alpha)(x) &\coloneqq \psi^\T(x)\alpha \ .
\end{align}
Without loss of generality we can restrict $\cF$ to the image of the function
\begin{align}
	\varphi: ~&\cX \times \cY \rightarrow \cF\\
	\varphi(x, y) &\coloneqq \psi(x)y\ .
\end{align}
We will not detail why this is possible.
\Todo{Maybe it is straightforward to see that cF is still Banach in this case.It is clear that the inner product still works.}
It especially requires $\mathrm{im}(\varphi)$ to be a Hilbert space.
The intuition behind this ...
\Todo{Intuition: Move from X to some weird high dimensional space which allows learning non linearities.
But it doesnt really matter what this feature space cF looks like as we dont have to care about it.
Therefore we can restrict F to whatever we want - as long as it makes sense.}
Restricting $\cF$ like this has the advantage that $\cH$ is then equal to the closure of the span of $\{\psi^\T\alpha\ |\ \alpha \in \cF\}$.
From this result we obtain the following theorem.
\begin{theorem}
	\label{theo:f-h-correspondence-equation}
	For all $\alpha, \beta \in \cF$ it holds true that
	\begin{equation}
		\left< \psi^\T(\cdot) \alpha, \psi^\T(\cdot) \beta\right>_\cH = \left< \alpha, \beta\right>_\cF \ .
	\end{equation}
\end{theorem}
\begin{proof}
	
\end{proof}
\begin{corollary}
	\label{cor:feature-space-norm}
	\begin{equation}
		\norm{\psi^\T(\cdot)\alpha}_\cH^2 = \norm{\alpha}_\cF^2 \ .
	\end{equation}
\end{corollary}
\begin{proof}
	This is an immediate consequence of \cref{theo:f-h-correspondence-equation}.
	Let $\alpha \in \cF$, then
	\begin{equation}
		\norm{\psi^\T(\cdot)\alpha}_\cH^2 = \left< \psi^\T(\cdot)\alpha, \psi^\T(\cdot)\alpha \right>_\cH
		= \left< \alpha, \alpha \right>_\cF = \norm{\alpha}_\cF^2 \ .
	\end{equation}
\end{proof}
