An alternative formulation of Lagrangian mechanics is the Hamiltonian formalism.
In its own, it does not add anything particularly new to the Lagrangian theory but rather gives a more powerful framework to work with the already established theory.
In essence, a change of variables from $(q, \dot{q}, t)$ to $(q, p, t)$ is applied through a certain kind of transformation called \emph{Legendre transformation}.
The obtained $(q, p)$ are known as the \emph{canonical variables}, concretely canonical coordinate and canonical momentum.
We will briefly cover the derivation of the Hamiltonian formulation within the scope of our application, but exclude most of the theoretical details.
For a complete formal derivation from a physical point of view, see \cite[Chapter~8]{goldstein01}.
For a more mathematical approach, consult \cite[Chapter~2]{marsden10}.

First, we define the \emph{canonical momentum}
\begin{equation}
\label{eq:canonical-momentum}
	p(t) \coloneqq \grad_{\dot{q}} \fL(t, q, \dot{q}) = \bGamma(q(t), q(t))^{-1} \dot{q}(t) \ .
\end{equation}
\sloppy{
Note that just as above, this equation is actually defined element-wise: 
$p_{i, j} = \frac{\partial}{\partial \dot{q}_{i,j}} \fL(t, q(t),\dot{q}(t))$.
}
Next, we define the \emph{Hamiltonian} function, often also called \emph{energy function} \cite{marsden10}:
\begin{equation}
	\fH(t, q, p) \coloneqq p(t)^\mathrm{T}\dot{q}(t) - \fL(t, q(t), \dot{q}(t)) \ .
\end{equation}
Formally the Hamiltonian is defined as a function $\fH(t, x_1, x_2): [0,1] \times \R \times \R \rightarrow \R$; \emph{without} the connection between $p$ and $q$ defined in \cref{eq:canonical-momentum}.
This means that we mostly treat the $p, q$ as independent variables unless mentioned otherwise.
In a physical context, the Hamiltonian can often be expressed as the sum of the system's kinetic and potential energies and represents the total energy of the system, hence the name "energy function".
For our application, we can calculate the Hamiltonian explicitly.
We get
\begin{align}
	\fH(q, p, t) &=p(t)^\mathrm{T}\dot{q}(t) - \fL(t, q(t), \dot{q}(t))\\
	&= \frac{1}{2} \dot{q}(t)^\T \bGamma(q(t), q(t))^{-1} \dot{q}(t) \\
	&= \frac{1}{2} \dot{q}(t)^\T \bGamma(q(t), q(t))^{-1} \bGamma(q(t), q(t)) \bGamma(q(t), q(t))^{-1} \dot{q}(t)\\
	&= \frac{1}{2}  \left(\bGamma(q(t), q(t))^{-1} \dot{q}(t)\right)^\T \bGamma(q(t), q(t)) \bGamma(q(t), q(t))^{-1} \dot{q}(t)\\
	&= \frac{1}{2} p(t)^\T \bGamma(q(t), q(t)) p(t) \ .
\end{align}

As you can see, the equations already become quite cluttered.
Thus, despite the risk of confusion, we will often omit the time dependence of $p$ and $q$ in the following.
For example, we write $\Gamma(q, q)$ instead of $\Gamma(q(t), q(t))$, but mean the same.

The next theorem describes the classical correspondence between Lagrangian and Hamiltonian mechanics:
\begin{theorem}
	\label{theo:hamiltonian-dynamic}
	If $q$ minimizes \cref{prob:cont-least-action} and $p = \grad_{\dot{q}} \fL(t, q, \dot{q})$, $(q, p)$ follow Hamilton's equations
	\begin{equation}
	\label{eq:hamiltonian-system}
		\begin{split}
			\dot{q} &= \grad_p \fH(q, p) = \bGamma(q, q) p\\
			\dot{p} &= -\grad_q \fH(q, p)
			= -\grad_q \left(\frac{1}{2} p^\mathrm{T} \bGamma(q, q) p\right)
		\end{split}
	\end{equation}
	with $q(0) = X$.
\end{theorem}
\begin{proof}
	Let $q$ be a minimizer of \cref{prob:cont-least-action}.
	Then $q(0) = X$ and, as we have seen, $q$ fulfills the Euler-Lagrange equations.
	The claim immediately follows from the equivalence of the Euler-Lagrange equations and Hamilton's equations under the diffeomorphism defined in \cref{eq:canonical-momentum}, as can be seen in e.g. \cite{marsden10, goldstein01}.
\end{proof}

From the fact that $\frac{\partial \fL}{\partial t} = 0$ it follows that:
\begin{corollary}
	\label{cor:energy-preservation}
	Along the trajectory $q$ the energy is preserved.
\end{corollary}
\begin{proof}
	The Lagrangian is not explicitly time dependent, that is $\frac{\partial \fL}{\partial t} = 0$.
	Write $\fL = \fL(t, q(t), \dot{q}(t))$.
	The Euler-Lagrange equations imply:
	\begin{equation}
		\frac{\mathrm{d}}{\mathrm{d} t} \fL = \sum_{i=1}^{N}\left(\frac{\partial \fL}{\partial q_i} \dot{q_i} + \frac{\partial \fL}{\partial \dot{q_i}} \ddot{q_i} \right )
		=\sum_{i=1}^{N}\left( \left( \frac{\mathrm{d}}{\mathrm{d} t} \frac{\partial \fL}{\partial \dot{q_i}} 
		\right)\dot{q_i}+ \frac{\partial \fL}{\partial \dot{q_i}} \ddot{q_i}\right )
		= \frac{\mathrm{d}}{\mathrm{d} t} \left(\sum_{i=1}^{N} \frac{\partial \fL}{\partial \dot{q_i}} \dot{q_i}\right) \ .
	\end{equation}
	From the definition of $p$ it follows that
	\begin{equation}
		0 = \frac{\mathrm{d}}{\mathrm{d} t} \left( \left(\sum_{i=1}^{N} \frac{\partial \fL}{\partial \dot{q_i}} \dot{q_i} \right) -  \fL \right) 
		= \frac{\mathrm{d}}{\mathrm{d} t} \left( \left(\sum_{i=1}^{N} p_i \dot{q_i} \right ) - \fL \right)
		= \frac{\mathrm{d}}{\mathrm{d} t} \fH \ ,
	\end{equation}
	which means that the energy function does not change over time and hence is constant.
\end{proof}

Our next task will be to show that there exists a unique solution to the Hamiltonian system in \cref{eq:hamiltonian-system}.
If this is the case, we could solve -- or at least approximate -- the flow of the Hamiltonian system to acquire a solution to \cref{prob:cont-least-action}:
A minimizer of \cref{prob:cont-least-action} solves the Hamiltonian system and if that system has a unique solution given suitable initial conditions, this must be it.
Of course, we can only identify the flow once we have determined the value of $p(0)$ as otherwise we would be lacking an initial value.

\begin{theorem}
	There exists a unique solution $(q, p)$ for the Hamiltonian system in \cref{eq:hamiltonian-system} with $q \in C^2([0, 1], \cX^N)$ and $p \in C^1([0, 1], \cX^N)$.
\end{theorem}

\begin{proof}
	Recall \cref{eq:kernel-feature-map}, which reads $\Gamma(x_1, x_2) = \psi^\mathrm{T}(x_1)\psi(x_2)$.
	Using this equality, we can rewrite \cref{eq:hamiltonian-system} as
	\begin{equation}
	\label{eq:feature-hamiltonian}
		\begin{split}
			\dot{q}_i &= \psi^\mathrm{T}(q_i) \alpha\\
			\dot{p}_i &= -\frac{\partial}{\partial q_i} \left(p_i^\mathrm{T} \psi^\mathrm{T}(q_i) \alpha \right)\ ,
		\end{split}
	\end{equation}
	with $\alpha \coloneqq \sum_{k=1}^{N} \psi(q_k) p_k$.
	From \cref{cor:feature-space-norm} we get:
	\begin{align}
		\norm{\alpha}_\cF^2 &= \norm{\psi^\T(x) \alpha}_\cV^2\\
		&= \norm{\sum_{i=1}^N \psi^\T(x) \psi(q_i)p_i}_\cV^2 \\
		&= \norm{\sum_{i=1}^N \Gamma(x, q_i) p_i}_\cV^2\\
		&= \left<\sum_{i=1}^N\Gamma(x, q_i)p_i, \sum_{j=1}^N \Gamma(x, q_j) p_j \right>_\cV\\
		&= \sum_{i=1}^N \sum_{j=1}^N \left<\Gamma(x, q_i)p_i, \Gamma(x, q_j) p_j \right>_\cV\\
		&= \sum_{i=1}^N \sum_{j=1}^N \left<p_i, \Gamma(q_i, q_j) p_j\right>_\cX\\
		&= p^\T \bGamma(q, q) p \ .
	\end{align}
	In the second to last step we again used the reproducing property of the kernel $\Gamma$.
	This gives us $\norm{\alpha}_\cF^2 = 2 \fH(q, p)$.
	\cref{cor:energy-preservation} states that $\fH$ is constant across time, which implies that $\norm{\alpha}_\cF^2$ is, too (and thus  $\norm{\alpha}_\cF$).
	Our goal is to use a global version of the Picard-Lindelöf theorem \cite[~Theorem 1.2.3]{arino06} to prove that the Hamiltonian system does have a unique solution.
	From this it immediately follows that $q$ and $p$ are $C^1$.
	The right side of the first line in \cref{eq:feature-hamiltonian} is also differentiable by $t$.
	Thus $q$ is $C^2$.
	
	For being able to apply Picard-Lindelöf we need the vector field $(q, p)$ to be globally Lipschitz continuous.
	Showing component-wise Lipschitz continuity suffices because we can choose the global constant $L$ as the maximum of the components'.
	We will show the Lipschitz property by proofing that $\dot{q}$ and $\dot{p}$ are bounded.
	This will do:
	Let $\dot{q}$ and $\dot{p}$ be bounded by $L \in \R$ and $q_1, p_1, q_2, p_2 \in \cX$, $x \coloneqq (q_1, p_1)^\T, y \coloneqq (q_2, p_2)^\T$ and
	\begin{equation}
	\label{eq:hamiltonian-time-derivative}
		\begin{pmatrix}
			\dot{q}\\\dot{p}
		\end{pmatrix}
		= \frac{\mathrm{d}}{\mathrm{d} t}\begin{pmatrix}q\\p\end{pmatrix}
		= f\left(t, \begin{pmatrix}q\\p\end{pmatrix}\right) 
		\coloneqq 			
		\begin{pmatrix}\bGamma(q, q)p\\ 
		-\frac{\partial \left(\frac{1}{2} p^\mathrm{T} \bGamma(q, q) p\right)}{\partial q}
		\end{pmatrix} 
	\end{equation}
	Then by the mean value theorem there exists a point $z \coloneqq (q^\ast, p^\ast)$ with $q^\ast,\ p^\ast \in \cX$ such that
	\begin{align}
		\norm{f\left(t, x\right) - f\left(t, y\right) }
		= \norm{
			\frac{\mathrm{d}}{\mathrm{d} t}f\left(t, z \right)}
		\cdot \norm{x-y} \ .
	\end{align}
	Because of the equality in \cref{eq:hamiltonian-time-derivative} we immediately arrive at the desired result:
	$\norm{f\left(t, x\right) - f\left(t, y\right) } \leq L \norm{x - y}$.
	For the detailed requirements of the Picard-Lindelöf theorem the reader is referred to \cite{arino06, tenenbaum85}.
	
	All that is left to show now is the boundedness of $\dot{q}$ and $\dot{p}$.
	That of of $\dot{q}$ is easy to see from \cref{eq:feature-hamiltonian}:
	\begin{equation}
		\norm{\dot{q}_i}_\cX \leq \norm{\psi^T(q_i)}_O \norm{\alpha}_\cF 
		\leq \sup_{x \in \cX}\norm{\psi^T(x)}_O \norm{\alpha}_\cF\ .
	\end{equation}
	Here, $\norm{\cdot}_O$ is the operator norm \cite{conway07}.
	Because of the assumption in \cref{cond:feature-condition}, $\psi$ is bounded.
	This implies that the adjoint $\psi^\T$ is, too.
	We already saw that $\alpha$ is constant.
	It remains to show the boundedness of $\dot{p_i}$.
	First consider each entry $\dot{p}_{i, j}$ of $\dot{p}_i$.
	By common differentiation rules and Cauchy-Schwarz we get
	\begin{align}
		\abs{p_{i,j}} &= \abs{\frac{\partial}{\partial q_{i, j}}\left<p_i, \psi^\T(q_i) \alpha \right> }\\
		&= \abs{\left<p_i, \frac{\partial}{\partial q_{i, j}} \psi^\T(q_i) \alpha \right> }\\
		&\leq \norm{p_i}_\cX \cdot \norm{\frac{\partial}{\partial q_{i, j}} \left(\psi^T(q_i) \alpha\right) }_\cX \ .
	\end{align}
	Observe the second factor, $\psi^\T(q_i) \alpha$. $\psi^\T$ is a matrix (actually a bounded linear operator in $L(\cF, \cX)$, but we are finite-dimensional), and $\alpha \in \cF$.
	The matrix-vector product is a continuous, bilinear operator and thus we can apply the product rule for partial derivatives.
	$\alpha$ does not depend on $q_{i, j}$, thus the equality $\frac{\partial}{\partial q_{i, j}} \left(\psi^T(q_i) \alpha\right) = \left(\frac{\partial}{\partial q_{i, j}} \psi^T(q_i)\right)  \alpha$ holds true. The partial derivative of $\psi^\T$ is again a bounded linear operator.
	\begin{equation}
		\norm{\left(\frac{\partial}{\partial q_{i, j}} \psi^T(q_i)\right)  \alpha}_\cX \leq \norm{\frac{\partial}{\partial q_{i, j}} \psi^T(q_i)}_O \norm{\alpha}_\cF \ .
	\end{equation}
	Together with the already established inequality we get
	\begin{align}
		\abs{\dot{p}_{i,j}} &\leq \norm{p_i}_\cX \cdot \norm{\frac{\partial}{\partial q_{i, j}} \left(\psi^T(q_i)\right) \alpha }_\cX\\
		&\leq  \norm{p_i}_\cX \cdot \norm{\frac{\partial}{\partial q_{i, j}} \psi^T(q_i)}_O \cdot \norm{\alpha }_\cF \ .
	\end{align}
	In total, this gives
	\begin{align}
		\norm{\dot{p}_i}^2 &= \sum_{i=1}^N \dot{p}_{i,j}^2\\
		&\leq \norm{p_i}_\cX^2 \cdot \norm{\alpha }_\cF^2 \cdot \sum_{i=1}^N  \norm{\frac{\partial}{\partial q_{i, j}} \psi^T(q_i)}_O^2\\
		& = \norm{p_i}_\cX^2 \cdot \norm{\alpha }_\cF^2 \cdot \norm{\grad \psi^T(q_i)}^2 \\
		& \leq \norm{p_i}_\cX^2 \cdot \norm{\alpha }_\cF^2 \cdot \sup_{x \in \cX}\norm{\grad \psi^T(x)}^2 \ ,
	\end{align}
	which is, of course, equivalent to 
	\begin{equation}
	\label{eq:norm-p-inequality}
		\norm{\dot{p}_i}\leq \norm{p_i}_\cX \cdot \norm{\alpha }_\cF \cdot \sup_{x \in \cX}\norm{\grad \psi^T(x)} \ .
	\end{equation}
	All factors on the right side are bounded: $p_i$ is a continuous function defined on a compact interval, $\alpha$ is constant, as we have seen, and the last term by \cref{cond:feature-condition}.
	This means $\dot{p_i}$ and therefore $\dot{p}$ are bounded, too.
	This concludes the proof.
\end{proof}

See Allansonnière for Theorem 3.7 of paper

\subsection{Geodesic Shooting}

Using the Hamiltonian representation, we can now formulate another equivalent problem.
TODO: Why geodesic shooting?

\begin{problem}
	\label{prob:geodesic-shooting}
	\begin{cases}
		\text{Minimize~}& \frac{\nu}{2} p(0)^\T \bGamma(X, X)p(0) + l(q(1), Y)\\
		\text{such that~} & p = \bGamma(q, q)^{-1}\dot{q},\ q(0) = X \text{~and~} (q,p) \text{~follow Hamilton's equations \ref{eq:hamiltonian-system}} \ .
	\end{cases}
\end{problem}

For later use, we explicitly define the objective function of this problem:
\begin{equation}
	\label{eq:geodesic-shooting-objective}
	\fV(p(0), X, Y) \coloneqq \frac{\nu}{2} p(0)^\T \bGamma(X, X)p(0) + l(q(1), Y) \ .
\end{equation}

\begin{theorem}
\label{theo:geodesic-shooting}
	$q \in C^1([0, 1], \cX^N)$ minimizes \cref{prob:cont-least-action} if and only if with $p\coloneqq \bGamma(q, q)^{-1}\dot{q}$, $p$ minimizes \cref{prob:geodesic-shooting}.
\end{theorem}
\begin{proof}
	Sketch (for now):
	\Todo{The argumentation with Lagrangian = Hamiltonian feels too simple but might be plausible}
	Let $q$ minimize \cref{prob:cont-least-action} and define $p$ as above.
	Then $q(0) = X$ and by \cref{theo:hamiltonian-dynamic} $(q, p)$ satisfy Hamilton's equations.
	We have $\cL(t, q, \dot{q}) = \cH(t, q, p)$ and know from \cref{cor:energy-preservation} that $\cH$ is constant across time.
	This implies that
	\begin{equation}
		\mathcal{A}(q) = \int_{0}^{1} \cL(t, q, \dot{q}) \mathrm{d}t 
		= \int_{0}^{1} \cH(t, q, p) \mathrm{d}t = \cH(\beta, q, p)
	\end{equation}
	for $\beta \in [0, 1]$.
	Therefore $(q, p)$ minimize $\nu \cH(\beta, q, p) + l(q(1), Y)$ and for $\beta = 0$ we get
	\begin{align}
		\nu \cH(0, q, p) + l(q(1), Y) &= \frac{\nu}{2} p(0)^\T \bGamma(q(0), q(0))p(0) + l(q(1), Y)\\
		&= \frac{\nu}{2} p(0)^\T \bGamma(X, X)p(0) + l(q(1), Y) \ .
	\end{align}
	Vice versa, if $p(0)$ minimizes $\fV$ and $(q, p)$ follow the Hamilton's equations, with the same argument, $q$ minimizes \cref{prob:cont-least-action}.
\end{proof}