An alternative formulation of Lagrangian mechanics is the Hamiltonian formalism.
On its own, it does not add anything particularly new to the Lagrangian theory but rather gives a more powerful framework to work with the already established theory.
In essence, a change of variables from $(t, q, \dot{q})$ to $(t, q, p)$ is applied through a certain kind of transformation called \emph{Legendre transformation}.
The obtained $(q, p)$ are known as the \emph{canonical variables}, specifically canonical coordinate and canonical momentum.
We will briefly cover the derivation of the Hamiltonian formulation within the scope of our application, but exclude most of the theoretical details.
For a complete formal derivation from a physical point of view, see \cite[Chapter~8]{goldstein01}.
For a more mathematical approach, consult \cite[Chapter~2]{marsden10}.

First, we define the \emph{canonical momentum}
\begin{equation}
\label{eq:canonical-momentum}
	p(t) \coloneqq \grad_{\dot{q}} \fL(t, q, \dot{q}) = \bGamma(q(t), q(t))^{-1} \dot{q}(t) \ .
\end{equation}
\sloppy{Note that just as above, this equation is actually defined element-wise: 
$p_{i, j} = \frac{\partial}{\partial \dot{q}_{i,j}} \fL(t, q(t),\dot{q}(t))$.}
Next, we define the \emph{Hamiltonian} function, often also called \emph{energy function} \cite{marsden10}:
\begin{equation}
	\fH(t, q(t), p(t)) \coloneqq p(t)^\mathrm{T}\dot{q}(t) - \fL(t, q(t), \dot{q}(t)) \ .
\end{equation}
Formally the Hamiltonian is defined as a function $\fH: [0,1] \times \cX^N \times \cX^N \rightarrow \R$; \emph{without} the connection between $p$ and $q$ defined in \cref{eq:canonical-momentum}.
This means that we mostly treat $p, q$ as independent variables unless mentioned otherwise.
In a physical context, the Hamiltonian can often be expressed as the sum of the system's kinetic and potential energies and represents the total energy of the system, hence the name "energy function".
For our application, we can calculate the Hamiltonian explicitly.
We get
\begin{align}
	\fH(t, q(t), p(t)) &=p(t)^\mathrm{T}\dot{q}(t) - \fL(t, q(t), \dot{q}(t))\\
	&= \frac{1}{2} \dot{q}(t)^\T \bGamma(q(t), q(t))^{-1} \dot{q}(t) \\
	&= \frac{1}{2} \dot{q}(t)^\T \bGamma(q(t), q(t))^{-1} \bGamma(q(t), q(t)) \bGamma(q(t), q(t))^{-1} \dot{q}(t)\\
	&= \frac{1}{2}  \left(\bGamma(q(t), q(t))^{-1} \dot{q}(t)\right)^\T \bGamma(q(t), q(t)) \bGamma(q(t), q(t))^{-1} \dot{q}(t)\\
	&= \frac{1}{2} p(t)^\T \bGamma(q(t), q(t)) p(t) \ .
\end{align}

For better readability and despite the risk of confusion, we will often omit the time dependence of $p$ and $q$ in the following.
For example, we write $\bGamma(q, q)$ instead of $\bGamma(q(t), q(t))$.

The next theorem describes the classical correspondence between Lagrangian and Hamiltonian mechanics:
\newline
\begin{theorem}
	\label{theo:hamiltonian-dynamic}
	If $q$ minimizes \cref{prob:cont-least-action} and $p = \grad_{\dot{q}} \fL(t, q, \dot{q})$, $(q, p)$ follow Hamilton's equations
	\begin{equation}
	\label{eq:hamiltonian-system}
		\begin{split}
			\dot{q} &= \grad_p \fH(q, p) = \bGamma(q, q) p\\
			\dot{p} &= -\grad_q \fH(q, p)
			= -\grad_q \left(\frac{1}{2} p^\mathrm{T} \bGamma(q, q) p\right)
		\end{split}
	\end{equation}
	with $q(0) = X$.
\end{theorem}
\begin{proof}
	Let $q$ be a minimizer of \cref{prob:cont-least-action}.
	Then $q(0) = X$ and, as we have seen, $q$ fulfills the Euler-Lagrange equations.
	The claim immediately follows from the equivalence of the Euler-Lagrange equations and Hamilton's equations under the diffeomorphism defined in \cref{eq:canonical-momentum}, as can be seen in e.g. \cite{marsden10, goldstein01}.
\end{proof}

From the fact that $\frac{\partial \fL}{\partial t} = 0$ we obtain the following corollary:
\begin{corollary}
	\label{cor:energy-preservation}
	If $p(t) = \grad_{\dot{q}} \fL(t, q, \dot{q})$, the Hamiltonian $\fH$ is constant along the trajectory $q$.
\end{corollary}
\begin{proof}
	The Lagrangian is not explicitly time dependent, that is $\frac{\partial \fL}{\partial t} = 0$.
	Write $\fL = \fL(t, q(t), \dot{q}(t))$.
	The Euler-Lagrange equations imply:
	\begin{equation}
		\frac{\mathrm{d}}{\mathrm{d} t} \fL = \sum_{i=1}^{N}\left(\frac{\partial \fL}{\partial q_i} \dot{q_i} + \frac{\partial \fL}{\partial \dot{q_i}} \ddot{q_i} \right )
		=\sum_{i=1}^{N}\left( \left( \frac{\mathrm{d}}{\mathrm{d} t} \frac{\partial \fL}{\partial \dot{q_i}} 
		\right)\dot{q_i}+ \frac{\partial \fL}{\partial \dot{q_i}} \ddot{q_i}\right )
		= \frac{\mathrm{d}}{\mathrm{d} t} \left(\sum_{i=1}^{N} \frac{\partial \fL}{\partial \dot{q_i}} \dot{q_i}\right) \ .
	\end{equation}
	From the definition of $p$ it follows that
	\begin{equation}
		0 = \frac{\mathrm{d}}{\mathrm{d} t} \left( \left(\sum_{i=1}^{N} \frac{\partial \fL}{\partial \dot{q_i}} \dot{q_i} \right) -  \fL \right) 
		= \frac{\mathrm{d}}{\mathrm{d} t} \left( \left(\sum_{i=1}^{N} p_i \dot{q_i} \right ) - \fL \right)
		= \frac{\mathrm{d}}{\mathrm{d} t} \fH \ ,
	\end{equation}
	which means that the energy function does not change over time and hence is constant.
\end{proof}
If the Hamiltonian is constant along a trajectory $q$, one often speaks of \emph{energy preservation} along $q$.

Our next objective will be to show that there exists a unique solution to the Hamiltonian system in \cref{eq:hamiltonian-system}.
If this is the case, we could solve -- or at least approximate -- the flow of the Hamiltonian system to acquire a solution to \cref{prob:cont-least-action}, as we will see in the next section.
Of course, we can only identify the flow once we have determined the value of $p(0)$ as otherwise we would be lacking an initial value.

\begin{theorem}
	\label{theo:hamiltonian-system-solution}
	There exists a unique solution $(q, p)$ for the Hamiltonian system in \cref{eq:hamiltonian-system} with $q \in C^2([0, 1], \cX^N)$ and $p \in C^1([0, 1], \cX^N)$.
\end{theorem}

\begin{proof}
	In order to show that the Hamiltonian system does have a unique solution, we want to use a global version of the Picard-Lindelöf theorem \cite[Theorem~1.2.3]{arino06}.
	If such a solution exists, it immediately follows that $q$ and $p$ are $C^1$.
	The right side of the first line in \cref{eq:hamiltonian-system} is also continuously differentiable by $t$, because $p$ is and $\Gamma$ has continuous partial derivatives.
	Thus $q$ is $C^2$.
	
	To apply Picard-Lindelöf we need the Hamiltonian system to be globally Lipschitz continuous.
	Define
	\begin{equation}
	\label{eq:hamiltonian-time-derivative}
		\begin{pmatrix}
			\dot{q}\\ \dot{p}
		\end{pmatrix}
		= f\left(t, \begin{pmatrix}q\\ p\end{pmatrix}\right) 
		\coloneqq 			
		\begin{pmatrix}
			\bGamma(q, q)p\\ 
			-\grad_q \left(\frac{1}{2} p^\mathrm{T} \bGamma(q, q) p\right)
		\end{pmatrix} \ .
	\end{equation}
	Then we have to show that $f$ is globally Lipschitz.
	We do this by proving that the Jacobian matrix $J(t, (q, p)^\T)$ of the system $f$ is bounded.
	This will do:
	Assume $J$ is bounded by $L \in \R$ and $q_1, p_1, q_2, p_2 \in \cX^N$, $x \coloneqq (q_1, p_1)^\T, y \coloneqq (q_2, p_2)^\T$.
	The mean value theorem states that there exists a point $z \coloneqq (q^\ast, p^\ast)$ with $q^\ast,\ p^\ast \in \cX^N$ such that
	\begin{align}
		\norm{f\left(t, x\right) - f\left(t, y\right)}_{\cX^{2N}}
		= \norm{J(t, z)}_O\cdot \norm{x-y}_{\cX^{2N}} \ .
	\end{align}
	Here, $\norm{\cdot}_O$ is the operator norm \cite{conway07}.
	We then immediately arrive at the desired result:
	$\norm{f\left(t, x\right) - f\left(t, y\right)}_{\cX^{2N}} \leq L \norm{x - y}_{\cX^{2N}}$.
	For the detailed requirements of the Picard-Lindelöf theorem the reader is referred to \cite{arino06}.
	
	Now, all that is left to show is the boundedness of the Jacobian $J(t, (q, p)^\T)$.
	For that, we have to prove the boundedness of four types of partial derivatives for $i, j \in [N]$:
	\begin{align}
		\frac{\partial}{\partial q_{i}} \left(\bGamma(q, q) p\right)_j \ ,\
		&\frac{\partial}{\partial p_{i}} \left(\bGamma(q, q) p\right)_j \\
		\frac{\partial}{\partial q_{i}} \left(-\grad_q \left(\frac{1}{2} p^\mathrm{T} \bGamma(q, q) p\right)\right)_j \ ,\
		&\frac{\partial}{\partial p_{i}} \left(-\grad_q \left(\frac{1}{2} p^\mathrm{T} \bGamma(q, q) p\right)\right)_j\\
	\end{align}
	The index $j$ signifies the entry at position $j$.
	We will demonstrate the boundedness for two cases.
	First, we have
	\begin{equation}
		\frac{\partial}{\partial q_{i}} \left(\bGamma(q, q) p\right)_j
		= \frac{\partial}{\partial q_{i}} \sum_{k=1}^N \Gamma(q_j, q_k) p_j
		=  \sum_{k=1}^N \left(\frac{\partial}{\partial q_{i}} \Gamma(q_j, q_k)\right) p_j \ .
	\end{equation}
	As $q$, $p$ and $\Gamma$ and its first (and second) order partial derivatives are bounded, this expression is also bounded.
	For the second case, we have
	\begin{align}
		\frac{\partial}{\partial q_{i}} \left(-\grad_q \left(\frac{1}{2} p^\mathrm{T} \bGamma(q, q) p\right)\right)_j
		&= - \frac{\partial}{\partial q_i} \frac{\partial}{\partial q_j} \left(\frac{1}{2}
		\sum_{k=1}^N\sum_{l=1}^{N} p_k \Gamma(q_k, q_l) p_l\right)\\
		&= -\frac{1}{2} \left(
		\sum_{k=1}^N\sum_{l=1}^{N} p_k \left(\frac{\partial}{\partial q_i} \frac{\partial}{\partial q_j} \Gamma(q_k, q_l)\right) p_l\right) \ .
	\end{align}
	The boundedness again follows from that of $q, p$ and $\bGamma$.
	That of the other partial derivatives is similarly easy to see.
	This shows that the Jacobian $J$ is indeed bounded and hence concludes the proof.
	
%	That of of $\dot{q}$ is easy to see from \cref{eq:feature-hamiltonian}:
%	\begin{equation}
%	\norm{\dot{q}_i}_\cX \leq \norm{\psi^T(q_i)}_O \norm{\alpha}_\cF 
%	\leq \sup_{x \in \cX}\norm{\psi^T(x)}_O \norm{\alpha}_\cF\ .
%	\end{equation}
%	Here, $\norm{\cdot}_O$ is the operator norm \cite{conway07}.
%	Because of the assumption in \cref{cond:feature-condition}, $\psi$ is bounded.
%	This implies that the adjoint $\psi^\T$ is, too.
%	We already saw that $\alpha$ is constant.
%	It remains to show the boundedness of $\dot{p_i}$.
%	First consider each entry $\dot{p}_{i, j}$ of $\dot{p}_i$.
%	By common differentiation rules and Cauchy-Schwarz we get
%	\begin{align}
%	\abs{p_{i,j}} &= \abs{\frac{\partial}{\partial q_{i, j}}\left<p_i, \psi^\T(q_i) \alpha \right> }\\
%	&= \abs{\left<p_i, \frac{\partial}{\partial q_{i, j}} \psi^\T(q_i) \alpha \right> }\\
%	&\leq \norm{p_i}_\cX \cdot \norm{\frac{\partial}{\partial q_{i, j}} \left(\psi^T(q_i) \alpha\right) }_\cX \ .
%	\end{align}
%	Observe the second factor, $\psi^\T(q_i) \alpha$. $\psi^\T$ is a matrix (actually a bounded linear operator in $L(\cF, \cX)$, but we are finite-dimensional), and $\alpha \in \cF$.
%	The matrix-vector product is a continuous, bilinear operator and thus we can apply the product rule for partial derivatives.
%	$\alpha$ does not depend on $q_{i, j}$, thus the equality $\frac{\partial}{\partial q_{i, j}} \left(\psi^T(q_i) \alpha\right) = \left(\frac{\partial}{\partial q_{i, j}} \psi^T(q_i)\right)  \alpha$ holds true. The partial derivative of $\psi^\T$ is again a bounded linear operator.
%	\begin{equation}
%	\norm{\left(\frac{\partial}{\partial q_{i, j}} \psi^T(q_i)\right)  \alpha}_\cX \leq \norm{\frac{\partial}{\partial q_{i, j}} \psi^T(q_i)}_O \norm{\alpha}_\cF \ .
%	\end{equation}
%	Together with the already established inequality we get
%	\begin{align}
%	\abs{\dot{p}_{i,j}} &\leq \norm{p_i}_\cX \cdot \norm{\frac{\partial}{\partial q_{i, j}} \left(\psi^T(q_i)\right) \alpha }_\cX\\
%	&\leq  \norm{p_i}_\cX \cdot \norm{\frac{\partial}{\partial q_{i, j}} \psi^T(q_i)}_O \cdot \norm{\alpha }_\cF \ .
%	\end{align}
%	In total, this gives
%	\begin{align}
%	\norm{\dot{p}_i}^2 &= \sum_{i=1}^N \dot{p}_{i,j}^2\\
%	&\leq \norm{p_i}_\cX^2 \cdot \norm{\alpha }_\cF^2 \cdot \sum_{i=1}^N  \norm{\frac{\partial}{\partial q_{i, j}} \psi^T(q_i)}_O^2\\
%	& = \norm{p_i}_\cX^2 \cdot \norm{\alpha }_\cF^2 \cdot \norm{\grad \psi^T(q_i)}^2 \\
%	& \leq \norm{p_i}_\cX^2 \cdot \norm{\alpha }_\cF^2 \cdot \sup_{x \in \cX}\norm{\grad \psi^T(x)}^2 \ ,
%	\end{align}
%	which is, of course, equivalent to 
%	\begin{equation}
%	\label{eq:norm-p-inequality}
%	\norm{\dot{p}_i}\leq \norm{p_i}_\cX \cdot \norm{\alpha }_\cF \cdot \sup_{x \in \cX}\norm{\grad \psi^T(x)} \ .
%	\end{equation}
%	All factors on the right side are bounded: $p_i$ is a continuous function defined on a compact interval, $\alpha$ is constant, as we have seen, and the last term by \cref{cond:feature-condition}.
%	This means $\dot{p_i}$ and therefore $\dot{p}$ are bounded, too.
%	This concludes the proof.
\end{proof}

%\subsubsection{Outlook: Hamiltonian System in Feature Space}
%Although this is not necessary for the results in other sections, we will still take a look at the Hamiltonian system with its representation in feature space, as it of interest in its own.
%
%Recall \cref{eq:kernel-feature-map}, which reads $\Gamma(x_1, x_2) = \psi^\mathrm{T}(x_1)\psi(x_2)$.
%Using this equality, we can rewrite \cref{eq:hamiltonian-system} as
%\begin{equation}
%\label{eq:feature-hamiltonian}
%	\begin{split}
%		\dot{q}_i &= \psi^\mathrm{T}(q_i) \alpha\\
%		\dot{p}_i &= -\frac{\partial}{\partial q_i} \left(p_i^\mathrm{T} \psi^\mathrm{T}(q_i) \alpha \right)\ ,
%	\end{split}
%\end{equation}
%with $\alpha \coloneqq \sum_{k=1}^{N} \psi(q_k) p_k$.
%
%A very interesting consequence of \cref{cor:feature-space-norm} is that $\alpha$ is constant:
%\begin{align}
%	\norm{\alpha}_\cF^2 &= \norm{\psi^\T(x) \alpha}_\cV^2\\
%	&= \norm{\sum_{i=1}^N \psi^\T(x) \psi(q_i)p_i}_\cV^2 \\
%	&= \norm{\sum_{i=1}^N \Gamma(x, q_i) p_i}_\cV^2\\
%	&= \left<\sum_{i=1}^N\Gamma(x, q_i)p_i, \sum_{j=1}^N \Gamma(x, q_j) p_j \right>_\cV\\
%	&= \sum_{i=1}^N \sum_{j=1}^N \left<\Gamma(x, q_i)p_i, \Gamma(x, q_j) p_j \right>_\cV\\
%	&= \sum_{i=1}^N \sum_{j=1}^N \left<p_i, \Gamma(q_i, q_j) p_j\right>_\cX\\
%	&= p^\T \bGamma(q, q) p \ .
%\end{align}
%In the second to last step we again used the reproducing property of the kernel $\Gamma$.
%This gives us $\norm{\alpha}_\cF^2 = 2 \fH(q, p)$.
%\cref{cor:energy-preservation} states that $\fH$ is constant across time, which implies that $\norm{\alpha}_\cF^2$ is, too (and thus  $\norm{\alpha}_\cF$).
