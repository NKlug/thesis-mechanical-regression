Next, we introduce block operator matrices which are closely related to the product space of a Hilbert Space $\cY$.
It allows simpler and more concise notation for what would otherwise require statements on a vector entry basis.
Although most of the time the notation will seem intuitive, we have to ensure that all transformations involving it are mathematically sound.

The $N$-fold product space of $\cY$ is defined as $\cY^N \coloneqq \bigtimes_{i=1}^N \cY$.
The inner product on $\cY$ can be extended to $\cY^N$ in a natural way:
Let $A, B \in \cY^N$.
Then $\left < A, B \right >_{\cY^N} \coloneqq \sum_{i=1}^{N} \left <A_i, B_i \right >_\cY$ fulfills the properties of an inner product on $\cY^N$ and thus makes $\cY^N$ a Hilbert space.

\begin{definition}[Block Operator Matrix]
	A \emph{Block Operator Matrix} $\mathbf{A}$ is an $(N \times N)$-matrix with elements in $L(\cY, \cY)$.
\end{definition}
Although the definition of block operator matrices does not directly involve the product space, they are closely connected, which can be seen in the following proposition.
\begin{proposition}
  $L(\cY, \cY)^{N \times N} \cong L(\cY^N, \cY^N)$.
\end{proposition}
This property is an immediate consequence of the definition of the product space.
It enables us to express the evaluation of a bounded linear operator in a fashion akin to a matrix-vector product.
Let $\mathbf{A} \in L(\cY, \cY)^{N\times N}$ be the block operator matrix associated to a linear operator $\mathbf{A}^\prime \in L(\cY^N, \cY^N)$ and let $U \in \cY^N$.
Then we can define a function
\begin{gather}
	\varphi: L(Y, Y)^{N\times N} \times \cY^N \rightarrow \cY^N \\
	\varphi(\mathbf{A}, U) = V \ ,
%	\text{, where $V$ is defined as} \\
%	V_{i, j} &\coloneqq \sum_{k = 1}^N \mathbf{A}_{i,k} \cdot U_{k, j} \ .
\end{gather}
where $V$ is defined as the vector with elements $V_{i, j} \coloneqq \sum_{k = 1}^N \mathbf{A}_{i,k} \cdot U_{k, j}$.
This is similar to the standard matrix-vector product known from linear algebra.
The only difference is that instead of multiplying elements of a field, here the operator $\mathbf{A}_{i, j}$ is evaluated at the element $U_{k, j} \in \cY$.

It already seems like block operator matrices can be treated as regular matrices, but this has not been fully proven yet.
We will show two important properties: That block operator matrices  form a ring (together with point-wise addition and matrix multiplication) and that we can in fact treat $\varphi$ like the standard matrix-vector product.

\begin{theorem}
	\label{theo:product-space-ring}
	$(L(\cY, \cY)^{N\times N}, +, \circ)$ is a ring.
\end{theorem}
\begin{proof}
	We conduct the proof by showing that $\cL(\cY^N, \cY^N) \cong \dK^{nN \times nN}$, where $\dK = \R$ or $\dK = \mathbb{C}$ is the underlying field.
	Let $\mathbf{A} \in L(\cY^N, \cY^N)$ and $n$ be the finite dimension of $\cY$.
	First, we have $\cL(\cY) \cong \dK^{n\times n}$ by mapping an operator to its transformation matrix with respect to a fixed choice of basis.
	Thus we can regard $\mathbf{A}$ as an $(N \times N)$ matrix consisting of $(n \times n)$ matrices.
	This yields an isomorphism $\phi: L(\cY^N, \cY^N) \cong \left(\dK^{n\times n}\right)^{N \times N}$.
	Now, the idea is to define $\psi:\left(\dK^{nxn}\right)^{N \times N} \rightarrow \dK^{nN \times nN}$ as the function that "flattens" the matrix.
%	This is visualized in \cref{fig:block-operator-example} and should suffice as a definition.
	One can easily check that $\psi \circ \phi$ is the desired isomorphism.
\end{proof}

%\begin{figure}
	This will be an illustration of isomorphisms in the block operator matrix example.
	\caption{And this will be the supercool caption.}
	\label{fig:block-operator-example}
\end{figure}

Similar to the isomorphism in the proof, we get an isomorphism $\cY^N \cong \dK^{nN}$, where $\dK$ is the field of $\cY$.
These new representations make it particularly easy to evaluate a bounded linear operator in $L(\cY^N, \cY^N)$.
First, map $\mathbf{A} \in L(\cY, \cY)^{N\times N}$ to $A \in \dK^{nN \times nN}$ and $\mathbf{y} \in \cY^N$ to $y \in \dK^{nN}$.
Then calculate $x = A \cdot y$ as the standard matrix-vector product and map the resulting vector $x$ back to $\mathbf{x} \in \cY^N$.
This means not only that $\varphi$ can in fact be treated like the standard matrix-vector product, but it also makes implementation easier, as we will see in \cref{sec:algorithm}.

The following corollary summarizes the section on block operator matrices.
\begin{corollary}
	\label{cor:matrix-vector-equivalence}
	\label{cor:matrix-ring-equivalence}
	Let $\cY$ be a real Hilbert space, $\delta$ the evaluation of a bounded linear operator on $\cY^N$ and $\partial$ the standard matrix-vector product. 
	Then the following diagram commutes:
	\begin{equation}
		\begin{tikzcd}
		 	L(\cY, \cY)^{N\times N} \arrow[leftrightarrow]{r}{\cong} &L(\cY^N, \cY^N) \arrow{r}{\delta} \arrow[swap, leftrightarrow]{d}{\cong} & \cY^N \arrow[leftrightarrow]{d}{\cong} \\%
			&\R^{nN \times nN} \arrow{r}{\partial}& \R^{nN} \ .
		\end{tikzcd}
	\end{equation}
\end{corollary}