This section gives a brief introduction to the block operator notation used in later sections.
It allows simpler and more concise notation for vectors like $X \in \cX^N$ and $Y \in \cY^N$ for what would otherwise require statements on an vector entry basis.
Although most of the time the notation will seem intuitive, we still have to ensure that all transformations involving it are mathematically sound.

Block operator matrices are defined on the $N$-fold product space of $\cY$.
As a short reminder, the $N$-fold product space of $\cY$ is defined as $\cY^N \coloneqq \bigtimes_{i=1}^N \cY$.
The inner product on $\cY$ can be extended to $\cY^N$ in a natural way:
Let $A, B \in \cY^N$.
Then $\left < A, B \right >_{\cY^N} \coloneqq \sum_{i=1}^{N} \left <A_i, B_i \right >$ fulfills the properties of an inner product on $\cY^N$ and thus makes $\cY^N$ a Hilbert space.

\begin{definition}[Block Operator Matrix]
	A \emph{Block Operator Matrix} $\mathbf{A}$ is an $(N \times N)$-Matrix with Elements in $L(\cY, \cY)$.
\end{definition}
\Todo{By definition of the product space L(YN, YN) is isomorphic to L(Y, Y)NxN}
Although the definition of block operator matrices does not involve the product space, they are closely connected, which can be seen in the following lemma.
\begin{lemma}
  $L(\cY, \cY)^{N \times N} \cong L(\cY^N, \cY^N)$.
\end{lemma}
The proof is quite intuitive and will not be conducted here.

For $\mathbf{A} \in L(Y, Y)^{N\times N}$ and $U \in \cY^N$ we can define a function
\begin{align}
	\sigma:~&L(Y, Y)^{N\times N} \times \cY^N \rightarrow \cY^N \\
	\sigma(\mathbf{A}, U) & = V \coloneqq \mathbf{A} \cdot U\ \text{, where $V$ is defined as} \\
	V_{i, j} &\coloneqq \sum_{k = 1}^N \mathbf{A}_{i,k} \cdot U_{k, j} \\
\end{align}
This is similar to the standard matrix-vector product known from linear algebra.
The only difference is that instead of multiplying elements of a field, we here apply the operator $\mathbf{A}_{i, j}$ to element $U_{k, j} \in \cY$.
It already seems like block operator matrices can be treated as regular matrices, but this has not been proven yet.
We will show two important properties: That block operator matrices form a ring and that we can treat $\sigma$ like the standard matrix-vector product.

\begin{theorem}
	$(L(\cY, \cY)^{N\times N}, +, \circ)$ is a  ring.
\end{theorem}
\begin{proof}
	We conduct the proof by showing that $\cL(\cY^N, \cY^N) \cong \dK^{nN \times nN}$, where $\dK = \R$ or $\dK = \mathbb{C}$ is the underlying field.
	Let $\mathbf{A} \in L(\cY^N, \cY^N)$ $n \coloneqq \mathbf{dim}(\cY) < \infty$ (we required $\cY$ to be finite).
	First, we have $\cL(\cY) \stackrel{\phi}{\cong}\R^{n\times n}$ by mapping an operator to its transformation matrix with respect to a fixed choice of basis.
	Thus we can regard $\mathbf{A}$ as a big $(N \times N)$ matrix of $(n \times n)$ matrices.
	Actually, $\phi$ induces an isomorphism $L(\cY^N, \cY^N) \cong \left(\dK^{nxn}\right)^{N \times N}$.
	Now, the idea is to define $\xi:\left(\dK^{nxn}\right)^{N \times N} \rightarrow \dK^{nN \times nN}$ as the function that "flattens" this matrix.
	This is visualized in \cref{fig:block-operator-example} and should suffice as a definition.
	One can easily check that $\xi \circ \phi$ is the desired isomorphism.
\end{proof}

\begin{figure}
	This will be an illustration of isomorphisms in the block operator matrix example.
	\caption{And this will be the supercool caption.}
	\label{fig:block-operator-example}
\end{figure}

As a consequence of the proof, we get the following corollary.
\begin{corollary}
	\label{cor:matrix-ring-equivalence}
	If $\cY$ is a real Hilbert space, $\cL(\cY, \cY) \cong \R^{nN \times nN}$.
\end{corollary}
Similarly, we get an isomorphism $\cY^N \cong \R^nN$.
These new representations of $\mathbf{A}$ and $U$ make it particularly easy to compute $\sigma$:
Simply map $\mathbf{A}$ to $A \in \R^{nN \times nN}$ and $\mathbf{y}$ to $y \in \R^{nN}$, calculate $A \cdot y = x$ as the standard matrix-vector product and map the resulting vector $x$ back to $\mathbf{x} \in\cY^N$.
This is summed up in the following diagram

\begin{corollary}
	\label{cor:matrix-vector-equivalence}
	super cool commuting diagram.
\end{corollary}