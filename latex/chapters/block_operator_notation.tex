Later on, it would be quite cumbersome to always treat the $X_i$ separately.
Hence we will now introduce block operator syntax which will enable us to write equations more concisely.
As in later parts we will almost exclusively rely on this notation, we first have to make sure that all used syntax is mathematically correct.
In the following we will define the product space of $\cY$, some useful notation and assure that all future calculations involving that new notation are mathematically sound.

Let $\cY^N \coloneqq \bigtimes_{i=1}^N \cY$ be the $N$-fold product space and $A, B \in \cY^N$.
Defining $\left < A, B \right >_{\cY^N} \coloneqq \sum_{i=1}^{N} \left <A_i, B_i \right >$ fulfills the properties of an inner product on $\cY^N$ and thus makes $\cY^N$ a Hilbert space itself.


\begin{definition}[Block Operator Matrix]
	A \emph{Block Operator Matrix} $\mathbf{A}$ is an $(N \times N)$-Matrix with Elements in $\cL(Y, Y)$.
\end{definition}

\begin{definition}[$\cY^N$]
	content... 
\end{definition}
Explain: We can express the corresponding inner product as matrix multiplication.

\begin{theorem}
	$(\cL(\cY^N, \cY^N), +, \circ)$ is a  ring.
\end{theorem}
\begin{proof}
	We conduct the proof by showing that $\cL(\cY^N, \cY^N) \cong \R^{nN \times nN}$.
\end{proof}

As a consequence of the proof, we get the following corollary.
\begin{corollary}
	\label{cor:matrix-ring-equivalence}
	$\cL(\cY^N, \cY^N) \cong \R^{nN \times nN}$.
\end{corollary}

\begin{corollary}
	\label{cor:matrix-vector-equivalence}
	Block-Operator Matrix times vector is isomorphic to standard matrix times vector.
\end{corollary}

\begin{figure}
	This will be an illustration of isomorphisms in the block operator matrix example.
	\caption{And this will be the supercool caption.}
	\label{fig:block-operator-example}
\end{figure}

\begin{example}
	Let $\cY = \R^n$.
	\Todo{Discard Example, make it lemma/theorem.
		Replace $\R$ by general field, as Hilbert Spaces are vector spaces,
		make first isomorphism monoid homomorphism, maybe even ring homomorphism.
		This means we can work with $\mathbf{K}$ just like a regular matrix.
		Check if $\mathbf{K}$ really is symmetric or not just hermitian.
		Also prove that there is an inverse under certain circumstances.}
	Then we have $\cL(\cY) \cong \R^{n\times n}$, which means applying a (bounded) linear operator to a $y \in \cY$ can be expressed as matrix-vector multiplication.
	We now claim that in this case $\cL(\cY^N) \cong \R^{nN \times nN}$ and that the computation of $\mathbf{A}\cdot \mathbf{y}$ can be expressed as a standard matrix-vector product.
	For the first part, let $\mathbf{A} \in \cL(\cY^N)$ and $\mathbf{y} \in \cY^N$. 
	Write $\mathbf{A} = (\mathbf{A}_{i, j})$ with $\mathbf{A}_{i,j} \in \cL(\cY)$. 
	We obtain the corresponding $(nN \times nN)$-matrix by first applying the canonical isomorphism $\cL(\cY) \cong \R^{n\times n}$ to each $\mathbf{A}_{i,j}$ and then "flattening" the result.
	For the second part, we first have to consider another isomorphism $\cY^N \cong \R^{nN}$ which is given by consecutively pasting the elements of $\cY^N$ into one vector of size $nN$.
	\cref{fig:block-operator-example} illustrates both isomorphisms.
	
	These new representations of $\mathbf{A}$ and $\mathbf{y}$ make it particularly easy to compute $\mathbf{A}\cdot \mathbf{y} = \mathbf{x}$:
	Simply map $\mathbf{A}$ to $A \in \R^{nN \times nN}$ and $\mathbf{y}$ to $y \in \R^{nN}$, calculate $A \cdot y = x$ as the standard matrix-vector product and map the resulting vector $x$ back to $\mathbf{x} \in\cY^N$.	
	\Todo{Image: Block Operator Matrix -> "flattened" Matrix}
\end{example}

With $K$ being the Kernel associated with the RKHS $\mathcal{H}$ and $A, B \in \cY^N$ we will write $\mathbf{K}(A, B)$ for the matrix with the entries $K(A_i, B_j)$ at position $(i, j)$.

