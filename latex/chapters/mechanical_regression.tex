\section{Mechanical Regression}

In \cref{sec:neural-networks} we have seen that Neural Networks are a prominent way of approximating a solution to the Supervised Learning problem.
ResNets in particular enabled the use of very deep architectures.
In this section, we take a closer look at ResNets and derive their continuous limit as the number of residual blocks tends towards infinity.
On the way of doing so, we show that the ResNet regression approach can be regarded as a discrete mechanical system following Hamilton's stationary action principle -- thus the term \emph{Mechanical Regression}.
In the limit, this system converges to a continuous version of the same principle, which proves to be equivalent to what in image registration is known as geodesic shooting \cite{allassonniere05}.

\citet{owhadi20} compares these results to the problem formulations used in image shape analysis, computational anatomy and image registration \cite{bibid}.
He views the continuous solution as a generalization of image registration and calls it \emph{idea registration}, arguing that just as in the former, data points are aligned through various transformations in their respective feature spaces.
The difference lies in the fact that in idea registration the data points can be arbitrary rather than mere landmarks and the spaces can be high dimensional (compared to the two or three dimensional spaces images lie in).

This section essentially follows Chapter 3 of \cite{owhadi20} and aims to provide a profound and comprehensive explanation of Mechanical Regression and how it can be interpreted as the continuous limit of ResNets.

\subsection{Modeling Residual Neural Networks}

Recall the Supervised Learning problem: Given training data $X$ and $Y$ consisting of $X_i \in \cX$, $Y_i \in \cY$ and $f^\dagger(X) = Y$, approximate $f^\dagger$.
On possibility to solve this problem are Residual Neural Networks, which approximate the target function $f^\dagger$ through a series of residual blocks.
Mathematically, we can write
\begin{equation}
	f^\ast \coloneqq f \circ \Phi_L 
\end{equation}
for the approximate solution, where
\begin{equation}
\Phi_L \coloneqq \phi_L \circ \phi_{L-1} \circ \ldots \circ \phi_1.
\end{equation} 
$\Phi_L$ is the composition of $L$ Residual Blocks, that is, functions $\phi_k = I + v_k$.
Here, the $v_k$ are functions mapping the input space $\cX$ onto itself and $I$ is the identity operator.
Thus, we can regard $\Phi_L$ as a large deformation of $\cX$.
$f: \cX \to \cY$ is a function mapping the deformed space to the target space $\cY$.

By default, the $v_k$ and $f$ can be arbitrary functions.
However, it poses a challenge to not only approximate the target with any series of functions, but with such that generalize well, meaning that they also perform well on data other than the training data $(X, Y)$.
Thus, it is common in Machine Learning \cite{goodfellow16} to apply some form of regularization.
One popular approach, which we will also use here, is to apply penalties to the parameter's norms -- which are, in this case, the $v_k$ and $f$.

We still need to define appropriate spaces for the functions $v_k$ and $f$.
As we need norms for both, we introduce two RKHSs:
$\cV \subseteq \{\cX \rightarrow \cX\}$ and $\cH \subseteq \{\cX \rightarrow \cY\}$.
We want $v_k \in \cV$ for all $k$ and $f \in \cH$.
By \cref{theo:kernel-for-rkhs} there is a Kernel associated with each RKHS.
Let $\Gamma$ be the Kernel associated with $\cV$ and $K$ that of $\cH$.
Then we can identify $f$ and the $v_k$ as solutions to the following problem:
\begin{problem}
	\label{prob:min-v-f}
	\begin{cases}
		\text{Minimize~} & \nu \cdot \frac{L}{2} \sum_{k=1}^{L} \norm{v_k}_\cV^2
		+ \lambda \norm{f}_\cH^2 
		+ l((f \circ \Phi_L)(X), Y) \\
		\text{such that~} & v_1, \ldots, v_L \in \cV, f \in \cH \ .
	\end{cases}
\end{problem}
Here, $\nu$ and $\lambda$ are strictly positive balancing parameters.
$l$ is a (positive) loss measuring the similarity of the predicted outputs, that is, the image of $X$ under $f \circ \Phi_L$.
Note that, just as intended, $v_k$ and $f$ with large norms are penalized.

Utilizing the Ridge Regression Loss $l_R$ (\cref{eq:ridge-regression-loss}), we can rewrite the above minimization problem as
\begin{problem}
	\begin{cases}
		\text{Minimize~} & \nu \cdot \frac{L}{2} \sum_{k=1}^{L} \norm{v_k}_\cV^2
		+ l_R(\Phi_L(X), Y) \\
		\text{such that~} & v_1, \ldots, v_L \in \cV\ .
	\end{cases}
\end{problem}
By hiding the regularity of $f$ in the loss we can, for now, focus exclusively on the functions $v_k$.
For our calculations we will only assume that $l$ is a positive, continuous loss function.
If desired we can later balance the $v_k$ with the regularity of $f$ by choosing the loss appropriately.

In the calculations, we will work under the following conditions.
\begin{condition}
	\label{cond:feature-condition}\mbox{}
	\vspace*{-\parsep}
	\vspace*{-\baselineskip}
	\begin{enumerate}
		\item There exist a finite dimensional feature space and map $\cF$ and $\psi$ such that $\psi$ and its first and second order partial derivatives are continuous and uniformly bounded.
		\item There exists an $r > 0$ such that $\forall z \in \cX^N:~z^\T \bGamma(X, X) z \geq r z^\T z$.
		\item $\cX$ and $\cY$ are finite-dimensional.
		\item $l: \cX^N \times \cY^N \rightarrow$ is a positive and continuous loss.
	\end{enumerate}
\end{condition}
Note that the first condition implies that the function $(x_1, x_2) \rightarrow \Gamma(x_1, x_2)$ and its first and second order partial derivatives are continuous and uniformly bounded.
This follows immediately from the fact that we can write $\Gamma(x_1, x_2) = \psi^\T(x_1)\psi(x_2)$.
Furthermore, we have the following equivalence.
\begin{lemma}
	\cref{cond:feature-condition} (2) is equivalent to the non-singularity of $\bGamma(X, X)$.
\end{lemma}
\begin{proof}
	If $\bGamma(X, X)$ was singular, there would exist a vector $z \in \cX^N$ with $\norm{z}_{\cX^N} > 0$ and $\bGamma(X, X) z = 0$, implying $z^\T \bGamma(X, X)z = 0$. 
	A contradiction.
	Conversely, if there exists a $0 \neq z \in \cX^N$ such that $z^\T \bGamma(X, X) z < \epsilon z^\T z$ for all $\epsilon > 0$, then $\bGamma(X, X) z = 0$.
\end{proof}
Ideally, we would not need additional conditions like those introduced above.
They are, however, automatically satisfied in most real world examples and ease calculations and proofs considerably.

\subsection{Discrete Stationary Action Principle}

Our main goal will now be to show that the minimization problem \ref{prob:min-v-f} can in fact be considered as a discrete solver of a mechanical system.
In order to do that, we first reformulate the problem by introducing additional variables $q_{i,j}$ with $2 \leq i \leq L+1$, $1 \leq j \leq N$:
\begin{align}
	q_{1, j} &\coloneqq X_j \, \\
	q_{i, j} &\coloneqq (\phi_{i-1} \circ \ldots \circ \phi_1) (X_j) \ .
\end{align}
Write $q_i$ for the vector with entries $q_{i,j}$ and hence we get $q_1 = X$ and $q_i = \phi_i(q_{i-1})$.
Regard the following minimization problem:
\begin{problem}
	\label{prob:min-q}
	\begin{cases}
		\text{Minimize~} & \nu \cdot \frac{1}{2} \sum_{k=1}^{L} \left(\frac{q_{k+1} - q_k}{\Delta t}\right)^\mathrm{T} \bGamma(q_k, q_k)^{-1} \left(\frac{q_{k+1} - q_k}{\Delta t}\right) \Delta t+ l(q_{L+1}, Y) \\
		\text{such that~} & q_1 = X \text{~and~} q_2, \ldots, q_{L+1} \in \cX^N \ .
	\end{cases}
\end{problem}
We will now prove that this problem is in fact equivalent to \cref{prob:min-v-f} and also that we can obtain a closed form expression for the $v_k$ from its optimal points.

\begin{theorem}
	\label{theo:v-q-problem-equivalence}
	$v_1, \ldots, v_L \in \cV$ minimize \cref{prob:min-v-f} if and only if $q_1, \ldots, q_{L+1} \in \cX^N$ minimize \cref{prob:min-q} and $v_k(x) = \bGamma(x, q_k)^\mathrm{T}\bGamma(q_k, q_k)^{-1} (q_{k+1} - q_k)$.
\end{theorem}
\begin{proof}
	Using $q_k$ define above we have $v_k(q_k) = q_{k+1} - q_k$ for all $k \in [L]$.
	We can rewrite \cref{prob:min-v-f} as
	\begin{problem}
		\label{prob:min-q-v}
		\begin{cases}
			\text{Minimize~} & \nu \cdot \frac{L}{2} \sum_{k=1}^{L} \norm{v_k}_\cV^2
			+ l(q_{L+1}, Y) \\
			\text{such that~} & v_1, \ldots, v_L \in \cV,\ \forall k \in [L]: v_k(q_k) = q_{k+1} - q_k, \\
			& q_1 = X \text{~and~} q_2, \ldots, q_{L+1} \in \cX^N \ .
		\end{cases}
	\end{problem}
	Here we just added the $q_k$ as additional variables but then constrained them to certain values, concretely $q_{k+1} = q_k + v_k(q_k) = \phi_k(q_k)$, with $q_1 = X$.
	Recursively we get $q_{L+1} = \Phi_L(X)$ which means the target function remains the same as in \cref{prob:min-v-f}.
	Note that even though we seemingly added constraints to the domains of the $v_k$ they are still the same: We just have to choose $q_{k+1}$ accordingly (which we can, as they are unconstrained).
	
	We will now derive closed form expressions for the $v_k$ as as functions of the $q_k$.
	For that, let $q_k \in \cX^N$, $2 \leq k \leq N$ be arbitrary but fixed .
	This means $l(q_{L+1}, Y)$ is constant and $v_k \in V_q(k) \coloneqq \{v \in \cV~|~ v_k(q_k) = q_{k+1} - q_k\}$.
%	These sets remain convex: Let $\lambda \in [0, 1]$ and $v_{k, 1}, v_{k, 2} \in V_{k, q}$, then we get
%	\begin{align}
%		(\lambda v_{k, 1} + (1 - \lambda) v_{k, 2})(q_k) &= (\lambda v_{k, 1})(q_k) + ((1 - \lambda) v_{k, 2})(q_k)\\
%		&=\lambda(q_{k+1} - q_k) + (1 - \lambda)(q_{k+1} - q_k)\\
%		&= q_{k+1} - q_k \ .
%	\end{align}
	We are left with the minimization of the following problem:
	\begin{problem}
		\begin{cases}
			\text{Minimize} &\sum_{k=1}^L \norm{v_k}_\cV^2\\
			\text{such that} & \forall k \in [L]: v_k \in V_q(k) \ ,
		\end{cases}
	\end{problem}
	where each summand is bounded below by $0$.
	One can easily verify that minimization problems of this type have a global minimum at any point where each of the summands is minimal.
	The boundedness ensures that at least one such minimum exists.

	Thus, we have to find the minima of $\min\{\norm{v_k}_\cV^2\ |\ v_k \in \cV,\ v_k(q_k) = q_{k+1} - q_k\} = l_{\text{OR}}(q_k, q_{k+1} - q_k)$.
	Here, $l_{\text{OR}}$ is the optimal recovery loss defined in \cref{eq:optimal-recovery-loss}.
	From that section we conveniently get the following representations for the minimal $v_k$ and the target value:
	\begin{align}
		v_k & = \bGamma(x, q_k)^\mathrm{T}\bGamma(q_k, q_k)^{-1} (q_{k+1} - q_k) \ ,\\
		\norm{v_k}_\cH^2 &= (q_{k+1} - q_k)^\mathrm{T} \bGamma(q_k, q_k)^{-1} (q_{k+1} - q_k) \ .
	\end{align}
	In these expressions $\bGamma(q_k, q_k)$ is the block operator matrix with entries $\Gamma(q_{k,i}, q_{k, j})$ and $\bGamma(x, q_k)$ the vector $(\Gamma(x, q_{k, i}))_{i \in [N]}$.
	Having now computed optimal $v_k$ -- or rather their squared $\cH$-norms -- as a function of $q_2, \ldots, q_{L+1}$ we can reformulate \cref{prob:min-q-v} without the $v_k$ as variables.
	Define $\Delta t \coloneqq \frac{1}{L}$.
	We get:
	\begin{problem}
		\begin{cases}
			\text{Minimize~} & \nu \cdot \frac{1}{2} \sum_{k=1}^{L}  
			\left(\frac{q_{k+1} - q_k}{\Delta t}\right)^\mathrm{T} \bGamma(q_k, q_k)
			\left(\frac{q_{k+1} - q_k}{\Delta t}\right) \cdot \Delta t
			+ l(q_{L+1}, Y) \\
			\text{such that~} & q_1 = X \text{~and~} q_2, \ldots, q_{L+1} \in \cX^N \ .
		\end{cases}
	\end{problem}
	As all steps and transformations hold true in both ways, this concludes the proof.
\end{proof}

When comparing the target function of \cref{prob:min-q} with the results found in discrete mechanics literature \cite[~Chapter VI.6.2]{hairer06}, in particular with those for the discretized stationary action principle, it turns out that there are strong similarities.
We will not explore these similarities in depth for the discrete case but rather take a closer look at the continuous counterpart in the next section.

\subsection{Stationary Action Principle}

Take a closer look at the first term of the objective function in \cref{prob:min-q} (the parameter $\nu$ has been deliberately omitted here):
\begin{equation}
	\label{eq:discrete-lagrangian}
	\frac{1}{2} \sum_{k=1}^{L} \left(\frac{q_{k+1} - q_k}{\Delta t}\right)^\mathrm{T} \mathbf{\Gamma}(q_k, q_k)^{-1} \frac{q_{k+1} - q_k}{\Delta t} \Delta t \ .
\end{equation}
With some imagination this very much looks like an approximation of the integral of some continuous function with a step function using $L$ intervals of width $\Delta t$.
This would require the sequence $q_k$ to be a discrete equidistant sampling of a function $q: A \rightarrow \cX^N$, where $A$ is an interval in $\R$.
Without loss of generality we choose $A = [0, 1]$ and write $q_k = q(k \Delta t)$.
\Todo{Nice wording: The dot signifying the time derivative d dt	}
This way, $\frac{q_{k+1} - q_k}{\Delta t}$ can be interpreted as a forward differential quotient approximating the derivative $\dot{q}$ on at the point $k \Delta t = \frac{k}{L}$ and we get the approximation
\begin{equation}
	\left(\frac{q_{k+1} - q_k}{\Delta t}\right)^\mathrm{T} \mathbf{\Gamma}(q_k, q_k)^{-1} \frac{q_{k+1} - q_k}{\Delta t} 
	\approx \dot{q}\left(\frac{k}{L}\right)^\mathrm{T} \mathbf{\Gamma}\left(q\left(\frac{k}{L}\right), q\left(\frac{k}{L}\right)\right)^{-1}\dot{q}\left(\frac{k}{L}\right)
\end{equation}
This leads us to the obvious question if the solutions of the problem in \cref{prob:min-q} are also approximating the solutions of a similar continuous problem.
As it turns out, they do.
Consider 
\begin{equation}
\label{eq:action}
	\mathcal{A}(q) \coloneqq \int_{0}^{1} \fL(q, \dot{q}, t) \mathrm{d}t \ ,
\end{equation}
with

\begin{equation}
\label{eq:lagrangian}
	\fL(q, \dot{q}, t) \coloneqq \frac{1}{2} \dot{q}(t)^\mathrm{T} \mathbf{\Gamma}(q(t), q(t))^{-1} \dot{q}(t) \ .
\end{equation}
\Todo{Mention difference and similarity between physical and mathematical approach in definition of Lagrangian}
Here, the Lagrangian is a quadratic form.
In a physical interpretation, the Lagrangian above resembles that of a system of free moving particles who don't interact with each other.
It usually represents the kinetic energy $T$ of the system.
This interpretation would imply that in our case, the potential energy $U$ of the system is $0$.

As a result, we get the following theorem.
\begin{theorem}
	The minimal value of \cref{prob:min-q} converges towards the minimal value of             
	\info{Theorem 3.11 from paper} 
	\begin{problem}
	\label{prob:cont-least-action}
		\begin{cases}
			\text{Minimize~} & \nu \mathcal{A}(q) + l(q(1), Y)\\
			\text{such that~} & q \in C^1([0,1], \cX^N),\ q(0) = X \ .
		\end{cases}
	\end{problem}
\end{theorem}
As usual, $C^1([0,1], \cX^N)$ is the set of continuously differentiable functions $q: [0, 1] \rightarrow \cX^N$.
We will save the proof for later.

\Todo{Link between mathematical approach, i.e. variational calculus, and physical approach: Physical "interpretation" can be derived from mathematical approach + some axioms}
Equations \ref{eq:action} and \ref{eq:lagrangian} closely resemble characteristic equations from the framework of Lagrangian and Hamiltonian mechanics.
Within that framework, $q$ is the trajectory of the particle, $\fL$ the \emph{Lagrangian function} and $\mathcal{A}$ the \emph{Action}.
In our case, each training sample constitutes a particle, therefore $q$ -- a vector -- describes the trajectories of not one, but $N$ particles.
We will now explore if the rich theory of Lagrangian physics can help us analyze our problem.

The action is the integral over the Lagrangian between two points in time $t_1, t_2$.
Here, we chose $t_1 = 0$ and $t_2 = 1$.
One core principle in Lagrangian Mechanics is \emph{Hamilton's Principle}, also called the \emph{Principle of Least Action}.
It states that
\Todo{Citation needed!}
the evolution of $q$ over time between two states $q(t_1)$ and $q(t_2)$ at times $t_1$ and $t_2$ is a stationary point of the action $\mathcal{A}$.
We do not have the time to dive deeper into the underlying theory, but the interested reader is referred to \Todo{References, one of them could be Feynman lectures 2 chapter 19}.
Note that "Principle of Least Action" is a misnomer and should rather be "Principle of Stationary Action".


Hamilton's principle has a strong physical background:
The true evolution of a system $q$ between two states $q(t_1)$ and $q(t_2)$ is a stationary point of the action.
\Todo{Why doesn't the endpoint have to be fixed? This is e.g. mentioned in Classical Mechanics by Goldstein p45}

The path of least action need not always be the path for which the action becomes minimal, but rather stationary.
However, if we can show that there exists a minimum of \cref{prob:cont-least-action}, this will, \Todo{Why exactly -- do we have some similar theorem like "partial derivative zero implies extremum?"}by ..., be a point of stationary action

Let $q(1)$ be arbitrary but fixed.
Then a minimizer $q^\ast$ of \cref{prob:cont-least-action} with $q^\ast(1) = q(1)$ is a minimizer of $\mathcal{A}(q)$.
By [Kielhöfer - Variationsrechnung, Satz 1.4.2] it follows that $q^\ast$ fulfills the Euler-Lagrange equation:
\begin{equation}
	\frac{\mathrm{d}}{\mathrm{d}t} \frac{\partial \fL}{\partial \dot{q}} - \frac{\partial \fL}{\partial q} = 0 \ .
\end{equation}
Bear in mind that $q$ is a vector.
So to be precise, we would have to write $N$ Euler-Lagrange equations, one for each $q_i$ (just imagine $q_i$ and $\dot{q}_i$ in the equation above).
Nonetheless, in the following we can always treat all $q_i$ the same.
So for the sake of readability and convenience we will stick to the vector notation, which together with block operator matrices allows concise formulation.
The Euler-Lagrange equation with the Lagrangian from \cref{eq:lagrangian} reads
\begin{equation}
\label{eq:concrete-lagrangian}
	\frac{\mathrm{d}}{\mathrm{d}t} \left(\mathbf{\Gamma}(q, q)^{-1} \dot{q} \right)
	= \frac{\partial}{\partial q} \left(\frac{1}{2} \dot{q}^\mathrm{T} \mathbf{\Gamma}(q, q)^{-1} \dot{q}\right) \ .
\end{equation}

As this is true for all endpoints $q(1)$, it especially holds true for the value for which the whole expression is minimal.

\subsection{Hamiltonian Representation}

An alternative formulation of Lagrangian mechanics is the Hamiltonian formalism.
On its own, it does not add anything particularly new to the Lagrangian theory but rather gives a more powerful framework to work with the already established theory.
In essence, a change of variables from $(t, q, \dot{q})$ to $(t, q, p)$ is applied through a certain kind of transformation called \emph{Legendre transformation}.
The obtained $(q, p)$ are known as the \emph{canonical variables}, specifically canonical coordinate and canonical momentum.
We will briefly cover the derivation of the Hamiltonian formulation within the scope of our application, but exclude most of the theoretical details.
For a complete formal derivation from a physical point of view, see \cite[Chapter~8]{goldstein01}.
For a more mathematical approach, consult \cite[Chapter~2]{marsden10}.

First, we define the \emph{canonical momentum}
\begin{equation}
\label{eq:canonical-momentum}
	p(t) \coloneqq \grad_{\dot{q}} \fL(t, q, \dot{q}) = \bGamma(q(t), q(t))^{-1} \dot{q}(t) \ .
\end{equation}
\sloppy{Note that just as above, this equation is actually defined element-wise: 
$p_{i, j} = \frac{\partial}{\partial \dot{q}_{i,j}} \fL(t, q(t),\dot{q}(t))$.}
Next, we define the \emph{Hamiltonian} function, often also called \emph{energy function} \cite{marsden10}:
\begin{equation}
	\fH(t, q(t), p(t)) \coloneqq p(t)^\mathrm{T}\dot{q}(t) - \fL(t, q(t), \dot{q}(t)) \ .
\end{equation}
Formally the Hamiltonian is defined as a function $\fH: [0,1] \times \cX^N \times \cX^N \rightarrow \R$; \emph{without} the connection between $p$ and $q$ defined in \cref{eq:canonical-momentum}.
This means that we mostly treat $p, q$ as independent variables unless mentioned otherwise.
In a physical context, the Hamiltonian can often be expressed as the sum of the system's kinetic and potential energies and represents the total energy of the system, hence the name "energy function".
For our application, we can calculate the Hamiltonian explicitly.
We get
\begin{align}
	\fH(t, q(t), p(t)) &=p(t)^\mathrm{T}\dot{q}(t) - \fL(t, q(t), \dot{q}(t))\\
	&= \frac{1}{2} \dot{q}(t)^\T \bGamma(q(t), q(t))^{-1} \dot{q}(t) \\
	&= \frac{1}{2} \dot{q}(t)^\T \bGamma(q(t), q(t))^{-1} \bGamma(q(t), q(t)) \bGamma(q(t), q(t))^{-1} \dot{q}(t)\\
	&= \frac{1}{2}  \left(\bGamma(q(t), q(t))^{-1} \dot{q}(t)\right)^\T \bGamma(q(t), q(t)) \bGamma(q(t), q(t))^{-1} \dot{q}(t)\\
	&= \frac{1}{2} p(t)^\T \bGamma(q(t), q(t)) p(t) \ .
\end{align}

For better readability and despite the risk of confusion, we will often omit the time dependence of $p$ and $q$ in the following.
For example, we write $\bGamma(q, q)$ instead of $\bGamma(q(t), q(t))$.

The next theorem describes the classical correspondence between Lagrangian and Hamiltonian mechanics:
\newline
\begin{theorem}
	\label{theo:hamiltonian-dynamic}
	If $q$ minimizes \cref{prob:cont-least-action} and $p = \grad_{\dot{q}} \fL(t, q, \dot{q})$, $(q, p)$ follow Hamilton's equations
	\begin{equation}
	\label{eq:hamiltonian-system}
		\begin{split}
			\dot{q} &= \grad_p \fH(q, p) = \bGamma(q, q) p\\
			\dot{p} &= -\grad_q \fH(q, p)
			= -\grad_q \left(\frac{1}{2} p^\mathrm{T} \bGamma(q, q) p\right)
		\end{split}
	\end{equation}
	with $q(0) = X$.
\end{theorem}
\begin{proof}
	Let $q$ be a minimizer of \cref{prob:cont-least-action}.
	Then $q(0) = X$ and, as we have seen, $q$ fulfills the Euler-Lagrange equations.
	The claim immediately follows from the equivalence of the Euler-Lagrange equations and Hamilton's equations under the diffeomorphism defined in \cref{eq:canonical-momentum}, as can be seen in e.g. \cite{marsden10, goldstein01}.
\end{proof}

From the fact that $\frac{\partial \fL}{\partial t} = 0$ we obtain the following corollary:
\begin{corollary}
	\label{cor:energy-preservation}
	If $p(t) = \grad_{\dot{q}} \fL(t, q, \dot{q})$, the Hamiltonian $\fH$ is constant along the trajectory $q$.
\end{corollary}
\begin{proof}
	The Lagrangian is not explicitly time dependent, that is $\frac{\partial \fL}{\partial t} = 0$.
	Write $\fL = \fL(t, q(t), \dot{q}(t))$.
	The Euler-Lagrange equations imply:
	\begin{equation}
		\frac{\mathrm{d}}{\mathrm{d} t} \fL = \sum_{i=1}^{N}\left(\frac{\partial \fL}{\partial q_i} \dot{q_i} + \frac{\partial \fL}{\partial \dot{q_i}} \ddot{q_i} \right )
		=\sum_{i=1}^{N}\left( \left( \frac{\mathrm{d}}{\mathrm{d} t} \frac{\partial \fL}{\partial \dot{q_i}} 
		\right)\dot{q_i}+ \frac{\partial \fL}{\partial \dot{q_i}} \ddot{q_i}\right )
		= \frac{\mathrm{d}}{\mathrm{d} t} \left(\sum_{i=1}^{N} \frac{\partial \fL}{\partial \dot{q_i}} \dot{q_i}\right) \ .
	\end{equation}
	From the definition of $p$ it follows that
	\begin{equation}
		0 = \frac{\mathrm{d}}{\mathrm{d} t} \left( \left(\sum_{i=1}^{N} \frac{\partial \fL}{\partial \dot{q_i}} \dot{q_i} \right) -  \fL \right) 
		= \frac{\mathrm{d}}{\mathrm{d} t} \left( \left(\sum_{i=1}^{N} p_i \dot{q_i} \right ) - \fL \right)
		= \frac{\mathrm{d}}{\mathrm{d} t} \fH \ ,
	\end{equation}
	which means that the energy function does not change over time and hence is constant.
\end{proof}
If the Hamiltonian is constant along a trajectory $q$, one often speaks of \emph{energy preservation} along $q$.

Our next objective will be to show that there exists a unique solution to the Hamiltonian system in \cref{eq:hamiltonian-system}.
If this is the case, we could solve -- or at least approximate -- the flow of the Hamiltonian system to acquire a solution to \cref{prob:cont-least-action}, as we will see in the next section.
Of course, we can only identify the flow once we have determined the value of $p(0)$ as otherwise we would be lacking an initial value.

\begin{theorem}
	\label{theo:hamiltonian-system-solution}
	There exists a unique solution $(q, p)$ for the Hamiltonian system in \cref{eq:hamiltonian-system} with $q \in C^2([0, 1], \cX^N)$ and $p \in C^1([0, 1], \cX^N)$.
\end{theorem}

\begin{proof}
	In order to show that the Hamiltonian system does have a unique solution, we want to use a global version of the Picard-Lindelöf theorem \cite[Theorem~1.2.3]{arino06}.
	If such a solution exists, it immediately follows that $q$ and $p$ are $C^1$.
	The right side of the first line in \cref{eq:hamiltonian-system} is also continuously differentiable by $t$, because $p$ is and $\Gamma$ has continuous partial derivatives.
	Thus $q$ is $C^2$.
	
	To apply Picard-Lindelöf we need the Hamiltonian system to be globally Lipschitz continuous.
	Define
	\begin{equation}
	\label{eq:hamiltonian-time-derivative}
		\begin{pmatrix}
			\dot{q}\\ \dot{p}
		\end{pmatrix}
		= f\left(t, \begin{pmatrix}q\\ p\end{pmatrix}\right) 
		\coloneqq 			
		\begin{pmatrix}
			\bGamma(q, q)p\\ 
			-\grad_q \left(\frac{1}{2} p^\mathrm{T} \bGamma(q, q) p\right)
		\end{pmatrix} \ .
	\end{equation}
	Then we have to show that $f$ is globally Lipschitz.
	We do this by proving that the Jacobian matrix $J(t, (q, p)^\T)$ of the system $f$ is bounded.
	This will do:
	Assume $J$ is bounded by $L \in \R$ and $q_1, p_1, q_2, p_2 \in \cX^N$, $x \coloneqq (q_1, p_1)^\T, y \coloneqq (q_2, p_2)^\T$.
	The mean value theorem states that there exists a point $z \coloneqq (q^\ast, p^\ast)$ with $q^\ast,\ p^\ast \in \cX^N$ such that
	\begin{align}
		\norm{f\left(t, x\right) - f\left(t, y\right)}_{\cX^{2N}}
		= \norm{J(t, z)}_O\cdot \norm{x-y}_{\cX^{2N}} \ .
	\end{align}
	Here, $\norm{\cdot}_O$ is the operator norm \cite{conway07}.
	We then immediately arrive at the desired result:
	$\norm{f\left(t, x\right) - f\left(t, y\right)}_{\cX^{2N}} \leq L \norm{x - y}_{\cX^{2N}}$.
	For the detailed requirements of the Picard-Lindelöf theorem the reader is referred to \cite{arino06}.
	
	Now, all that is left to show is the boundedness of the Jacobian $J(t, (q, p)^\T)$.
	For that, we have to prove the boundedness of four types of partial derivatives for $i, j \in [N]$:
	\begin{align}
		\frac{\partial}{\partial q_{i}} \left(\bGamma(q, q) p\right)_j \ ,\
		&\frac{\partial}{\partial p_{i}} \left(\bGamma(q, q) p\right)_j \\
		\frac{\partial}{\partial q_{i}} \left(-\grad_q \left(\frac{1}{2} p^\mathrm{T} \bGamma(q, q) p\right)\right)_j \ ,\
		&\frac{\partial}{\partial p_{i}} \left(-\grad_q \left(\frac{1}{2} p^\mathrm{T} \bGamma(q, q) p\right)\right)_j\\
	\end{align}
	The index $j$ signifies the entry at position $j$.
	We will demonstrate the boundedness for two cases.
	First, we have
	\begin{equation}
		\frac{\partial}{\partial q_{i}} \left(\bGamma(q, q) p\right)_j
		= \frac{\partial}{\partial q_{i}} \sum_{k=1}^N \Gamma(q_j, q_k) p_j
		=  \sum_{k=1}^N \left(\frac{\partial}{\partial q_{i}} \Gamma(q_j, q_k)\right) p_j \ .
	\end{equation}
	As $q$, $p$ and $\Gamma$ and its first (and second) order partial derivatives are bounded, this expression is also bounded.
	For the second case, we have
	\begin{align}
		\frac{\partial}{\partial q_{i}} \left(-\grad_q \left(\frac{1}{2} p^\mathrm{T} \bGamma(q, q) p\right)\right)_j
		&= - \frac{\partial}{\partial q_i} \frac{\partial}{\partial q_j} \left(\frac{1}{2}
		\sum_{k=1}^N\sum_{l=1}^{N} p_k \Gamma(q_k, q_l) p_l\right)\\
		&= -\frac{1}{2} \left(
		\sum_{k=1}^N\sum_{l=1}^{N} p_k \left(\frac{\partial}{\partial q_i} \frac{\partial}{\partial q_j} \Gamma(q_k, q_l)\right) p_l\right) \ .
	\end{align}
	The boundedness again follows from that of $q, p$ and $\bGamma$.
	That of the other partial derivatives is similarly easy to see.
	This shows that the Jacobian $J$ is indeed bounded and hence concludes the proof.
	
%	That of of $\dot{q}$ is easy to see from \cref{eq:feature-hamiltonian}:
%	\begin{equation}
%	\norm{\dot{q}_i}_\cX \leq \norm{\psi^T(q_i)}_O \norm{\alpha}_\cF 
%	\leq \sup_{x \in \cX}\norm{\psi^T(x)}_O \norm{\alpha}_\cF\ .
%	\end{equation}
%	Here, $\norm{\cdot}_O$ is the operator norm \cite{conway07}.
%	Because of the assumption in \cref{cond:feature-condition}, $\psi$ is bounded.
%	This implies that the adjoint $\psi^\T$ is, too.
%	We already saw that $\alpha$ is constant.
%	It remains to show the boundedness of $\dot{p_i}$.
%	First consider each entry $\dot{p}_{i, j}$ of $\dot{p}_i$.
%	By common differentiation rules and Cauchy-Schwarz we get
%	\begin{align}
%	\abs{p_{i,j}} &= \abs{\frac{\partial}{\partial q_{i, j}}\left<p_i, \psi^\T(q_i) \alpha \right> }\\
%	&= \abs{\left<p_i, \frac{\partial}{\partial q_{i, j}} \psi^\T(q_i) \alpha \right> }\\
%	&\leq \norm{p_i}_\cX \cdot \norm{\frac{\partial}{\partial q_{i, j}} \left(\psi^T(q_i) \alpha\right) }_\cX \ .
%	\end{align}
%	Observe the second factor, $\psi^\T(q_i) \alpha$. $\psi^\T$ is a matrix (actually a bounded linear operator in $L(\cF, \cX)$, but we are finite-dimensional), and $\alpha \in \cF$.
%	The matrix-vector product is a continuous, bilinear operator and thus we can apply the product rule for partial derivatives.
%	$\alpha$ does not depend on $q_{i, j}$, thus the equality $\frac{\partial}{\partial q_{i, j}} \left(\psi^T(q_i) \alpha\right) = \left(\frac{\partial}{\partial q_{i, j}} \psi^T(q_i)\right)  \alpha$ holds true. The partial derivative of $\psi^\T$ is again a bounded linear operator.
%	\begin{equation}
%	\norm{\left(\frac{\partial}{\partial q_{i, j}} \psi^T(q_i)\right)  \alpha}_\cX \leq \norm{\frac{\partial}{\partial q_{i, j}} \psi^T(q_i)}_O \norm{\alpha}_\cF \ .
%	\end{equation}
%	Together with the already established inequality we get
%	\begin{align}
%	\abs{\dot{p}_{i,j}} &\leq \norm{p_i}_\cX \cdot \norm{\frac{\partial}{\partial q_{i, j}} \left(\psi^T(q_i)\right) \alpha }_\cX\\
%	&\leq  \norm{p_i}_\cX \cdot \norm{\frac{\partial}{\partial q_{i, j}} \psi^T(q_i)}_O \cdot \norm{\alpha }_\cF \ .
%	\end{align}
%	In total, this gives
%	\begin{align}
%	\norm{\dot{p}_i}^2 &= \sum_{i=1}^N \dot{p}_{i,j}^2\\
%	&\leq \norm{p_i}_\cX^2 \cdot \norm{\alpha }_\cF^2 \cdot \sum_{i=1}^N  \norm{\frac{\partial}{\partial q_{i, j}} \psi^T(q_i)}_O^2\\
%	& = \norm{p_i}_\cX^2 \cdot \norm{\alpha }_\cF^2 \cdot \norm{\grad \psi^T(q_i)}^2 \\
%	& \leq \norm{p_i}_\cX^2 \cdot \norm{\alpha }_\cF^2 \cdot \sup_{x \in \cX}\norm{\grad \psi^T(x)}^2 \ ,
%	\end{align}
%	which is, of course, equivalent to 
%	\begin{equation}
%	\label{eq:norm-p-inequality}
%	\norm{\dot{p}_i}\leq \norm{p_i}_\cX \cdot \norm{\alpha }_\cF \cdot \sup_{x \in \cX}\norm{\grad \psi^T(x)} \ .
%	\end{equation}
%	All factors on the right side are bounded: $p_i$ is a continuous function defined on a compact interval, $\alpha$ is constant, as we have seen, and the last term by \cref{cond:feature-condition}.
%	This means $\dot{p_i}$ and therefore $\dot{p}$ are bounded, too.
%	This concludes the proof.
\end{proof}

%\subsubsection{Outlook: Hamiltonian System in Feature Space}
%Although this is not necessary for the results in other sections, we will still take a look at the Hamiltonian system with its representation in feature space, as it of interest in its own.
%
%Recall \cref{eq:kernel-feature-map}, which reads $\Gamma(x_1, x_2) = \psi^\mathrm{T}(x_1)\psi(x_2)$.
%Using this equality, we can rewrite \cref{eq:hamiltonian-system} as
%\begin{equation}
%\label{eq:feature-hamiltonian}
%	\begin{split}
%		\dot{q}_i &= \psi^\mathrm{T}(q_i) \alpha\\
%		\dot{p}_i &= -\frac{\partial}{\partial q_i} \left(p_i^\mathrm{T} \psi^\mathrm{T}(q_i) \alpha \right)\ ,
%	\end{split}
%\end{equation}
%with $\alpha \coloneqq \sum_{k=1}^{N} \psi(q_k) p_k$.
%
%A very interesting consequence of \cref{cor:feature-space-norm} is that $\alpha$ is constant:
%\begin{align}
%	\norm{\alpha}_\cF^2 &= \norm{\psi^\T(x) \alpha}_\cV^2\\
%	&= \norm{\sum_{i=1}^N \psi^\T(x) \psi(q_i)p_i}_\cV^2 \\
%	&= \norm{\sum_{i=1}^N \Gamma(x, q_i) p_i}_\cV^2\\
%	&= \left<\sum_{i=1}^N\Gamma(x, q_i)p_i, \sum_{j=1}^N \Gamma(x, q_j) p_j \right>_\cV\\
%	&= \sum_{i=1}^N \sum_{j=1}^N \left<\Gamma(x, q_i)p_i, \Gamma(x, q_j) p_j \right>_\cV\\
%	&= \sum_{i=1}^N \sum_{j=1}^N \left<p_i, \Gamma(q_i, q_j) p_j\right>_\cX\\
%	&= p^\T \bGamma(q, q) p \ .
%\end{align}
%In the second to last step we again used the reproducing property of the kernel $\Gamma$.
%This gives us $\norm{\alpha}_\cF^2 = 2 \fH(q, p)$.
%\cref{cor:energy-preservation} states that $\fH$ is constant across time, which implies that $\norm{\alpha}_\cF^2$ is, too (and thus  $\norm{\alpha}_\cF$).


\subsection{Continuous Limit and Adherence Values}

The only thing left to prove now is that as $L \rightarrow \infty$, \cref{prob:min-q} does indeed converge towards \cref{prob:cont-least-action}.

\begin{figure}
	\makebox[\textwidth][c]{
		\centering
		\begin{tikzpicture}[font=\scriptsize, align=center]
				\node[draw, above left, align=center] (a) {
					ResNet \ref{prob:min-v-f}\\
					$\begin{cases}
						\text{Min~} & \frac{\nu \cdot L}{2} \sum_{k=1}^{L} \norm{v_k}_\cV^2
						+ l_R(\Phi_L(X), Y) \\
						\text{s.t.~} & v_1, \ldots, v_L \in \cV,\ \Phi_L = (I + v_L) \circ \ldots \circ (I + v_1)
					\end{cases}$};
				\node[right=of a, align=center,outer sep=-2em] (z1) {\large$\stackrel{L \rightarrow \infty}{\longrightarrow}$\\
					\scriptsize\cref{theo:problem-convergence}};
				\node[draw, right= of z1, align=center] (b) {
					Limit of ResNet \ref{prob:resnet-limit}\\
					$\begin{cases}
						\text{Min~}& \frac{\nu}{2} \int_{0}^{1} \norm{v}_\mathcal{V}^2 \mathrm{d}t
						+ l(\phi_v(X, 1), Y)\\
						\text{s.t.~}& v \in C([0, 1], \mathcal{V}),\ \Phi(x, 0) = x,\\
						&\dot{\Phi}^v(x, t) = \mathbf{\Gamma}(\Phi^v(x, t), q(t)) \bGamma(q(t), q(t))^{-1} \dot{q}(t)
					\end{cases}$};
				\node[below= of a,outer sep=-2em] (z2) {\large{$\big\Updownarrow$} \scriptsize\cref{theo:v-q-problem-equivalence}};
				\node[draw, below= of z2, align=center] (c) {
					Discrete Least Action \ref{prob:min-q}\\
					$\begin{cases}
						\text{Min~} & \frac{\nu}{2} \sum_{k=1}^{L} \left(\frac{q_{k+1} - q_k}{\Delta t}\right)^\mathrm{T} \bGamma(q_k, q_k)^{-1} \left(\frac{q_{k+1} - q_k}{\Delta t}\right) \Delta t\\
						&+ l(q_{L+1}, Y) \\
						\text{s.t.~} & q_1 = X,\ q_2, \ldots, q_{L+1} \in \cX^N,\ t = \frac{1}{L}\\
						& v_k(x) = \bGamma(x, q_k)^\mathrm{T}\bGamma(q_k, q_k)^{-1} (q_{k+1} - q_k)
					\end{cases}$};
				\node[below= of b,outer sep=-2em] (z3) {\large{$\big\Updownarrow$} \scriptsize\cref{theo:v-q-continuous-problem-equivalence}};
				\node[right=of c, align=center,outer sep=-2em] (z4) {\large$\stackrel{L \rightarrow \infty}{\longrightarrow}$\\
					\scriptsize\cref{theo:problem-convergence}};
				\node[draw, below= of z3, right=of z4, align=center] (d) {
					Least Action \ref{prob:cont-least-action}\\
					$\begin{cases}
					\text{Min~} & \frac{\nu}{2} \int_{0}^{1} \dot{q}(t)^\mathrm{T} \mathbf{\Gamma}(q(t), q(t))^{-1}  \dot{q}(t) \mathrm{d}t + l(q(1), Y)\\
					\text{s.t.~} & q \in C^1([0,1], \cX^N),\ q(0) = X
					\end{cases}$};
				\node[below= of d,outer sep=-2em] (z5) {\large{$\big\Updownarrow$} \scriptsize\cref{theo:geodesic-shooting}};
				\node[draw, below= of z5, align=center] (e) {
					Geodesic Shooting \ref{prob:geodesic-shooting}\\
					$\begin{cases}
						\text{Min~} & \frac{\nu}{2} p(0)^\T \bGamma(X, X)p(0) + l(q(1), Y)\\
						\text{s.t.~} & 	p = \bGamma(q, q)^{-1}\dot{q},\ q(0) = X,\ \dot{q} = \bGamma(q, q) p,\\
						&\dot{p} = -\grad_q \left(\frac{1}{2} p^\mathrm{T} \bGamma(q, q) p\right)
					\end{cases}$};
				\node[below= of c,outer sep=-2em] (z6) {\large{$\big\Updownarrow$} \scriptsize\cref{theo:discrete-shooting-min-q-equivalence}};
				\node[draw, below= of z6, align=center] (f) {
					Discrete Geodesic Shooting \ref{prob:discrete-geodesic-shooting}\\
					$\begin{cases}
					\text{Min~} & \frac{\nu}{2} \sum_{k=1}^L p_k^\T \bGamma(q_k, q_k) p_k \Delta t + l(q_{L+1}, Y)\\
					\text{s.t.~} & p_k = \bGamma(q_k, q_k)^{-1} \frac{q_{k+1} - q_k}{\Delta t},\ q_1 = X \\
					& q_{k+1} = q_k + \Delta t \bGamma(q_k, q_k) p_k,\\
					& p_{k+1} = p_k + \frac{\Delta t}{2} \grad_{q_k} \left(p_k^\T \bGamma(q_k, q_k) p_k\right)
					\end{cases}$};
				\node[right=of f, align=center,outer sep=-2em] (z7) {\large$\stackrel{L \rightarrow \infty}{\longrightarrow}$\\
					\scriptsize\cref{theo:problem-convergence}};
		\end{tikzpicture}
	}
	\caption{Summary of problems, their correspondence, their continuous limits and the respective theorems. Discrete problems are on the left, continuous on the right.}
	\label{fig:convergence}
\end{figure}