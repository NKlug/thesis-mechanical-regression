\section{Mechanical Regression}

In \cref{sec:neural-networks} we have seen that Neural Networks are a prominent way of approximating a solution to the Supervised Learning problem.
ResNets in particular enabled the use of very deep architectures.
In this section, we take a closer look at ResNets and derive their continuous limit as the number of residual blocks tends towards infinity.
On the way of doing so, we show that the ResNet regression approach can be regarded as a discrete mechanical system following Hamilton's stationary action principle -- thus the term \emph{Mechanical Regression}.
In the limit, this system converges to a continuous version of the same principle, which proves to be equivalent to what in image registration is known as geodesic shooting \cite{allassonniere05}.

\citet{owhadi20} compares these results to the problem formulations used in image shape analysis, computational anatomy and image registration \cite{bibid}.
He views the continuous solution as a generalization of image registration and calls it \emph{idea registration}, arguing that just as in the former, data points are aligned through various transformations in their respective feature spaces.
The difference lies in the fact that in idea registration the data points can be arbitrary rather than mere landmarks and the spaces can be high dimensional (compared to the two or three dimensional spaces images lie in).

This section essentially follows Chapter 3 of \cite{owhadi20} and aims to provide a profound and comprehensive explanation of Mechanical Regression and how it can be interpreted as the continuous limit of ResNets.

\subsection{Modeling Residual Neural Networks}

Recall the Supervised Learning problem: Given training data $X$ and $Y$ consisting of $X_i \in \cX$, $Y_i \in \cY$ and $f^\dagger(X) = Y$, approximate $f^\dagger$.
On possibility to solve this problem are Residual Neural Networks, which approximate the target function $f^\dagger$ through a series of residual blocks.
Mathematically, we can write
\begin{equation}
	f^\ast \coloneqq f \circ \Phi_L 
\end{equation}
for the approximate solution, where
\begin{equation}
\Phi_L \coloneqq \phi_L \circ \phi_{L-1} \circ \ldots \circ \phi_1.
\end{equation} 
$\Phi_L$ is the composition of $L$ Residual Blocks, that is, functions $\phi_k = I + v_k$.
Here, the $v_k$ are functions mapping the input space $\cX$ onto itself and $I$ is the identity operator.
Thus, we can regard $\Phi_L$ as a large deformation of $\cX$.
$f: \cX \to \cY$ is a function mapping the deformed space to the target space $\cY$.

By default, the $v_k$ and $f$ can be arbitrary functions.
However, it poses a challenge to not only approximate the target with any series of functions, but with such that generalize well, meaning that they also perform well on data other than the training data $(X, Y)$.
Thus, it is common in Machine Learning \cite{goodfellow16} to apply some form of regularization.
One popular approach, which we will also use here, is to apply penalties to the parameter's norms -- which are, in this case, the $v_k$ and $f$.

We still need to define appropriate spaces for the functions $v_k$ and $f$.
As we need norms for both, we introduce two RKHSs:
$\cV \subseteq \{\cX \rightarrow \cX\}$ and $\cH \subseteq \{\cX \rightarrow \cY\}$.
We want $v_k \in \cV$ for all $k$ and $f \in \cH$.
By \cref{theo:kernel-for-rkhs} there is a Kernel associated with each RKHS.
Let $\Gamma$ be the Kernel associated with $\cV$ and $K$ that of $\cH$.
Then we can identify $f$ and the $v_k$ as solutions to the following problem:
\begin{problem}
	\label{prob:min-v-f}
	\begin{cases}
		\text{Minimize~} & \nu \cdot \frac{L}{2} \sum_{k=1}^{L} \norm{v_k}_\cV^2
		+ \lambda \norm{f}_\cH^2 
		+ l((f \circ \Phi_L)(X), Y) \\
		\text{such that~} & v_1, \ldots, v_L \in \cV, f \in \cH \ .
	\end{cases}
\end{problem}
Here, $\nu$ and $\lambda$ are strictly positive balancing parameters.
$l$ is a (positive) loss measuring the similarity of the predicted outputs, that is, the image of $X$ under $f \circ \Phi_L$.
Note that, just as intended, $v_k$ and $f$ with large norms are penalized.

Utilizing the Ridge Regression Loss $l_R$ (\cref{eq:ridge-regression-loss}), we can rewrite the above minimization problem as
\begin{problem}
	\begin{cases}
		\text{Minimize~} & \nu \cdot \frac{L}{2} \sum_{k=1}^{L} \norm{v_k}_\cV^2
		+ l_R(\Phi_L(X), Y) \\
		\text{such that~} & v_1, \ldots, v_L \in \cV\ .
	\end{cases}
\end{problem}
By hiding the regularity of $f$ in the loss we can, for now, focus exclusively on the functions $v_k$.
For our calculations we will only assume that $l$ is a positive, continuous loss function.
If desired we can later balance the $v_k$ with the regularity of $f$ by choosing the loss appropriately.

In the calculations, we will work under the following conditions.
\begin{condition}
	\label{cond:feature-condition}\mbox{}
	\vspace*{-\parsep}
	\vspace*{-\baselineskip}
	\begin{enumerate}
		\item There exist a finite dimensional feature space and map $\cF$ and $\psi$ such that $\psi$ and its first and second order partial derivatives are continuous and uniformly bounded.
		\item There exists an $r > 0$ such that $\forall z \in \cX^N:~z^\T \bGamma(X, X) z \geq r z^\T z$.
		\item $\cX$ and $\cY$ are finite-dimensional.
		\item $l: \cX^N \times \cY^N \rightarrow$ is a positive and continuous loss.
	\end{enumerate}
\end{condition}
Note that the first condition implies that the function $(x_1, x_2) \rightarrow \Gamma(x_1, x_2)$ and its first and second order partial derivatives are continuous and uniformly bounded.
This follows immediately from the fact that we can write $\Gamma(x_1, x_2) = \psi^\T(x_1)\psi(x_2)$.
Furthermore, we have the following equivalence.
\begin{lemma}
	\cref{cond:feature-condition} (2) is equivalent to the non-singularity of $\bGamma(X, X)$.
\end{lemma}
\begin{proof}
	If $\bGamma(X, X)$ was singular, there would exist a vector $z \in \cX^N$ with $\norm{z}_{\cX^N} > 0$ and $\bGamma(X, X) z = 0$, implying $z^\T \bGamma(X, X)z = 0$. 
	A contradiction.
	Conversely, if there exists a $0 \neq z \in \cX^N$ such that $z^\T \bGamma(X, X) z < \epsilon z^\T z$ for all $\epsilon > 0$, then $\bGamma(X, X) z = 0$.
\end{proof}
Ideally, we would not need additional conditions like those introduced above.
They are, however, automatically satisfied in most real world examples and ease calculations and proofs considerably.

\subsection{Discrete Stationary Action Principle}

Our main goal will now be to show that the minimization problem \ref{prob:min-v-f} can in fact be considered as a discrete solver of a mechanical system.
In order to do that, we first reformulate the problem by introducing additional variables $q_{i,j}$ for $2 \leq i \leq L+1$, $j \in [N]$:
\begin{align}
	q_{1, j} &\coloneqq X_j \, \\
	q_{i, j} &\coloneqq (\phi_{i-1} \circ \ldots \circ \phi_1) (X_j) \ .
\end{align}
Writing $q_i$ for the vector with entries $q_{i,j}$, we get $q_1 = X$ and $q_i = \phi_i(q_{i-1})$.
Consider the following minimization problem:
\begin{problem}
	\label{prob:min-q}
	\begin{cases}
		\text{Minimize~} & \nu \cdot \frac{1}{2} \sum_{k=1}^{L} \left(\frac{q_{k+1} - q_k}{\Delta t}\right)^\mathrm{T} \bGamma(q_k, q_k)^{-1} \left(\frac{q_{k+1} - q_k}{\Delta t}\right) \Delta t+ l(q_{L+1}, Y) \\
		\text{such that~} & q_1 = X,\ q_2, \ldots, q_{L+1} \in \cX^N  \text{~and~} \Delta t = \frac{1}{L} \ .
	\end{cases}
\end{problem}
We will now prove that this problem is in fact equivalent to \cref{prob:min-v-f} and also that we can obtain a closed-form expression for the $v_k$ as a function of the $q_k$.
Remember from \cref{sec:preliminaries} that $\bGamma(x, q_k)$ denotes the vector with entries $\Gamma(x, q_{k,i})$.

\begin{theorem}
	\label{theo:v-q-problem-equivalence}
	$v_1, \ldots, v_L \in \cV$ minimize \cref{prob:min-v-f} if and only if $q_1, \ldots, q_{L+1} \in \cX^N$ minimize \cref{prob:min-q} and $v_k(x) = \bGamma(x, q_k)^\mathrm{T}\bGamma(q_k, q_k)^{-1} (q_{k+1} - q_k)$ for all $k \in [L]$.
\end{theorem}
\begin{proof}
	Let $v_1, \ldots, v_L$ minimize \cref{prob:min-v-f}.
	Using $q_k$ as defined above we have $v_k(q_k) = q_{k+1} - q_k$ for all $k \in [L]$.
	We can rewrite \cref{prob:min-v-f} as
	\begin{problem}
		\label{prob:min-q-v}
		\begin{cases}
			\text{Minimize~} & \nu \cdot \frac{L}{2} \sum_{k=1}^{L} \norm{v_k}_\cV^2
			+ l(q_{L+1}, Y) \\
			\text{such that~} & v_1, \ldots, v_L \in \cV,\ \forall k \in [L]: v_k(q_k) = q_{k+1} - q_k, \\
			& q_1 = X \text{~and~} q_2, \ldots, q_{L+1} \in \cX^N \ .
		\end{cases}
	\end{problem}
	Here we just added the $q_k$ as additional variables but then constrained them to $q_{k+1} = q_k + v_k(q_k)$, with $q_1 = X$.
	Recursively we get $q_{L+1} = \Phi_L(X)$ which means the target function remains the same as in \cref{prob:min-v-f}.
	Also, the domains of the $v_k$ do not change.
	This means that the minimal points are the same as in \cref{prob:min-v-f}.
	
	We will now derive closed form expressions for the $v_k$ as functions of the $q_k$.
	For that, let $q_k \in \cX^N$, $2 \leq k \leq N$ be arbitrary but fixed .
	This means $l(q_{L+1}, Y)$ is constant and $v_k \in V_q(k) \coloneqq \{v \in \cV~|~ v_k(q_k) = q_{k+1} - q_k\}$.
%	These sets remain convex: Let $\lambda \in [0, 1]$ and $v_{k, 1}, v_{k, 2} \in V_{k, q}$, then we get
%	\begin{align}
%		(\lambda v_{k, 1} + (1 - \lambda) v_{k, 2})(q_k) &= (\lambda v_{k, 1})(q_k) + ((1 - \lambda) v_{k, 2})(q_k)\\
%		&=\lambda(q_{k+1} - q_k) + (1 - \lambda)(q_{k+1} - q_k)\\
%		&= q_{k+1} - q_k \ .
%	\end{align}
	We are left with the following problem:
	\begin{problem}
		\begin{cases}
			\text{Minimize} &\sum_{k=1}^L \norm{v_k}_\cV^2\\
			\text{such that} & \forall k \in [L]: v_k \in V_q(k) \ ,
		\end{cases}
	\end{problem}
	where each summand is non-negative.
	One can easily verify that minimization problems of this type have a global minimum at any point where each of the summands is minimal.
	The boundedness of the summands ensures that at least one such minimum exists.

	Thus, we have to find the minima of $\min\{\norm{v_k}_\cV^2\ |\ v_k \in \cV,\ v_k(q_k) = q_{k+1} - q_k\} = l_{\text{OR}}(q_k, q_{k+1} - q_k)$.
	Here, $l_{\text{OR}}$ is the optimal recovery loss defined in \cref{eq:optimal-recovery-loss}.
	From that section we get the following representations for the minimal $v_k$ and the target value:
	\begin{align}
		v_k & = \bGamma(\cdot, q_k)^\mathrm{T}\bGamma(q_k, q_k)^{-1} (q_{k+1} - q_k) \ ,\\
		\norm{v_k}_\cH^2 &= (q_{k+1} - q_k)^\mathrm{T} \bGamma(q_k, q_k)^{-1} (q_{k+1} - q_k) \ .
	\end{align}
	In these expressions $\bGamma(q_k, q_k)$ is the block operator matrix with entries $\Gamma(q_{k,i}, q_{k, j})$ and $\bGamma(x, q_k)$ the vector $(\Gamma(x, q_{k, i}))_{i \in [N]}$.
	Having now computed optimal $v_k$ -- or rather their squared $\cH$-norms -- as a function of $q_2, \ldots, q_{L+1}$, we can reformulate \cref{prob:min-q-v} without the $v_k$ as variables.
	Define $\Delta t \coloneqq \frac{1}{L}$.
	We get
	\begin{problem}
		\begin{cases}
			\text{Minimize~} & \nu \cdot \frac{1}{2} \sum_{k=1}^{L}  
			\left(\frac{q_{k+1} - q_k}{\Delta t}\right)^\mathrm{T} \bGamma(q_k, q_k)^{-1}
			\left(\frac{q_{k+1} - q_k}{\Delta t}\right) \cdot \Delta t
			+ l(q_{L+1}, Y) \\
			\text{such that~} & q_1 = X \text{~and~} q_2, \ldots, q_{L+1} \in \cX^N \ .
		\end{cases}
	\end{problem}
	As all steps and transformations hold true in both ways, this concludes the proof.
\end{proof}

When comparing the target function of \cref{prob:min-q} with the results found in discrete mechanics literature \cite[~Chapter VI.6.2]{hairer06}, in particular with those for the discretized stationary action principle, it turns out that there are strong similarities.
We will not explore these similarities in depth for the discrete case but rather take a closer look at the continuous counterpart in the next section.

\subsection{Stationary Action Principle}

In order to show the similarity of \cref{prob:min-q} to a mechanical system, consider the first term of its objective function (the balancing parameter $\frac{\nu}{2}$ has been deliberately omitted):
\begin{equation}
	\label{eq:discrete-lagrangian}
	\sum_{k=1}^{L} \left(\frac{q_{k+1} - q_k}{\Delta t}\right)^\mathrm{T} \mathbf{\Gamma}(q_k, q_k)^{-1} \left(\frac{q_{k+1} - q_k}{\Delta t}\right) \Delta t \ .
\end{equation}
This very much looks like an approximation to the integral of some continuous function with a step function using $L$ intervals of width $\Delta t$.
This would require the sequence $q_k$ to be a discrete equidistant approximation of a function $q: A \rightarrow \cX^N$, where $A$ is an interval in $\R$.
Without loss of generality we choose $A = [0, 1]$ and let $q$ be such that $q_k \approx q(k \Delta t)$.
This way, $\frac{q_{k+1} - q_k}{\Delta t}$ can be interpreted as a forward difference quotient approximating the derivative of $q$ at the point $k \Delta t = \frac{k}{L}$.
Using physical nomenclature we call the domain of $q$ \emph{time} and write $\dot{q}$ for the (time-)derivative $\frac{\mathrm{d}}{\mathrm{d}t}q(t)$.
In mechanical applications, $q$ is the trajectory of particles and $\dot{q}$ their velocities.
This motivates the following approximation:
\begin{equation}
	\left(\frac{q_{k+1} - q_k}{\Delta t}\right)^\mathrm{T} \mathbf{\Gamma}(q_k, q_k)^{-1} \left(\frac{q_{k+1} - q_k}{\Delta t}\right)
	\approx \dot{q}\left(\frac{k}{L}\right)^\mathrm{T} \mathbf{\Gamma}\left(q\left(\frac{k}{L}\right), q\left(\frac{k}{L}\right)\right)^{-1}\dot{q}\left(\frac{k}{L}\right) \ .
\end{equation}
This leads us to the obvious question if the solutions of \cref{prob:min-q} are also approximating the solutions of a similar continuous problem.
As it turns out, they do.
This "convergence" will be the main result of this section and is presented at the end in \cref{theo:problem-convergence}.
For now, we will work under the assumption that it holds true and continue evolving the theory.

Consider the \emph{Lagrangian}
\begin{equation}
\label{eq:lagrangian}
\begin{split}
\fL: [0, 1] \times U \times \cX^N \rightarrow \R\\ 
\fL(t, x_1, x_2) \coloneqq \frac{1}{2}  x_2^\mathrm{T} \mathbf{\Gamma}(x_1, x_1)^{-1} x_2 \ .
\end{split}
\end{equation}
$U \subset \cX^N$ is the open subset on which $\bGamma(x_1, x_1)$ is invertible.
The Lagrangian has its origins in Lagrangian mechanics (from a physical point of view) and calculus of variations (the mathematical approach) and has a strong physical interpretation.
But first, a caveat concerning the Lagrangian:
It is not exclusively defined when the third argument is the time derivative of the second, but more generally.
Using this general definition, we can then construct a function $(t, q) \rightarrow \fL(t, q(t), \dot{q}(t))$, where $q: [0, 1] \rightarrow U$ and $\dot{q} = \frac{\mathrm{d}}{\mathrm{d}t}q(t)$.

Commonly, in the Lagrangian theory $q$ describes the trajectory of particles.
Within our application, each training sample constitutes a particle, therefore $q$ -- a vector -- describes the trajectories of not one, but $N$ particles.
The Lagrangian defined in \cref{eq:lagrangian} is a quadratic form.
In a physical interpretation it most closely resembles that of a system of free moving particles that don't interact with each other (which also is a quadratic form, though not dependent on the location).
Here, the particles are coupled by a location dependent matrix, which can be regarded as a mass matrix \cite{marsden10}.
However, in our case these "masses" are be location dependent.

%In a physical scenario, the Lagrangian is often the difference between the kinetic and the potential energy of (the particles of) the system, that is $L = T - V$.
%Usually $T$ does not depend on the second and $V$ not on the third argument of the function, which correspond to the locations and velocities of the particles.
%In physics, the Lagrangian theory is especially helpful in problems where constraining forces act.

Next, we can define the \emph{action}:
\begin{equation}
\label{eq:action}
	\cA(q) \coloneqq \int_{0}^{1} \fL(t, q(t), \dot{q}(t)) \mathrm{d}t \ .
\end{equation}
It also originates from theoretical physics, as we well see below.
But first, using this definition we can formulate a continuous version of \cref{prob:min-q}:
\begin{problem}
\label{prob:cont-least-action}
	\begin{cases}
		\text{Minimize~} & \nu \cA(q) + l(q(1), Y)\\
		\text{such that~} & q \in C^1([0,1], \cX^N),\ q(0) = X \ .
	\end{cases}
\end{problem}
As usual, $C^1([0,1], \cX^N)$ is the set of continuously differentiable functions $q: [0, 1] \rightarrow \cX^N$.
Now that we have established a problem formulation within the framework of theoretical physics, we can explore if the theory of Lagrangian mechanics can help us in the analysis of the problem.

One core principle in Lagrangian Mechanics is \emph{Hamilton's Stationary Action Principle}, also referred to as \emph{The Principle of Least Action} or just \emph{Hamilton's Principle}.
It postulates that\vspace{.5em}
\newline
\noindent{\emph{"The motion of the system from time $t_1$ to time $t_2$ is such that the [action] has a stationary value for the actual path of the motion." \cite{goldstein01}}}

This means that in the real world, the trajectories particles follow are a stationary point of the action.
A stationary point is one for which the \emph{variation} of the action is $0$.
Note that "Principle of Least Action" is hence a misnomer in general.
Instead, "Principle of Stationary Action" is more appropriate, as the action does not necessarily have to be minimal.
Regardless of that, within the scope of \cref{prob:cont-least-action} it is adequate to use the term "Least Action" because we are in fact seeking a minimum.
An introduction to the theory derived from the postulate can be found in various books on theoretical mechanics \cite{goldstein01, marsden10, feynman63} and on calculus of variations \cite{kielhofer18}.

Despite the fact that Hamilton's principle is postulated in physics, within the scope of \cref{prob:cont-least-action} we do not rely on it being a postulate.
By the problem's definition it is \emph{required} that the actual trajectory $q$ minimizes the action.

A core result in calculus of variations are the Euler-Lagrange equations.
It can be shown that every extremal point $q$ of the actions satisfies them.
The Euler-Lagrange (EL) equations  are given by
\begin{equation}
\frac{\mathrm{d}}{\mathrm{d}t} \frac{\partial \fL}{\partial \dot{q}_{i, j}} - \frac{\partial \fL}{\partial q_{i, j}} = 0
\end{equation}
and hold true for each component $q_{i, j}$ of the vector $q$ and its time-derivative $\dot{q}$.
The partial derivatives with respect to $q$ and $\dot{q}$ are to be read as the partial derivatives of the Lagrangian $\fL$ with respect to the second and third argument at the point $x_1 = q(t)$ and $x_2 = \dot{q}(t)$.

Now we will apply this theory to \cref{prob:cont-least-action}.
Here, we also seek to minimize the action, but in combination with the loss term which depends on the endpoint of the trajectory, $q(1)$.
However, this is not an obstacle, as the following lemma shows.
\begin{lemma}
	A minimizer $q$ of \cref{prob:cont-least-action} follows the Euler-Lagrange equations.
\end{lemma}
\begin{proof}
	Let $q(1) \coloneqq a \in \cX^N$ be arbitrary but fixed.
	Then $l(q(1), Y)$ is constant and the problem is reduced to the minimization of the first term -- the action.
	By \cite[~Proposition 1.4.1]{kielhofer18} it follows that $q$ satisfies the EL equations.
	As this holds true for all endpoints $a$, it especially hold true for the endpoint of the optimal $q$.
\end{proof}

As we have seen, the vector $q$ describes the trajectories of $N$ particles.
Setting $m \coloneqq \mathrm{dim}(\cX)$, this would result in $mN$ EL equations, one for each $q_{i, j}$, $i \in [N],\ j \in [m]$.
For simplicity we will combine them into one and write $\grad_{\dot{q}} \fL$ for the vector of partial derivatives with respect to $\dot{q}_{i, j}$ (the gradient).
$q(t)$ is an $(N\times1)$ vector with elements in $\cX$ and in turn each of its elements can be regarded as a $(m\times1)$ vector.
So technically speaking $q$ is $(N \times m)$, or equivalently (\cref{cor:matrix-ring-equivalence}) a $(mN \times 1)$ vector.
For the notation $\grad_q \fL$, the interpretation as $(mN \times 1)$ would be more appropriate.
However, for more flexibility we will not strictly define it like that and interpret such expressions with generosity.

$\grad_{\dot{q}} \fL$ is to be interpreted analogously.
Together with the block operator notation we can write the EL equations in a concise formulation.
Because of $\grad_{\dot{q}} \fL = \bGamma(q(t), q(t))^{-1} \dot{q}(t)$, we get
\begin{equation}
\label{eq:concrete-lagrangian}
	\frac{\mathrm{d}}{\mathrm{d}t} \left(\mathbf{\Gamma}(q(t), q(t))^{-1} \dot{q}(t) \right)
	= \grad_q \left(\frac{1}{2} \dot{q}(t)^\mathrm{T} \mathbf{\Gamma}(q(t), q(t))^{-1} \dot{q}(t)\right) \ .
\end{equation}


\subsection{Hamiltonian Representation}

An alternative formulation of Lagrangian mechanics is the Hamiltonian formalism.
In its own, it does not add anything particularly new but rather gives us a more powerful framework to work with the already established theory.
In essence, a change of variables from $(q, \dot{q}, t)$ to $(q, p, t)$ is applied through a certain kind of transformation called \emph{Legendre transformation}.
The obtained $(q, p)$ are known as the \emph{canonical variables}, concretely canonical coordinate and canonical momentum.
We will briefly derive the Hamiltonian formulation for our application, but not cover all the theoretical details.
For a complete formal derivation from a physical point of view, see \cite[Chapter~8]{goldstein01}.
For a more mathematical approach, see \cite[Chapter~2]{marsden10}.

First, define the canonical momentum
\begin{equation}
\label{eq:canonical-momentum}
	p(t) \coloneqq \frac{\partial \fL(q, \dot{q}, t)}{\partial \dot{q}} = \bGamma(q(t), q(t))^{-1} \dot{q}(t) \ .
\end{equation}
Note that just as we actually had $N$ Euler-Lagrange Equations above, we now also have $N$ canonical momenta, each defined as $p_i = \frac{\partial \fL}{\partial \dot{q}_i}$.
Again, we use vector notation, and call $p$ \emph{the} canonical momentum.
Next, we define the \emph{Hamiltonian function}, often also called \emph{energy function} \cite{marsden10}:
\begin{equation}
	\fH(q, p, t) \coloneqq p(t)^\mathrm{T}\dot{q}(t) - \fL(q, \dot{q}, t) \ .
\end{equation}
In physical context, $\fH$ can often be expressed as the sum of the system's kinetic and potential energies.
Hence $\fH$ is a measure of the total energy of the system.
Here, we get
\Todo{Add one more step at the beginning, starting from the above eq.}
\begin{align}
	\fH(q, p, t) &= \frac{1}{2} \dot{q}(t)^\T \bGamma(q(t), q(t))^{-1} \dot{q(t)} \\
	&= \frac{1}{2} \dot{q}(t)^\T \bGamma(q(t), q(t))^{-1} \bGamma(q(t), q(t)) \bGamma(q(t), q(t))^{-1} \dot{q}(t)\\
	&= \frac{1}{2}  \left(\bGamma(q(t), q(t))^{-1} \dot{q}(t)\right)^\T \bGamma(q(t), q(t)) \bGamma(q(t), q(t))^{-1} \dot{q}(t)\\
	&= \frac{1}{2} p(t)^\T \bGamma(q(t), q(t)) p(t) \ .
\end{align}

As you can see, the equations already become quite cluttered.
\Todo{Even if some confusion might arise...}
Thus, in the following we will, for simplicity, often omit the time dependence of $p$ and $q$.
For example, we write $\Gamma(q, q)$ instead of $\Gamma(q(t), q(t))$.

The next theorem describes the correspondence between Lagrangian and Hamiltonian mechanics:
\Todo{Important: Explain meaning of partial symbol (is to be read as a gradient) or write differently}
\begin{theorem}
	\label{theo:hamiltonian-dynamic}
	If $q$ minimizes \cref{prob:cont-least-action}, $(q, p)$ follow Hamilton's equations
	\begin{equation}
	\label{eq:hamiltonian-system}
		\begin{split}
			\dot{q} &= \grad_p \fH(q, p) = \bGamma(q, q) p\\
			\dot{p} &= -\grad_q \fH(q, p)
			= -\grad_q \left(\frac{1}{2} p^\mathrm{T} \bGamma(q, q) p\right)
		\end{split}
	\end{equation}
	with $q(0) = X$.
\end{theorem}
\begin{proof}
	Let $q$ be a minimizer of \cref{prob:cont-least-action}.
	Then $q(0) = X$ and, as we have seen, $q$ fulfills the Euler-Lagrange equations.
	The claim immediately follows from the equivalence of the Euler-Lagrange equations and Hamilton's equations under the diffeomorphism defined in \cref{eq:canonical-momentum}, as can be seen in e.g. \cite{marsden10, goldstein01}.
\end{proof}

\Todo{Elaborate!}
Just as before, $q(1)$ was unknown, now, $p(0)$ is to be determined.
This means we our problem now reduces to the search for the initial momentum $p(0)$.


From $\frac{\partial \fL}{\partial t} = 0$ it follows that:
\Todo{Think about whether to keep this. Seems kind of irrelevant}
\begin{corollary}
	\label{cor:energy-preservation}
	Along $q$ the energy is preserved.
\end{corollary}
\begin{proof}
	The Lagrangian is not explicitly time dependent, that is $\frac{\partial \fL}{\partial t} = 0$.
	Therefore the Euler-Lagrange equations imply
	\begin{equation}
		\frac{\mathrm{d}}{\mathrm{d} t} \fL = \sum_{i=1}^{N}\left(\frac{\partial \fL}{\partial q_i} \dot{q_i} + \frac{\partial \fL}{\partial \dot{q_i}} \ddot{q_i} \right )
		=\sum_{i=1}^{N}\left( \left( \frac{\mathrm{d}}{\mathrm{d} t} \frac{\partial \fL}{\partial \dot{q_i}} 
		\right)\dot{q_i}+ \frac{\partial \fL}{\partial \dot{q_i}} \ddot{q_i}\right )
		= \frac{\mathrm{d}}{\mathrm{d} t} \left(\sum_{i=1}^{N} \frac{\partial \fL}{\partial \dot{q_i}} \dot{q_i}\right) \ .
	\end{equation}
	It follows that
	\begin{equation}
		0 = \frac{\mathrm{d}}{\mathrm{d} t} \left( \left(\sum_{i=1}^{N} \frac{\partial \fL}{\partial \dot{q_i}} \dot{q_i} \right) -  \fL \right) 
		= \frac{\mathrm{d}}{\mathrm{d} t} \left( \left(\sum_{i=1}^{N} p_i \dot{q_i} \right ) - \fL \right)
		= \frac{\mathrm{d}}{\mathrm{d} t} \fH \ ,
	\end{equation}
	which means that the energy function does not change over time and hence is constant.
\end{proof}

Our next task will be to show that there exists a unique solution to the system in \cref{eq:hamiltonian-system}.
If this is the case, we could solve -- or at least approximate -- the flow of the Hamiltonian system to acquire a solution to \cref{prob:cont-least-action}:
A minimizer of \cref{prob:cont-least-action} solves the Hamiltonian system and if that system has a unique solution, this must be it.
Of course, we can only identify the flow once we have determined the optimal $q(1)$ as otherwise we would be lacking an initial value.

\begin{theorem}
	There exists a unique solution $(q, p)$ for the Hamiltonian system in \cref{eq:hamiltonian-system} with $q \in C^2([0, 1], \cX^N)$ and $p \in C^1([0, 1], \cX^N)$.
\end{theorem}

\begin{proof}
	Recall \cref{eq:kernel-feature-map}, which reads $\Gamma(x_1, x_2) = \psi^\mathrm{T}(x_1)\psi(x_2)$.
	Using this equality, we can rewrite \cref{eq:hamiltonian-system} as
	\begin{equation}
	\label{eq:feature-hamiltonian}
		\begin{split}
			\dot{q}_i &= \psi^\mathrm{T}(q_i) \alpha\\
			\dot{p}_i &= -\frac{\partial}{\partial q_i} \left(p_i^\mathrm{T} \psi^\mathrm{T}(q_i) \alpha \right)\ ,
		\end{split}
	\end{equation}
	with $\alpha \coloneqq \sum_{k=1}^{N} \psi(q_k) p_k$.
	From \cref{cor:feature-space-norm} we get:
	\begin{align}
		\norm{\alpha}_\cF^2 &= \norm{\psi^\T(x) \alpha}_\cV^2\\
		&= \norm{\sum_{i=1}^N \psi^\T(x) \psi(q_i)p_i}_\cV^2 \\
		&= \norm{\sum_{i=1}^N \Gamma(x, q_i) p_i}_\cV^2\\
		&= \left<\sum_{i=1}^N\Gamma(x, q_i)p_i, \sum_{j=1}^N \Gamma(x, q_j) p_j \right>_\cV\\
		&= \sum_{i=1}^N \sum_{j=1}^N \left<\Gamma(x, q_i)p_i, \Gamma(x, q_j) p_j \right>_\cV\\
		&= \sum_{i=1}^N \sum_{j=1}^N \left<p_i, \Gamma(q_i, q_j) p_j\right>_\cX\\
		&= p^\T \bGamma(q, q) p \ .
	\end{align}
	In the second to last step we again used the reproducing property of the kernel $\Gamma$.
	This gives us $\norm{\alpha}_\cF^2 = 2 \fH(q, p)$.
	\cref{cor:energy-preservation} states that $\fH$ is constant across time, which implies that $\norm{\alpha}_\cF^2$ is, too (and thus  $\norm{\alpha}_\cF$).
	Our goal is to use a global version of the Picard-Lindelöf theorem \cite[~Theorem 1.2.3]{arino06} to prove that the Hamiltonian system does have a unique solution.
	From this it immediately follows that $q$ and $p$ are $C^1$.
	The right side of the first line in \cref{eq:feature-hamiltonian} is also differentiable by $t$.
	Thus $q$ is $C^2$.
	
	For being able to apply Picard-Lindelöf we need the vector field $(q, p)$ to be globally Lipschitz continuous.
	Showing component-wise Lipschitz continuity suffices because we can choose the global constant $L$ as the maximum of the components'.
	We will show the Lipschitz property by proofing that $\dot{q}$ and $\dot{p}$ are bounded.
	\Todo{Is this bogus math? This looks like a system of PDEs, is Picard-Lindelöf even applicable? Not and yes!
	The derivative wrt time of (q, p) is the componentwise derivative!}
	This will do:
	Let $\dot{q}$ and $\dot{p}$ be bounded by $L \in \R$ and $q_1, p_1, q_2, p_2 \in \cX$, $x \coloneqq (q_1, p_1)^\T, y \coloneqq (q_2, p_2)^\T$ and
	\begin{equation}
	\label{eq:hamiltonian-time-derivative}
		\begin{pmatrix}
			\dot{q}\\\dot{p}
		\end{pmatrix}
		= \frac{\mathrm{d}}{\mathrm{d} t}\begin{pmatrix}q\\p\end{pmatrix}
		= f\left(t, \begin{pmatrix}q\\p\end{pmatrix}\right) 
		\coloneqq 			
		\begin{pmatrix}\bGamma(q, q)p\\ 
		-\frac{\partial \left(\frac{1}{2} p^\mathrm{T} \bGamma(q, q) p\right)}{\partial q}
		\end{pmatrix} 
	\end{equation}
	Then by the mean value theorem there exists a point $z \coloneqq (q^\ast, p^\ast)$ with $q^\ast,\ p^\ast \in \cX$ such that
	\begin{align}
		\norm{f\left(t, x\right) - f\left(t, y\right) }
		= \norm{
			\frac{\mathrm{d}}{\mathrm{d} t}f\left(t, z \right)}
		\cdot \norm{x-y} \ .
	\end{align}
	Because of the equality in \cref{eq:hamiltonian-time-derivative} we immediately arrive at the desired result:
	$\norm{f\left(t, x\right) - f\left(t, y\right) } \leq L \norm{x - y}$.
	For the detailed requirements of the Picard-Lindelöf theorem the reader is referred to \cite{arino06, tenenbaum85}.
	
	All that is left to show now is the boundedness of $\dot{q}$ and $\dot{p}$.
	That of of $\dot{q}$ is easy to see from \cref{eq:feature-hamiltonian}:
	\begin{equation}
		\norm{\dot{q}_i}_\cX \leq \norm{\psi^T(q_i)}_O \norm{\alpha}_\cF 
		\leq \sup_{x \in \cX}\norm{\psi^T(x)}_O \norm{\alpha}_\cF\ .
	\end{equation}
	Here, $\norm{\cdot}_O$ is the operator norm \cite{conway07}.
	Because of the assuption in \cref{cond:feature-condition}, $\psi$ is bounded.
	This implies that the adjoint $\psi^\T$ is, too.
	We already saw that $\alpha$ is constant.
	It remains to show the boundedness of $\dot{p_i}$.
	First consider each entry $\dot{p}_{i, j}$ of $\dot{p}_i$.
	By common differentiation rules and Cauchy-Schwarz we get
	\begin{align}
		\abs{p_{i,j}} &= \abs{\frac{\partial}{\partial q_{i, j}}\left<p_i, \psi^\T(q_i) \alpha \right> }\\
		&= \abs{\left<p_i, \frac{\partial}{\partial q_{i, j}} \psi^\T(q_i) \alpha \right> }\\
		&\leq \norm{p_i}_\cX \cdot \norm{\frac{\partial}{\partial q_{i, j}} \left(\psi^T(q_i) \alpha\right) }_\cX \ .
	\end{align}
	Observe the second factor, $\psi^\T(q_i) \alpha$. $\psi^\T$ is a matrix (actually a bounded linear operator in $L(\cF, \cX)$, but we are finite-dimensional), and $\alpha \in \cF$.
	The matrix-vector product is a continuous, bilinear operator and thus we can apply the product rule for partial derivatives.
	$\alpha$ does not depend on $q_{i, j}$, thus the equality $\frac{\partial}{\partial q_{i, j}} \left(\psi^T(q_i) \alpha\right) = \left(\frac{\partial}{\partial q_{i, j}} \psi^T(q_i)\right)  \alpha$ holds true. The partial derivative of $\psi^\T$ is again a bounded linear operator.
	\begin{equation}
		\norm{\left(\frac{\partial}{\partial q_{i, j}} \psi^T(q_i)\right)  \alpha}_\cX \leq \norm{\frac{\partial}{\partial q_{i, j}} \psi^T(q_i)}_O \norm{\alpha}_\cF \ .
	\end{equation}
	Together with the already established inequality we get
	\begin{align}
		\abs{\dot{p}_{i,j}} &\leq \norm{p_i}_\cX \cdot \norm{\frac{\partial}{\partial q_{i, j}} \left(\psi^T(q_i)\right) \alpha }_\cX\\
		&\leq  \norm{p_i}_\cX \cdot \norm{\frac{\partial}{\partial q_{i, j}} \psi^T(q_i)}_O \cdot \norm{\alpha }_\cF \ .
	\end{align}
	In total, this gives
	\begin{align}
		\norm{\dot{p}_i}^2 &= \sum_{i=1}^N \dot{p}_{i,j}^2\\
		&\leq \norm{p_i}_\cX^2 \cdot \norm{\alpha }_\cF^2 \cdot \sum_{i=1}^N  \norm{\frac{\partial}{\partial q_{i, j}} \psi^T(q_i)}_O^2\\
		& = \norm{p_i}_\cX^2 \cdot \norm{\alpha }_\cF^2 \cdot \norm{\grad \psi^T(q_i)}^2 \\
		& \leq \norm{p_i}_\cX^2 \cdot \norm{\alpha }_\cF^2 \cdot \sup_{x \in \cX}\norm{\grad \psi^T(x)}^2 \ ,
	\end{align}
	which is, of course, equivalent to 
	\begin{equation}
	\label{eq:norm-p-inequality}
		\norm{\dot{p}_i}\leq \norm{p_i}_\cX \cdot \norm{\alpha }_\cF \cdot \sup_{x \in \cX}\norm{\grad \psi^T(x)} \ .
	\end{equation}
	All factors on the right side are bounded: $p_i$ is a continuous function defined on a compact interval, $\alpha$ is constant, as we have seen, and the last term by \cref{cond:feature-condition}.
	This means $\dot{p_i}$ and therefore $\dot{p}$ are bounded, too.
	This concludes the proof.
\end{proof}

See Allansonnière for Theorem 3.7 of paper

\subsection{Geodesic Shooting}

Using the Hamiltonian representation, we can now formulate another equivalent problem.
TODO: Why geodesic shooting?

\begin{problem}
	\label{prob:geodesic-shooting}
	\begin{cases}
		\text{Minimize~}& \frac{\nu}{2} p(0)^\T \bGamma(X, X)p(0) + l(q(1), Y)\\
		\text{such that~} & p = \bGamma(q, q)^{-1}\dot{q},\ q(0) = X \text{~and~} (q,p) \text{~follow Hamilton's equations \ref{eq:hamiltonian-system}} \ .
	\end{cases}
\end{problem}

For later use, we explicitly define the objective function of this problem:
\begin{equation}
	\label{eq:geodesic-shooting-objective}
	\fV(p(0), X, Y) \coloneqq \frac{\nu}{2} p(0)^\T \bGamma(X, X)p(0) + l(q(1), Y) \ .
\end{equation}

\begin{theorem}
\label{theo:geodesic-shooting}
	$q \in C^1([0, 1], \cX^N)$ minimizes \cref{prob:cont-least-action} if and only if with $p\coloneqq \bGamma(q, q)^{-1}\dot{q}$, $p$ minimizes \cref{prob:geodesic-shooting}.
\end{theorem}
\begin{proof}
	Sketch (for now):
	\Todo{The argumentation with Lagrangian = Hamiltonian feels too simple but might be plausible}
	Let $q$ minimize \cref{prob:cont-least-action} and define $p$ as above.
	Then $q(0) = X$ and by \cref{theo:hamiltonian-dynamic} $(q, p)$ satisfy Hamilton's equations.
	We have $\cL(t, q, \dot{q}) = \cH(t, q, p)$ and know from \cref{cor:energy-preservation} that $\cH$ is constant across time.
	This implies that
	\begin{equation}
		\mathcal{A}(q) = \int_{0}^{1} \cL(t, q, \dot{q}) \mathrm{d}t 
		= \int_{0}^{1} \cH(t, q, p) \mathrm{d}t = \cH(\beta, q, p)
	\end{equation}
	for $\beta \in [0, 1]$.
	Therefore $(q, p)$ minimize $\nu \cH(\beta, q, p) + l(q(1), Y)$ and for $\beta = 0$ we get
	\begin{align}
		\nu \cH(0, q, p) + l(q(1), Y) &= \frac{\nu}{2} p(0)^\T \bGamma(q(0), q(0))p(0) + l(q(1), Y)\\
		&= \frac{\nu}{2} p(0)^\T \bGamma(X, X)p(0) + l(q(1), Y) \ .
	\end{align}
	Vice versa, if $p(0)$ minimizes $\fV$ and $(q, p)$ follow the Hamilton's equations, with the same argument, $q$ minimizes \cref{prob:cont-least-action}.
\end{proof}

\subsection{Geodesic Shooting}

Using the Hamiltonian formulation, we can now formulate another problem equivalent to \cref{prob:cont-least-action}.
The search for the optimal trajectory $q$ is reduced to the search for the optimal initial momentum $p(0)$.
A similar method, called \emph{Geodesic Shooting} has been introduced by \citet{allassonniere05} in the field of image registration.
In this method, an optimal control problem is also reduced to finding optimal initial momenta for a Hamiltonian system.
For mechanical regression, the equivalent problem is given as follows.
\begin{problem}
	\label{prob:geodesic-shooting}
	\begin{cases}
		\text{Minimize~}& \frac{\nu}{2} p(0)^\T \bGamma(X, X)p(0) + l(q(1), Y)\\
		\text{such that~} & p = \bGamma(q, q)^{-1}\dot{q},\ q(0) = X \\
		&\text{and~} (q,p) \text{~follow Hamilton's equations \ref{eq:hamiltonian-system}} \ .
	\end{cases}
\end{problem}
For later use, we explicitly define the objective function of this problem:
\begin{equation}
\label{eq:geodesic-shooting-objective}
\fV(p(0), X, Y) \coloneqq \frac{\nu}{2} p(0)^\T \bGamma(X, X)p(0) + l(q(1), Y) \ .
\end{equation}

The following theorem states the equivalence between \cref{prob:cont-least-action} and \cref{prob:geodesic-shooting}, which means that the Least Action formulation can be solved by Geodesic Shooting.
\begin{theorem}
	\label{theo:geodesic-shooting}
	$q \in C^1([0, 1], \cX^N)$ minimizes \cref{prob:cont-least-action} if and only if with $p\coloneqq \bGamma(q, q)^{-1}\dot{q}$, $p$ minimizes \cref{prob:geodesic-shooting}.
\end{theorem}
\begin{proof}
	Let $q$ minimize \cref{prob:cont-least-action} and define $p = \bGamma(q, q)^{-1}\dot{q}$.
	Then $q(0) = X$ and by \cref{theo:hamiltonian-dynamic} $(q, p)$ satisfy Hamilton's equations \ref{eq:hamiltonian-system}.
	We have $\cL(t, q(t), \dot{q}(t)) = \cH(t, q(t), p(t))$ and know from \cref{cor:energy-preservation} that $\cH$ is constant across time, that is for $t_1, t_2 \in [0, 1]$ $\cH(t_1, q(t_1), p(t_1)) = \cH(t_2, q(t_2), p(t_2))$.
	This implies that
	\begin{equation}
	\mathcal{A}(q) = \int_{0}^{1} \cL(t, q(t), \dot{q}(t)) \mathrm{d}t 
	= \int_{0}^{1} \cH(t, q(t), p(t)) \mathrm{d}t = \cH(t_1, q, p) \ .
	\end{equation}
	Therefore $(q, p)$ minimize $\nu \cH(t_1, q(t_1), p(t_1) + l(q(1), Y)$ and for $t_1 = 0$ we get
	\begin{align}
	\nu \cH(0, q(0), p(0)) + l(q(1), Y) &= \frac{\nu}{2} p(0)^\T \bGamma(q(0), q(0))p(0) + l(q(1), Y)\\
	&= \frac{\nu}{2} p(0)^\T \bGamma(X, X)p(0) + l(q(1), Y) \ ,
	\end{align}
	which is the target function $\fV$ defined in \cref{prob:geodesic-shooting}.
	Vice versa, if $p(0)$ minimizes $\fV$ and $(q, p)$ follow Hamilton's equations, with the same argument, $q$ minimizes \cref{prob:cont-least-action}.
\end{proof}

\subsection{Continuous Limit and Adherence Values}


The only thing left to prove now is that as $L \rightarrow \infty$, \cref{prob:min-q} does indeed converge towards \cref{prob:cont-least-action}.

\begin{figure}
	\makebox[\textwidth][c]{
		\centering
		\begin{tikzpicture}[font=\scriptsize, align=center]
				\node[draw, above left, align=center] (a) {
					ResNet \ref{prob:min-v-f}\\
					$\begin{cases}
						\text{Min~} & \frac{\nu \cdot L}{2} \sum_{k=1}^{L} \norm{v_k}_\cV^2
						+ l_R(\Phi_L(X), Y) \\
						\text{s.t.~} & v_1, \ldots, v_L \in \cV,\ \Phi_L = (I + v_L) \circ \ldots \circ (I + v_1)
					\end{cases}$};
				\node[right=of a, align=center,outer sep=-2em] (z1) {\large$\stackrel{L \rightarrow \infty}{\longrightarrow}$\\
					\scriptsize\cref{theo:problem-convergence}};
				\node[draw, right= of z1, align=center] (b) {
					Limit of ResNet \ref{prob:resnet-limit}\\
					$\begin{cases}
						\text{Min~}& \frac{\nu}{2} \int_{0}^{1} \norm{v}_\mathcal{V}^2 \mathrm{d}t
						+ l(\phi_v(X, 1), Y)\\
						\text{s.t.~}& v \in C([0, 1], \mathcal{V}),\ \Phi(x, 0) = x,\\
						&\dot{\Phi}^v(x, t) = \mathbf{\Gamma}(\Phi^v(x, t), q(t)) \bGamma(q(t), q(t))^{-1} \dot{q}(t)
					\end{cases}$};
				\node[below= of a,outer sep=-2em] (z2) {\large{$\big\Updownarrow$} \scriptsize\cref{theo:v-q-problem-equivalence}};
				\node[draw, below= of z2, align=center] (c) {
					Discrete Least Action \ref{prob:min-q}\\
					$\begin{cases}
						\text{Min~} & \frac{\nu}{2} \sum_{k=1}^{L} \left(\frac{q_{k+1} - q_k}{\Delta t}\right)^\mathrm{T} \bGamma(q_k, q_k)^{-1} \left(\frac{q_{k+1} - q_k}{\Delta t}\right) \Delta t\\
						&+ l(q_{L+1}, Y) \\
						\text{s.t.~} & q_1 = X,\ q_2, \ldots, q_{L+1} \in \cX^N,\ t = \frac{1}{L}\\
						& v_k(x) = \bGamma(x, q_k)^\mathrm{T}\bGamma(q_k, q_k)^{-1} (q_{k+1} - q_k)
					\end{cases}$};
				\node[below= of b,outer sep=-2em] (z3) {\large{$\big\Updownarrow$} \scriptsize\cref{theo:v-q-continuous-problem-equivalence}};
				\node[right=of c, align=center,outer sep=-2em] (z4) {\large$\stackrel{L \rightarrow \infty}{\longrightarrow}$\\
					\scriptsize\cref{theo:problem-convergence}};
				\node[draw, below= of z3, right=of z4, align=center] (d) {
					Least Action \ref{prob:cont-least-action}\\
					$\begin{cases}
					\text{Min~} & \frac{\nu}{2} \int_{0}^{1} \dot{q}(t)^\mathrm{T} \mathbf{\Gamma}(q(t), q(t))^{-1}  \dot{q}(t) \mathrm{d}t + l(q(1), Y)\\
					\text{s.t.~} & q \in C^1([0,1], \cX^N),\ q(0) = X
					\end{cases}$};
				\node[below= of d,outer sep=-2em] (z5) {\large{$\big\Updownarrow$} \scriptsize\cref{theo:geodesic-shooting}};
				\node[draw, below= of z5, align=center] (e) {
					Geodesic Shooting \ref{prob:geodesic-shooting}\\
					$\begin{cases}
						\text{Min~} & \frac{\nu}{2} p(0)^\T \bGamma(X, X)p(0) + l(q(1), Y)\\
						\text{s.t.~} & 	p = \bGamma(q, q)^{-1}\dot{q},\ q(0) = X,\ \dot{q} = \bGamma(q, q) p,\\
						&\dot{p} = -\grad_q \left(\frac{1}{2} p^\mathrm{T} \bGamma(q, q) p\right)
					\end{cases}$};
				\node[below= of c,outer sep=-2em] (z6) {\large{$\big\Updownarrow$} \scriptsize\cref{theo:discrete-shooting-min-q-equivalence}};
				\node[draw, below= of z6, align=center] (f) {
					Discrete Geodesic Shooting \ref{prob:discrete-geodesic-shooting}\\
					$\begin{cases}
					\text{Min~} & \frac{\nu}{2} \sum_{k=1}^L p_k^\T \bGamma(q_k, q_k) p_k \Delta t + l(q_{L+1}, Y)\\
					\text{s.t.~} & p_k = \bGamma(q_k, q_k)^{-1} \frac{q_{k+1} - q_k}{\Delta t},\ q_1 = X \\
					& q_{k+1} = q_k + \Delta t \bGamma(q_k, q_k) p_k,\\
					& p_{k+1} = p_k + \frac{\Delta t}{2} \grad_{q_k} \left(p_k^\T \bGamma(q_k, q_k) p_k\right)
					\end{cases}$};
				\node[right=of f, align=center,outer sep=-2em] (z7) {\large$\stackrel{L \rightarrow \infty}{\longrightarrow}$\\
					\scriptsize\cref{theo:problem-convergence}};
		\end{tikzpicture}
	}
	\caption{Summary of problems, their correspondence, their continuous limits and the respective theorems. Discrete problems are on the left, continuous on the right.}
	\label{fig:convergence}
\end{figure}

\subsection{Continuous Residual Neural Network} 
This is equation 1.12 from \cite{owhadi20}.
See how this fits here
\begin{equation}
	\label{eq:phi-v-differential-equation}
	\dot{\Phi}^v(x, t) = \mathbf{\Gamma}(\Phi^v(x, t), q) p
\end{equation}

Let $C([0, 1], \cV)$ be the space of continuous functions $v: [0, 1] \times \cV$, such that the function $v(t)(x)$ is globally Lipschitz in $t$ and $x$.
For simplicity, we will write $v(t, x)$ instead of $v(t)(x)$.
Let $\Phi_v \in C([0, 1], \cV)$ be the solution to the following initial value problem.
\begin{equation}
	\begin{cases}
		&\dot{\Phi}(t, x) = v(\Phi(t, x))\\
		&\Phi(0, x) = x \ .
	\end{cases}
\end{equation}
$\dot{\Phi}(t, x)$ signifies the time derivative $\frac{\mathrm{d}}{\mathrm{d}t}\Phi$.

The idea is that $\Phi_L \stackrel{L \rightarrow \infty}{\longrightarrow} \Phi$ in the sense that for $k \in [L]$, $\phi_k \circ \phi_{k-1} \circ \cdots \circ \phi_1$ is an approximation to $\Phi_v(\frac{k}{L}, \cdot)$.
Similarly, $v_k$ from the discrete formulation approximates $v(\frac{k}{L}, \cdot)$.

\begin{problem}
	\label{prob:resnet-limit}
	\begin{cases}
		\text{Minimize~}& \frac{\nu}{2} \int_{0}^{1} \norm{v(t)}_\cV^2 \mathrm{d}t
		+ l(\Phi_v(1, X), Y)\\
		\text{such that~}& v \in C([0, 1], \cV)\\
	\end{cases}
\end{problem}
Here $\Phi_v(1, X)$ denotes the vector $(\Phi_v(t, X_i))_{i \in [N]}$.

\begin{theorem}
	\label{theo:v-q-continuous-problem-equivalence}
	$v$ minimizes \cref{prob:resnet-limit} if and only if $v$ fulfills
	\begin{equation}
			\dot{\Phi}^v(x, t) = \mathbf{\Gamma}(\Phi^v(x, t), q(t)) \bGamma(q(t), q(t))^{-1} \dot{q}(t)
	\end{equation}
	such that $\Phi_v(x, 0) = x$ for all $x \in X$ and $q$ minimizes \cref{prob:cont-least-action}.
\end{theorem}
The proof of this theorem is very similar to that of \cref{theo:v-q-problem-equivalence}, only that the discrete elements $q_k \in \cX^N$ are replaced by one continuous function $q: [0, 1] \rightarrow \cX^N$.
Within the scope of this thesis, we can only review the main ideas.
\begin{proof}
	Let $v$ be a minimizer of \cref{prob:resnet-limit}.
	First, define the function $q: [0, 1] \rightarrow \cX^N$ by as the vector with elements
	\begin{equation}
		q_i(t) \coloneqq \Phi_v(t, X_i) \ .
	\end{equation}
	Then $\Phi_v(1, X_i) = q_i(t)$ and thus $l(\Phi_v(1, X)) = q(1)$.
	Furthermore $\dot{q_i}(t) = \dot{\Phi}_v(t, X_i)$, or in vector notation: $\dot{q}(t) = \dot{\Phi}_v(t, X)$.
	Because $\Phi_v$ satisfies the differential equation in \cref{eq:phi-v-differential-equation}, it follows that
	\begin{equation}
		\dot{q}(t) = v(t, \Phi_v(t, X)) \ .
	\end{equation}
	Now, the idea is that if $t \rightarrow \min \norm{v(t, \Phi_v(t, \cdot))}_\cV^2$ is in $C([0, 1], \cV)$ -- which requires continuity and the global Lipschitz property -- this function is a minimizer of \cref{prob:resnet-limit}.
	For fixed $t \in [0, 1]$, we face an optimal recovery problem:
	\begin{equation}
		\min\{\norm{v(t, \Phi_v(t, \cdot))}_\cV^2\ |\ v(t, \Phi_v(t, X)) = \dot{q}(t)\} \ .
	\end{equation}
	\cref{eq:optimal-recovery-f} provides an explicit solution to this problem, which is:
	\begin{equation}
		v(t, \Phi_v(t, x)) = \bGamma(\Phi^v(t, x), q(t))\bGamma(q(t), q(t))^{-1}\dot{q}(t) \ .
	\end{equation}
	One can verify that this function is $C([0, 1], \cV)$.
\end{proof}

\subsection{Discrete Geodesic Shooting}

For completeness, we will briefly show the discrete formulation of \cref{prob:geodesic-shooting}.
The derivation involves the theory of discrete Lagrangian and Hamiltonian mechanics (see e.g. \cite{west04}) and is similar to the continuous case.
Consider the discrete Hamiltonian system
\begin{equation}
\label{eq:discrete-hamiltonian-system}
	\begin{split}
		q_{k+1} &= q_k + \Delta t \bGamma(q_k, q_k) p_k\\
		p_{k+1} &= p_k + \frac{\Delta t}{2} \grad_{q_{k+1}} \left(p_{k+1}^\T \bGamma(q_{k+1}, q_{k+1}) p_{k+1}\right)\ .
	\end{split}
\end{equation}
The discrete geodesic shooting formulation is as follows:
\begin{problem}
	\label{prob:discrete-geodesic-shooting}
	\begin{cases}
		\text{Minimize~} & \frac{\nu}{2} \sum_{k=1}^L p_k^\T \bGamma(q_k, q_k) p_k \Delta t + l(q_{L+1}, Y)\\
		\text{such that~} & p_k = \bGamma(q_k, q_k)^{-1} \frac{q_{k+1} - q_k}{\Delta t},\ q_1 = X, \Delta t = \frac{1}{L} \\
		&\text{~and~} (q_k, p_k) \text{~follow the discrete Hamiltonian equations \ref{eq:discrete-hamiltonian-system}}\ .
	\end{cases}
\end{problem}
Let $\fV_L(p_1, X, Y)$ denote the problem's objective function.
We end up with the following result.
\begin{theorem}
	\label{theo:discrete-shooting-min-q-equivalence}
	$q_1, \dots, q_{L+1}$ minimizes \cref{prob:min-q} if and only if with $p_k \coloneqq \bGamma(q_k, q_k)^{-1} \frac{q_{k+1} - q_k}{\Delta t}$, $p_1$ minimizes \cref{prob:discrete-geodesic-shooting}.
\end{theorem}

\subsection{Existence of Minimizers}

We already saw the equivalence of \cref{prob:cont-least-action,prob:geodesic-shooting,prob:resnet-limit}, in the sense that their minimizers bijectively correspond to each other -- that is, if they exist.
Now we will show that there do indeed exist such minimizers and also that the problems' minimal values are identical.
\begin{theorem}[Existence of minimizers for the continuous problems]
	\label{theo:continuous-solutions-existence}
		There exist minimal points for \cref{prob:cont-least-action,prob:geodesic-shooting,prob:resnet-limit} and their minimal values are identical.
\end{theorem}

\begin{proof}
	First, we show the existence of minimizers:
	\cref{theo:geodesic-shooting,theo:v-q-continuous-problem-equivalence} state that we can obtain a minimizer for each problem from a minimizer $p(0)$ of the geodesic shooting formulation in \cref{prob:geodesic-shooting}.
	Hence we just have to show the existence of such a $p(0)$.
	Define the ball $B_\rho \coloneqq \{p(0) \in \cX^N\ > \ \norm{p(0)}_{\cX^N}^2 \leq \rho^2 \}$ ( $\norm{\cdot}_{\cX^N}$ is the norm induced by the inner product on the product space $\cX^N$).
	Since we required $\cX$ to be finite dimensional, so is $\cX^N$.
	This implies that $B_\rho$ is compact.
	We now want to show the existence of a local minimum of $p(0) \rightarrow \fV(p(0), X, Y)$ on $B_\rho$.
	If we can show that $\fV$ is continuous in $p(0)$, this follows from the extreme value theorem.
	
	Furthermore, \cref{cond:feature-condition} (2) states that there exists an $r > 0$ such that for all $Z \in \cX^N$ $Z^\T \bGamma(X, X) Z \geq r Z^\T Z = r \norm{Z}_{\cX^N}^2$.
	\cref{cond:feature-condition} (4) requires the loss $l$ to be positive.
	These two facts imply that
	\begin{align}
		\lim_{\norm{p(0)} \rightarrow \infty} \fV(p(0), X, Y) 
		&= \lim_{\norm{p(0)} \rightarrow \infty} p(0)^\T \bGamma(X, X) p(0) + l(q(1), Y)\\
		& \geq 	\lim_{\norm{p(0)} \rightarrow \infty} p(0)^\T \bGamma(X, X) p(0) \\
		& \geq \lim_{\norm{p(0)} \rightarrow \infty} r \norm{p(0)}_{\cX^N}^2
		= \infty \ .
	\end{align}
	This means that there exists a $\rho > 0$ such that $B_\rho$ contains a global minimizer of $\fV$ and even a $\rho > 0$ such that all global minimizers are contained in a $B_\rho$.
	
	Now we are left with the proof that $\fV$ is continuous in $p(0)$.
	The first term of $\fV$ is a quadratic form in $p(0)$ and hence continuous.
	The second term -- the loss $l$ -- is continuous in $q(1)$.
	In order to see that $q(1)$ is continuous in $p(0)$ we use a result from the theory of ODEs.
	Again, write
	\begin{equation}
		\left(\dot{q}(t), \dot{p}(t)\right)^\T = f\left(t, \left(q(t), p(t)\right)^\T \right)
	\end{equation}
	for the Hamiltonian system in \cref{eq:hamiltonian-system}.
	We already know that $f$ is bounded for $t \in [0, 1]$, because $q(t)$ and $p(t)$ are.
	$f$ is continuous in $\left(q(t), p(t)\right)^\T$ and so is its Jacobian, as one can easily verify.
	The continuity of the Jacobian implies its boundedness.
	Let $(q_i, p_i)^\T$ be the unique solution to the Hamiltonian system $f$ with initial conditions $q_i(0) = X, p(0) = p_{0, i}$ for $i \in [2]$ (which exists according to \cref{theo:hamiltonian-system-solution}).
	Now, we use a result from the theory of ODEs:
	From \cite[Theorem~1.4.1]{arino06} it follows as a special case that:
	For all $\epsilon > 0$ there exists $\delta > 0$ such that:
	\begin{equation}
		\norm{p_{0, 1} - p_{0, 2}}_{\cX^{2N}} < \delta \Rightarrow 
		\left(\forall t \in [0, 1]:\ \norm{(q_1(t), p_1(t))^\T - (q_2(t), p_2(t))^\T}_{\cX^{2N}} < \epsilon \right) \ .
	\end{equation}
	Since $\norm{\cdot}_{\cX^{2N}}$ is the norm that is induced on the product space, it follows that especially $\norm{q_1(t) - q_2(t)}_{\cX^N} < \epsilon$, which proves the continuity of $q(1)$ in $p(0)$.
	
	The identity of the minimal values can be deduced directly from the theorems \cref{theo:v-q-continuous-problem-equivalence,theo:geodesic-shooting} and their proofs.
\end{proof}

An equivalent result as \cref{theo:continuous-solutions-existence} can be derived for the discrete case.
\begin{theorem}[Existence of solutions to the discrete problems]
	\label{theo:discrete-solutions-existence}
	There exist minimal points for \cref{prob:discrete-geodesic-shooting,prob:min-v-f,prob:min-q} and their minimal values are identical.
\end{theorem}
The proof is practically identical to the proof for the continuous case and will not be conducted here.

\subsection{Convergence}

It states convergence results for all three discrete problems we have dealt with so far.
Let $M_L(X, Y)$ be the set of minimizers of \cref{prob:discrete-geodesic-shooting}.
Respectively, let $M(X, Y)$ be the set of minimizers of \cref{prob:geodesic-shooting}.
\cref{theo:continuous-solutions-existence,theo:discrete-solutions-existence} imply that both sets are non-empty.
\begin{theorem}[Convergence theorem]
	\label{theo:problem-convergence}
	The minimal value of \cref{prob:discrete-geodesic-shooting,prob:min-q,prob:min-v-f} converges towards the minimal value of \cref{prob:cont-least-action,prob:geodesic-shooting,prob:resnet-limit} as $L \rightarrow \infty$.
	Furthermore, the limit of $M_L(X, Y)$ is contained in $M(X, Y)$:
	\begin{equation}
	\label{eq:limit-adherence}
		\bigcap_{L' \in \mathbb{N}} \closure\left(\bigcup_{L' \geq L} M_L(X, Y)\right) \subseteq M(X, Y) \ .
	\end{equation}
\end{theorem}
Here, $\closure (A)$ is the closure of the set $A$.
We will sketch the proof:
\begin{proof}
	From the previous equivalence theorems in this section we know that we can parameterize the minimal values and points by the initial momentum $p_1$ and $p(0)$ for the discrete \ref{prob:discrete-geodesic-shooting} and continuous geodesic shooting problem \ref{prob:geodesic-shooting} respectively.
	This means if suffices to show the convergence of $\fV_L(\cdot, X, Y)$ towards $\fV(\cdot, X, Y)$.
	Let $p_1$ minimize $\fV_L$.
	Then $p_1, \ldots, p_L$, $q_1, \ldots, q_L$ follow the discrete Hamiltonian system \ref{eq:discrete-hamiltonian-system}.
	It can be shown that this system converges uniformly to the continuous Hamiltonian system in \cref{eq:hamiltonian-system} in the sense that as $L \rightarrow \infty$, the interpolation of the solutions $p_k, q_k, k \in [L]$ form a trajectory that follows the continuous Hamiltonian system (we will not prove this here).
	This implies
	\begin{equation}
		\frac{\nu}{2} \sum_{k=1}^L p_k^\T \bGamma(q_k, q_k) p_k \Delta t + l(q_{L+1}, Y) 
		\stackrel{L \rightarrow \infty}{\longrightarrow} \frac{\nu}{2} \int_{0}^{1} p(t)^\T \bGamma(q(t), q(t)) p(t) \mathrm{d} t + l(q(1), Y) \ . 
	\end{equation}
	But because the Hamiltonian is constant along the $q, p$, the latter term is equal to $\frac{\nu}{2} p(0)^\T \bGamma(X, X) p(0) + l(q(1), Y)$.
	This shows the convergence of $\fV_L(\cdot, X, Y)$ towards $\fV(\cdot, X, Y)$, which is even uniform.
	In the proof of \cref{theo:continuous-solutions-existence} we saw that there exists a $\rho$ such that $B_\rho$ contains all global minima of $\fV$.
	This can also be shown for $\fV_L$, where $\rho$ can be chosen independently of $L$.
	Hence, choose $\rho$ such that all global minima of $\fV_L$ and $\fV$ are contained in $B_\rho$.
	Then, it follows that
	\begin{equation}
		\lim_{L \rightarrow \infty} \min_{p_1 \in B_\rho} \fV_L(p_1, X, Y) = \min_{p_1 \in B_\rho} \lim_{L \rightarrow \infty}  \fV_L(p_1, X, Y) = \min_{p_1 \in B_\rho} \fV(p_1, X, Y) \ .
	\end{equation}

	For the second part, let $(p_k)_{k \in \mathbb{N}}, p_k \in M_k(X, Y)$ be a sequence of optimal initial momenta.
	Let $p(0)$ be one of its adherence values, that is, a subsequence $(p_{k_n})_{n \in \mathbb{N}}$ converges towards $p(0)$.
	Then, $\lim_{n\rightarrow \infty} \fV_{k_n}(p_{k_n}) = \fV(p(0))$ and $p(0)$ minimizes $\fV$.
	In order to see that, let $\epsilon > 0$. 
	Because of the uniform convergence of $\fV_k$ towards $\fV$, there exists an $n_0 \in \mathbb{N}$ such that for all $n \geq n_0$:
	\begin{align}
		\abs{\fV_{k_n}(p_{k_n}) - \fV(p(0))} 
		&\leq \abs{\fV_{k_n}(p_{k_n}) - \fV_{k_n}(p(0))} + \abs{\fV_{k_n}(p(0)) - \fV(p(0))}\\
		&< \frac{\epsilon}{2} + \frac{\epsilon}{2} = \epsilon \ .
	\end{align}
\end{proof}

In \cref{eq:limit-adherence}, \citet{owhadi20} claims that equality holds.
However, this could not be shown here.

This concludes the section on mechanical regression.