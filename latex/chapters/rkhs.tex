In this section we define Reproducing Kernel Hilbert Spaces and show some of their basic properties.
There are multiple possibilities for the definition which are equivalent and lead to the same results.

A quick reminder: A \emph{Hilbert space} is a Banach space with an inner product.

Definition of RKHS of functions $\cX \rightarrow \R$ in \cite{sejdinovic12}, where $\cX$ is a set.
Now we want to define RKHS functions $\cX \rightarrow \cY$ where both $\cX$ and $\cY$ are finite Hilbert spaces \Todo{Actually we only require Y to be Hilbert}

Let $L(\cX, \cY)$ be the space of bounded linear functions $\cX \rightarrow \cY$ and $\cH$ be a Hilbert space of functions $\cX \rightarrow \cY$.

\begin{definition}
	\label{def:kernel}
	A function $K: \cX \times \cX \rightarrow L(\cY, \cY)$ is a \emph{Kernel} if
	\begin{enumerate}
		\item $K$ is Hermitian, that is $K(x_1, x_2) = K(x_2, x_1)^\T$ for all $x_1, x_2 \in \cX$. 
		$K(x_2, x_1)^\T$ is the adjoint of $K(x_1, x_2)$ with respect to $\left<\cdot, \cdot\right>$.
		\item $K$ is non-negative, that is for all $n \in \mathbb{N}$ and $((x_i, y_i))_{i \in [n]}, x_i \in \cX, y_i \in \cY$
		\begin{equation}
			\sum_{i=1}^n \sum_{j=1}^n \left< y_i, K(x_i, x_j)  y_j\right> \geq 0 \ .
		\end{equation}
	\end{enumerate}
\end{definition}

Such kernels are also called \emph{operator-valued Kernels} as their target space is the space of operators.
This generalizes the definition of scalar-valued kernels, which are functions $k: \cX \times \cX \rightarrow \R$.
In some cases a kernel is just a function $K: \cX \times \cX \rightarrow L(\cY, \cY)$ without the requirement of it being Hermitian and non-negative \cite{sejdinovic12}.
As we will not require kernels without those properties, we include them in the definition.
\Todo[inline]{Maybe equivalent that block operator matrix is positive semidef for all n}
The definition follows \citet{kadri16}, which examine Reproducing Kernel Hilbert Spaces in the setting where $\cY$ is a Hilbert space of functions.
However, their definition also applies to more general case where $\cY$ is an arbitrary Hilbert space.
\begin{definition}
	\label{def:rkhs}
	A \emph{Reproducing Kernel Hilbert Space} (RKHS) is a Hilbert Space $(\cH, \left<\cdot, \cdot\right>_\cH)$ of functions $f: \cX \rightarrow \cY$ together with a non-negative Kernel $K: \cX \times \cX \rightarrow L(\cY, \cY)$, such that:
	\begin{enumerate}
		\item For all $x\in \cX, y \in \cY$ the function $K(\cdot, x) y: \cX \rightarrow \cY$ lies in $\cH$.
		\item $\cH$ and $K$ fulfill the \emph{Reproducing Property}: For all $f \in \cH, x \in \cX, y \in \cY$
		\begin{equation}
			\label{eq:reproducing-property}
			\left< f, K(\cdot, x) y \right>_\cH = \left< f(x), y\right>_\cY \ .
		\end{equation}
	\end{enumerate}
\end{definition}

In an RKHS, the kernel is also called the \emph{reproducing kernel}.
This definition of an RKHS is in accord with earlier definitions \cite{berlinet04, sejdinovic12} which were stricter:
Only $\cY = \R$ was permitted.
However, as $\R$ together with the standard scalar product is a Hilbert space itself, it is easy to see that the definition above includes this as a special case.

Throughout this work, the scalar product $\left< x_1, x_2 \right>_\cH$ of a Hilbert Space $\cH$ will often be abbreviated with the common notation $x_1^\T x_2$ whenever appropriate.

One important result for RKHSs is that a kernel is associated with a unique RKHS and vice versa, which we will prove in the following theorem.
\begin{theorem}
	\label{theo:kernel-for-rkhs}
	The reproducing kernel of a RKHS $\cH$ is unique.
	Conversely, if $K$ is a kernel there exists a unique RKHS $\cH$ so that $K$ is the reproducing kernel of $\cH$.
\end{theorem}
\begin{proof}[Proof of the first part]
	Assume $K, K^\prime$ are two reproducing kernels for $\cH$ and let $f \in \cH, x \in \cX, y \in \cY$.
	Then
	\begin{equation}
		\left< f, K(\cdot, x)y\right>_\cH = \left<f(x), y\right>_\cY = \left<f, K^\prime(\cdot, x) y\right>_\cH \ .
	\end{equation}
	This implies
	\begin{equation}
		0 = \left< f, K(\cdot, x)y\right>_\cH - \left< f, K^\prime(\cdot, x)y\right>_\cH
		= \left< f, \left(K(\cdot, x) - K^\prime(\cdot, x)\right)y\right>_\cH \ .
	\end{equation}
	This equation can only hold true for all $f, x$ and $y$ if $K(\cdot, x) - K^\prime(\cdot, x) = 0$.
	But this means that for all $x^\prime \in \cX$ $K(x^\prime, x) = K^\prime(x^\prime, x)$.	
\end{proof}

We will not prove the second part here.
It involves a comparatively long construction which would be out of the scope of this work.
Essentially, a pre-Hilbert space $\cH_0$ of functions $f(\cdot) = \sum_{i=1}^n K(\cdot, x_i) y_i$ with $x_i \in \cX, y_i \in \cY$ is constructed.
$\cH_0$ can then be completed to form a Hilbert space, which turns out to be an RKHS.
A full proof for the scalar-case can be found in the work of \citet{aronszajn50} and in \cite{sejdinovic12, berlinet04}.
An extension to the more general case of operator-valued kernels is given in \cite{kadri16}.

\citet{kadri16} give a simple example for an RKHS which is vector-valued, i.e. 
\begin{example}
	\Todo[inline]{Vielleicht findet sich hier noch ein cooles Beispiel.}
	
\end{example}