In this section we define Reproducing Kernel Hilbert Spaces and show some of their basic properties.
For the definition, we will follow \cite{owhadi20}.

%Definition of RKHS of functions $\cX \rightarrow \R$ in \cite{sejdinovic12}, where $\cX$ is a set.
%Now we want to define RKHS functions $\cX \rightarrow \cY$ where both $\cX$ and $\cY$ are finite Hilbert spaces 

In the following, let $\cX$ be a set and $\cY$ an arbitrary, finite dimensional Hilbert space with inner product $\left<\cdot, \cdot \right>_\cY$.
If $\cY$ and $\mathcal{Z}$ are Hilbert spaces, denote the space of bounded linear operators $\cY \rightarrow \mathcal{Z}$ with $L(\cY, \mathcal{Z})$ .
We start by defining kernel functions.
\begin{definition}
	\label{def:kernel}
	A function $K: \cX \times \cX \rightarrow L(\cY, \cY)$ is called a \emph{kernel} if
	\begin{enumerate}
		\item $K$ is Hermitian, that is $K(x_1, x_2) = K(x_2, x_1)^\T$ for all $x_1, x_2 \in \cX$, where $K(x_2, x_1)^\T$ is the adjoint of $K(x_1, x_2)$ with respect to $\left<\cdot, \cdot\right>_\cY$.
		\item $K$ is non-negative, that is for all $n \in \mathbb{N}$ and $((x_i, y_i))_{i \in [n]}, x_i \in \cX, y_i \in \cY$
		\begin{equation}
			\sum_{i=1}^n \sum_{j=1}^n \left< y_i, K(x_i, x_j)  y_j\right> \geq 0 \ .
		\end{equation}
	\end{enumerate}
\end{definition}

Such kernels are also called \emph{operator-valued kernels} as their target space is a space of linear operators.
This generalizes the definition of scalar-valued kernels, which are functions $k: \cX \times \cX \rightarrow \R$.
%In other literature \cite{sejdinovic12}, kernels are defined as a function $K: \cX \times \cX \rightarrow L(\cY, \cY)$ without them being Hermitian and non-negative.

\begin{definition}
	A kernel $K: \cX \times \cX \rightarrow L(\cY, \cY)$ is called \emph{non-degenerate} if for each $n \in \mathbb{N}$ and all pairwise distinct $x_i \in \cX, y_i \in \cY, i \in [n]$
	\begin{equation}
		\sum_{i=1}^n \sum_{j=1}^n \left< y_i, K(x_i, x_j)  y_j\right> = 0 \Leftrightarrow \forall i \in [n]:~ y_i = 0
	\end{equation}
\end{definition}

\begin{lemma}
	\label{lem:kernel-non-singular}
	Let the kernel $K$ be non-degenerate.
	Then for all $n \in \mathbb{N}$ and pairwise distinct $x_i \in \cX, i \in [n]$ the matrix with entries $(K(x_i, x_j))_{i, j \in [n]}$ is non-singular.
\end{lemma}
\begin{proof}
	If the kernel of $(K(x_i, x_j))_{i, j \in [n]}$ contained a non-zero element $Y \in \cY^n$, it would follow that $\sum_{i=1}^n \sum_{j=1}^n \left< y_i, K(x_i, x_j)  y_j\right> = 0$; a contradiction.
\end{proof}
In \cref{sec:block-operator-notation} we will further explore matrices whose elements are operators.

Next, we will define Reproducing Kernel Hilbert Spaces (RKHSs).
The definition follows \citet{kadri16}, which examine RKHS in the setting where $\cY$ is a Hilbert space of functions.
However, their definition also applies to the more general cases where $\cY$ is an arbitrary Hilbert space.
\begin{definition}
	\label{def:rkhs}
	A \emph{Reproducing Kernel Hilbert Space} (RKHS) is a Hilbert space $(\cH, \left<\cdot, \cdot\right>_\cH)$ of functions $f: \cX \rightarrow \cY$ together with a kernel $K: \cX \times \cX \rightarrow L(\cY, \cY)$, such that
	\begin{enumerate}
		\item For all $x\in \cX, y \in \cY$ the function $K(\cdot, x) y: \cX \rightarrow \cY$ lies in $\cH$.
		\item $\cH$ and $K$ fulfill the \emph{Reproducing Property}: For all $f \in \cH, x \in \cX, y \in \cY$
		\begin{equation}
			\label{eq:reproducing-property}
			\left< f, K(\cdot, x) y \right>_\cH = \left< f(x), y\right>_\cY \ .
		\end{equation}
	\end{enumerate}
\end{definition}

In an RKHS, the kernel is also called the \emph{reproducing kernel}.
This definition of an RKHS is in accordance with earlier definitions \cite{berlinet04, sejdinovic12} which only permit $\cY = \R$.
However, as $\R$ together with the standard scalar product is a Hilbert space itself, it is easy to see that the definition above includes this special case, which we will refer to as the scalar case.

One important result for RKHSs is that a kernel is associated with a unique RKHS and vice versa:
\begin{theorem}
	\label{theo:kernel-for-rkhs}
	The reproducing kernel of a RKHS $\cH$ is unique.
	Conversely, if $K$ is a kernel there exists a unique RKHS $\cH$ so that $K$ is the reproducing kernel of $\cH$.
\end{theorem}
\begin{proof}[Proof of the first part]
	Assume $K, K^\prime$ are two reproducing kernels for $\cH$ and let $f \in \cH, x \in \cX, y \in \cY$.
	Then
	\begin{equation}
		\left< f, K(\cdot, x)y\right>_\cH = \left<f(x), y\right>_\cY = \left<f, K^\prime(\cdot, x) y\right>_\cH \ .
	\end{equation}
	This implies
	\begin{equation}
		0 = \left< f, K(\cdot, x)y\right>_\cH - \left< f, K^\prime(\cdot, x)y\right>_\cH
		= \left< f, \left(K(\cdot, x) - K^\prime(\cdot, x)\right)y\right>_\cH \ .
	\end{equation}
	This equation can only hold true for all $f, x$ and $y$ if $K(\cdot, x) - K^\prime(\cdot, x) = 0$.
	But this means that for all $x^\prime \in \cX$ $K(x^\prime, x) = K^\prime(x^\prime, x)$.	
\end{proof}

We will not prove the second part here.
It involves a comparatively long construction via a pre-Hilbert space $\cH_0$ of functions $f(\cdot) = \sum_{i=1}^n K(\cdot, x_i) y_i$ with $x_i \in \cX, y_i \in \cY$.
$\cH_0$ can then be completed to a Hilbert space, which turns out to be an RKHS.
A full proof for the scalar-case can be found in the work of \citet{aronszajn50} and in \cite{sejdinovic12, berlinet04}.
An extension to the more general case of operator-valued kernels is given in \cite{kadri16}.

Throughout this work, the scalar product $\left< x_1, x_2 \right>_\cH$ of a Hilbert Space $\cH$ will often be abbreviated with the common notation $x_1^\T x_2$ whenever appropriate.