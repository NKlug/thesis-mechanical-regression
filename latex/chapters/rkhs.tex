\section{Reproducing Kernel Hilbert Spaces}

\begin{definition}[RKHS]
	\label{def:rkhs}
	content...
\end{definition}

\begin{theorem}
	\label{theo:kernel-for-rkhs}
	For every RKHS $\mathfrak{H}$ there exists a unique associated Kernel $K$.
\end{theorem}
\begin{proof}
	content...
\end{proof}

Example: Gaussian/RBF-Kernel, Polynomial Kernel

\subsection{Product Space and Block Operator Notation}
Let $\cY^N \coloneqq \bigtimes_{i=1}^N \cY$ be the $N$-fold product space and $A, B \in \cY^N$.
Defining $\left < A, B \right >_{\cY^N} \coloneqq \sum_{i=1}^{N} \left <A_i, B_i \right >$ fulfills the properties of an inner product on $\cY^N$ and thus makes $\cY^N$ a Hilbert Space itself.

\begin{definition}[Block Operator Matrix]
	content...
\end{definition}

\begin{figure}
	This will be an illustration of isomorphisms in the block operator matrix example.
	\caption{And this will be the supercool caption.}
	\label{fig:block-operator-example}
\end{figure}

\begin{example}
	Let $\cY = \R^n$.
	Then we have $\cL(\cY) \cong \R^{n\times n}$, which means applying a (bounded) linear operator to a $y \in \cY$ can be expressed as matrix-vector multiplication.
	Then $\cL(\cY^N) \cong \R^{nN \times nN}$:
	To see this, let $\mathbf{A} \in \cL(\cY^N)$ and $\mathbf{y} \in \cY^N$. 
	Write $\mathbf{A} = (\mathbf{A}_{i, j})$ with $\mathbf{A}_{i,j} \in \cL(\cY)$. 
	We obtain the corresponding $(nN \times nN)$-matrix by first applying the canonical isomorphism $\cL(\cY) \cong \R^{n\times n}$ to each $\mathbf{A}_{i,j}$ and then "flattening" the result.
	Furthermore, we have another isomorphism $\cY^N \cong \R^{nN}$ which is given by consecutively pasting the elements of $\cY^N$ into one vector of size $nN$.
	\autoref{fig:block-operator-example} illustrates both isomorphisms.
	
	These new representations of $\mathbf{A}$ and $\mathbf{y}$ make it particularly easy to compute $\mathbf{A}\cdot \mathbf{y} = \mathbf{x}$:
	Simply map $\mathbf{A}$ to $A \in \R^{nN \times nN}$ and $\mathbf{y}$ to $y \in \R^{nN}$, calculate $A \cdot y = x$ as the standard matrix-vector product and map the resulting vector $x$ back to $\mathbf{x} \in\cY^N$.	
	\Todo{Image: Block Operator Matrix -> "flattened" Matrix}
\end{example}


\subsection{Optimal Recovery}

\begin{theorem}[Representer Theorem]
	\label{theo:representer}
\end{theorem}

The optimal recovery loss is defined as
\begin{equation}
	\label{eq:optimal-recovery-loss}
	l(X, Y) \coloneqq \min\{\norm{f}_\mathcal{H} ~|~ f\in \mathcal{H} \text{~and~} f(X) = Y\}
\end{equation}

By \autoref{theo:representer}, the $f \in \mathcal{H}$ which minimizes \autoref{eq:optimal-recovery-loss} admits a representation as 
\begin{equation}
	f(x) = \sum_{i=1}^N K(x, X_i) Z_i \ ,
\end{equation}
where the $Z_i$ are the solutions to the following linear system ($1 \leq j \leq N$):
\begin{equation}
	\sum_{i=1}^{N} K(X_j, X_i) Z_i = Y_j
\end{equation}

Using Block Operator notation

\subsection{Ridge Regression}

\begin{equation}
	\label{eq:ridge-regression-loss}
	l(X, Y) \coloneqq \inf_{f \in \mathcal{H}} \lambda \norm{f}_\mathcal{H}^2 
	+ l_\cY (f(X), Y)
\end{equation}