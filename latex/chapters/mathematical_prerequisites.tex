\section{Mathematical Preliminaries}

This section provides brief introductions to some of the theories involved in \cref{sec:mechanical-regression}.
Those involve some functional analysis, in particular Reproducing Kernel Hilbert Spaces, features spaces and feature maps, some notation for product spaces and specific loss functions.

\subsection{Reproducing Kernel Hilbert Spaces}

\section{Reproducing Kernel Hilbert Spaces}

\begin{definition}[RKHS]
	\label{def:rkhs}
	content...
\end{definition}

\begin{definition}[Kernel]
	\label{def:kernel}
\end{definition}

\begin{theorem}
	\label{theo:kernel-for-rkhs}
	For every RKHS $\mathfrak{H}$ there exists a unique associated Kernel $K$.
\end{theorem}
\begin{proof}[Idea of proof]
	content...
\end{proof}

Example: Gaussian/RBF-Kernel, Polynomial Kernel

\subsection{Features Space and Feature Maps}

\begin{equation}
	\label{eq:kernel-feature-map}
	K(x_1, x_2) = \psi^\T(x_1)\psi(x_2)
\end{equation}

\begin{theorem}
	\begin{equation}
		\left< \psi^\T(\cdot) \alpha, \psi^\T(\cdot) \beta\right>_\mathcal{H} = \left< \alpha, \beta\right>_\mathcal{F} \ .
	\end{equation}
\end{theorem}
\begin{corollary}
	\label{cor:feature-space-norm}
	\begin{equation}
		\norm{\psi^\T(\cdot)\alpha}_\mathcal{H}^2 = \norm{\alpha}_\mathcal{F}^2 \ .
	\end{equation}
\end{corollary}

\subsection{Product Space and Block Operator Notation}
Later on, it would be quite cumbersome to always treat the $X_i$ separately.
Hence we will now introduce block operator syntax which will enable us to write equations more concisely.
As in later parts we will almost exclusively rely on this notation, we first have to make sure that all used syntax is mathematically correct.
In the following we will define the product space of $\cY$, some useful notation and assure that all future calculations involving that new notation are mathematically sound.

Let $\cY^N \coloneqq \bigtimes_{i=1}^N \cY$ be the $N$-fold product space and $A, B \in \cY^N$.
Defining $\left < A, B \right >_{\cY^N} \coloneqq \sum_{i=1}^{N} \left <A_i, B_i \right >$ fulfills the properties of an inner product on $\cY^N$ and thus makes $\cY^N$ a Hilbert Space itself.


\begin{definition}[Block Operator Matrix]
	A \emph{Block Operator Matrix} $\mathbf{A}$ is an $(N \times N)$-Matrix with Elements in $\cL(Y, Y)$.
\end{definition}

\begin{definition}[$\cY^N$]
	content...
\end{definition}
Explain: We can express the corresponding inner product as matrix multiplication.

\begin{theorem}
	$(\cL(\cY^N, \cY^N), +, \circ)$ is a  ring.
\end{theorem}
\begin{proof}
	We conduct the proof by showing that $\cL(\cY^N, \cY^N) \cong \R^{nN \times nN}$.
\end{proof}

As a consequence of the proof, we get the following corollary.
\begin{corollary}
	\label{cor:matrix-ring-equivalence}
	$\cL(\cY^N, \cY^N) \cong \R^{nN \times nN}$.
\end{corollary}

\begin{corollary}
	\label{cor:matrix-vector-equivalence}
	Block-Operator Matrix times vector is isomorphic to standard matrix times vector.
\end{corollary}

\begin{figure}
	This will be an illustration of isomorphisms in the block operator matrix example.
	\caption{And this will be the supercool caption.}
	\label{fig:block-operator-example}
\end{figure}

\begin{example}
	Let $\cY = \R^n$.
	\Todo{Discard Example, make it lemma/theorem.
		Replace $\R$ by general field, as Hilbert Spaces are vector spaces,
		make first isomorphism monoid homomorphism, maybe even ring homomorphism.
		This means we can work with $\mathbf{K}$ just like a regular matrix.
		Check if $\mathbf{K}$ really is symmetric or not just hermitian.
		Also prove that there is an inverse under certain circumstances.}
	Then we have $\cL(\cY) \cong \R^{n\times n}$, which means applying a (bounded) linear operator to a $y \in \cY$ can be expressed as matrix-vector multiplication.
	We now claim that in this case $\cL(\cY^N) \cong \R^{nN \times nN}$ and that the computation of $\mathbf{A}\cdot \mathbf{y}$ can be expressed as a standard matrix-vector product.
	For the first part, let $\mathbf{A} \in \cL(\cY^N)$ and $\mathbf{y} \in \cY^N$. 
	Write $\mathbf{A} = (\mathbf{A}_{i, j})$ with $\mathbf{A}_{i,j} \in \cL(\cY)$. 
	We obtain the corresponding $(nN \times nN)$-matrix by first applying the canonical isomorphism $\cL(\cY) \cong \R^{n\times n}$ to each $\mathbf{A}_{i,j}$ and then "flattening" the result.
	For the second part, we first have to consider another isomorphism $\cY^N \cong \R^{nN}$ which is given by consecutively pasting the elements of $\cY^N$ into one vector of size $nN$.
	\cref{fig:block-operator-example} illustrates both isomorphisms.
	
	These new representations of $\mathbf{A}$ and $\mathbf{y}$ make it particularly easy to compute $\mathbf{A}\cdot \mathbf{y} = \mathbf{x}$:
	Simply map $\mathbf{A}$ to $A \in \R^{nN \times nN}$ and $\mathbf{y}$ to $y \in \R^{nN}$, calculate $A \cdot y = x$ as the standard matrix-vector product and map the resulting vector $x$ back to $\mathbf{x} \in\cY^N$.	
	\Todo{Image: Block Operator Matrix -> "flattened" Matrix}
\end{example}

With $K$ being the Kernel associated with the RKHS $\mathcal{H}$ and $A, B \in \cY^N$ we will write $\mathbf{K}(A, B)$ for the matrix with the entries $K(A_i, B_j)$ at position $(i, j)$.


\subsection{Optimal Recovery}
\label{sec:optimal-recovery}

\begin{theorem}[Representer Theorem]
	\label{theo:representer}
\end{theorem}

The optimal recovery loss is defined as
\begin{equation}
	\label{eq:optimal-recovery-loss}
	l(X, Y) \coloneqq \min\{\norm{f}_\mathcal{H}^2 ~|~ f\in \mathcal{H} \text{~and~} f(X) = Y\}
\end{equation}

By \cref{theo:representer}, the $f \in \mathcal{H}$ which minimizes \cref{eq:optimal-recovery-loss} admits a representation as 
\begin{equation}
	\label{eq:optimal-recovery-f}
	f(x) = \sum_{i=1}^N K(x, X_i) Z_i \ ,
\end{equation}
where the $Z_i$ are the solutions to the following linear system ($1 \leq j \leq N$):
\begin{equation}
	\sum_{i=1}^{N} K(X_j, X_i) Z_i = Y_j
\end{equation}
For better readability we translate this expression into block operator notation:
Introduce the block operator matrix $\mathbf{K}(X, X)$ with entries $K(X_i, X_j)$ and $Z = (Z_i)_{i \in [N]} \in \cY^N$.
Using block operator notation, we can write the linear system more concisely as $\mathbf{K}(X, X) \cdot Z = Y$.
Assuming the kernel $K$ is non-degenerate and the $X_i$ are pairwise distinct, we can, by \Todo{ref!}Lemma ??, find an inverse of $\mathbf{K}(X, X)$ and write $Z = \mathbf{K}(X, X)^{-1} \cdot Y$.

Let $x \in X$ and $\mathbf{K}(x, X)$ be the vector with elements $K(x, X_i)$.
Using this notation, we can rewrite \cref{eq:optimal-recovery-f} as
\begin{equation}
	f(x) = \mathbf{K}(x, X) \cdot \mathbf{K}(X, X) \cdot Y \ .
\end{equation}
The value of the loss function is the $\mathcal{H}$-norm of this $f$.
As $\mathcal{H}$ is a Hilbert Space, we can compute it using the scalar product:
\begin{align}
	\norm{f}_\mathcal{H}^2 &= \left< f, f\right>_\mathcal{H}\\
	&= \left< \sum_{i=1}^N K(x, X_i) Z_i, \sum_{i=1}^N K(x, X_j) Z_j \right>_\mathcal{H}\\
	&= \sum_{i=1}^N \sum_{j=1}^N \left< K(x, X_i) Z_i, K(x, X_j) Z_j \right>_\mathcal{H}\\
	&= \sum_{i=1}^N \sum_{j=1}^N \left< Z_i, K(X_i, X_j) Z_j \right>_\cY\\
	&= \left< Z, \mathbf{K}(X, X) \cdot Z\right >_{\cY^N}\\
	&= Z^\T \cdot \mathbf{K}(X, X) \cdot Z \\
	&= (\mathbf{K}(X, X)^{-1} \cdot Y)^\T \cdot \mathbf{K}(X, X) \cdot \mathbf{K}(X, X)^{-1} \cdot Y\\
	&= Y^\T \cdot (\mathbf{K}(X, X)^{-1})^\T \cdot Y\\
	&=  Y^\T \cdot \mathbf{K}(X, X)^{-1} \cdot Y \ .
\end{align}
In the fourth line we used the reproducing property and in the second to last line the fact $\mathbf{K}$ that $K$ is Hermitian, that is, $K(x, y)^\T = K(y, x)$.
All in all, this leaves us with an appealing and compact form for the optimal recovery loss:
\begin{equation}
	l(X, Y) = Y^\T \mathbf{K}(X, X)^{-1} Y \ .
\end{equation}
\subsection{Ridge Regression}

\begin{equation}
	\label{eq:ridge-regression-loss}
	l(X, Y) \coloneqq \inf_{f \in \mathcal{H}} \lambda \norm{f}_\mathcal{H}^2 
	+ l_\cY (f(X), Y)
\end{equation}

\begin{equation}
	\label{eq:ridge-regression-f}
	f(x) = \mathbf{K}(x, X)^\T \left(\mathbf{K}(x, X) + \lambda I\right)^{-1}Y \ .
\end{equation}
Ridge regression can be seen as a Tikhonov-regularized variant of optimal recovery.

\subsection{Features Space and Feature Maps}

Next, we define feature spaces and feature maps.
They are of special interest in machine learning and are especially relevant for Support Vector Machines (SVMs) \cite{steinwart08}, where they enable learning non-linear decision functions.
\citet{steinwart08} give an extended introduction to kernels, feature maps and feature space for the scalar case.
Here, we will define these terms for the more general case of operator-valued kernels and present the results that are relevant for later sections.

\begin{definition}
	\label{def:feature-map-space}
	A Hilbert space $\cF$ and a function $\psi: \cX \rightarrow L(\cY, \cF)$ are \emph{feature space} and \emph{feature map} for the Kernel $K$ if for all $x_1, x_2 \in \cX, y_1, y_2 \in \cY$:
	\begin{equation}
		\left< y_1, K(x_1, x_2) y_2\right>_\cY = \left< \psi(x_1) y_1, \psi(x_2) y_2\right>_\cF \ .
	\end{equation}
\end{definition}

Let $\psi^\T: \cX \rightarrow L(\cF, \cY)$ be the adjoint of $\psi$ in the sense that for all $x \in \cX, y\in \cY, f \in \cF$
\begin{equation}
 \left< \psi(x)y, f\right>_\cF = \left<y, \psi^\T(x) \alpha \right>_\cY \ .
\end{equation}
With the adjoint we can derive a concise equation for the kernel $K$.

\begin{equation}
	\label{eq:kernel-feature-map}
	K(x_1, x_2) = \psi^\T(x_1)\psi(x_2)
\end{equation}

\begin{theorem}
	\begin{equation}
		\left< \psi^\T(\cdot) \alpha, \psi^\T(\cdot) \beta\right>_\mathcal{H} = \left< \alpha, \beta\right>_\mathcal{F} \ .
	\end{equation}
\end{theorem}
\begin{corollary}
	\label{cor:feature-space-norm}
	\begin{equation}
		\norm{\psi^\T(\cdot)\alpha}_\mathcal{H}^2 = \norm{\alpha}_\mathcal{F}^2 \ .
	\end{equation}
\end{corollary}

\subsection{Product Space and Block Operator Notation}
Later on, it would be quite cumbersome to always treat the $X_i$ separately.
Hence we will now introduce block operator syntax which will enable us to write equations more concisely.
As in later parts we will almost exclusively rely on this notation, we first have to make sure that all used syntax is mathematically correct.
In the following we will define the product space of $\cY$, some useful notation and assure that all future calculations involving that new notation are mathematically sound.

Let $\cY^N \coloneqq \bigtimes_{i=1}^N \cY$ be the $N$-fold product space and $A, B \in \cY^N$.
Defining $\left < A, B \right >_{\cY^N} \coloneqq \sum_{i=1}^{N} \left <A_i, B_i \right >$ fulfills the properties of an inner product on $\cY^N$ and thus makes $\cY^N$ a Hilbert space itself.


\begin{definition}[Block Operator Matrix]
	A \emph{Block Operator Matrix} $\mathbf{A}$ is an $(N \times N)$-Matrix with Elements in $\cL(Y, Y)$.
\end{definition}

\begin{definition}[$\cY^N$]
	content... 
\end{definition}
Explain: We can express the corresponding inner product as matrix multiplication.

\begin{theorem}
	$(\cL(\cY^N, \cY^N), +, \circ)$ is a  ring.
\end{theorem}
\begin{proof}
	We conduct the proof by showing that $\cL(\cY^N, \cY^N) \cong \R^{nN \times nN}$.
\end{proof}

As a consequence of the proof, we get the following corollary.
\begin{corollary}
	\label{cor:matrix-ring-equivalence}
	$\cL(\cY^N, \cY^N) \cong \R^{nN \times nN}$.
\end{corollary}

\begin{corollary}
	\label{cor:matrix-vector-equivalence}
	Block-Operator Matrix times vector is isomorphic to standard matrix times vector.
\end{corollary}

\begin{figure}
	This will be an illustration of isomorphisms in the block operator matrix example.
	\caption{And this will be the supercool caption.}
	\label{fig:block-operator-example}
\end{figure}

\begin{example}
	Let $\cY = \R^n$.
	\Todo{Discard Example, make it lemma/theorem.
		Replace $\R$ by general field, as Hilbert Spaces are vector spaces,
		make first isomorphism monoid homomorphism, maybe even ring homomorphism.
		This means we can work with $\mathbf{K}$ just like a regular matrix.
		Check if $\mathbf{K}$ really is symmetric or not just hermitian.
		Also prove that there is an inverse under certain circumstances.}
	Then we have $\cL(\cY) \cong \R^{n\times n}$, which means applying a (bounded) linear operator to a $y \in \cY$ can be expressed as matrix-vector multiplication.
	We now claim that in this case $\cL(\cY^N) \cong \R^{nN \times nN}$ and that the computation of $\mathbf{A}\cdot \mathbf{y}$ can be expressed as a standard matrix-vector product.
	For the first part, let $\mathbf{A} \in \cL(\cY^N)$ and $\mathbf{y} \in \cY^N$. 
	Write $\mathbf{A} = (\mathbf{A}_{i, j})$ with $\mathbf{A}_{i,j} \in \cL(\cY)$. 
	We obtain the corresponding $(nN \times nN)$-matrix by first applying the canonical isomorphism $\cL(\cY) \cong \R^{n\times n}$ to each $\mathbf{A}_{i,j}$ and then "flattening" the result.
	For the second part, we first have to consider another isomorphism $\cY^N \cong \R^{nN}$ which is given by consecutively pasting the elements of $\cY^N$ into one vector of size $nN$.
	\cref{fig:block-operator-example} illustrates both isomorphisms.
	
	These new representations of $\mathbf{A}$ and $\mathbf{y}$ make it particularly easy to compute $\mathbf{A}\cdot \mathbf{y} = \mathbf{x}$:
	Simply map $\mathbf{A}$ to $A \in \R^{nN \times nN}$ and $\mathbf{y}$ to $y \in \R^{nN}$, calculate $A \cdot y = x$ as the standard matrix-vector product and map the resulting vector $x$ back to $\mathbf{x} \in\cY^N$.	
	\Todo{Image: Block Operator Matrix -> "flattened" Matrix}
\end{example}

With $K$ being the Kernel associated with the RKHS $\mathcal{H}$ and $A, B \in \cY^N$ we will write $\mathbf{K}(A, B)$ for the matrix with the entries $K(A_i, B_j)$ at position $(i, j)$.


\subsection{Optimal Recovery}
\label{sec:optimal-recovery}

\begin{theorem}[Representer Theorem]
	\label{theo:representer}
\end{theorem}

The optimal recovery loss is defined as
\begin{equation}
	\label{eq:optimal-recovery-loss}
	l(X, Y) \coloneqq \min\{\norm{f}_\mathcal{H}^2 ~|~ f\in \mathcal{H} \text{~and~} f(X) = Y\}
\end{equation}

By \cref{theo:representer}, the $f \in \mathcal{H}$ which minimizes \cref{eq:optimal-recovery-loss} admits a representation as 
\begin{equation}
	\label{eq:optimal-recovery-f}
	f(x) = \sum_{i=1}^N K(x, X_i) Z_i \ ,
\end{equation}
where the $Z_i$ are the solutions to the following linear system ($1 \leq j \leq N$):
\begin{equation}
	\sum_{i=1}^{N} K(X_j, X_i) Z_i = Y_j
\end{equation}
For better readability we translate this expression into block operator notation:
Introduce the block operator matrix $\mathbf{K}(X, X)$ with entries $K(X_i, X_j)$ and $Z = (Z_i)_{i \in [N]} \in \cY^N$.
Using block operator notation, we can write the linear system more concisely as $\mathbf{K}(X, X) \cdot Z = Y$.
Assuming the kernel $K$ is non-degenerate and the $X_i$ are pairwise distinct, we can, by \Todo{ref!}Lemma ??, find an inverse of $\mathbf{K}(X, X)$ and write $Z = \mathbf{K}(X, X)^{-1} \cdot Y$.

Let $x \in X$ and $\mathbf{K}(x, X)$ be the vector with elements $K(x, X_i)$.
Using this notation, we can rewrite \cref{eq:optimal-recovery-f} as
\begin{equation}
	f(x) = \mathbf{K}(x, X) \cdot \mathbf{K}(X, X) \cdot Y \ .
\end{equation}
The value of the loss function is the $\mathcal{H}$-norm of this $f$.
As $\mathcal{H}$ is a Hilbert Space, we can compute it using the scalar product:
\begin{align}
	\norm{f}_\mathcal{H}^2 &= \left< f, f\right>_\mathcal{H}\\
	&= \left< \sum_{i=1}^N K(x, X_i) Z_i, \sum_{i=1}^N K(x, X_j) Z_j \right>_\mathcal{H}\\
	&= \sum_{i=1}^N \sum_{j=1}^N \left< K(x, X_i) Z_i, K(x, X_j) Z_j \right>_\mathcal{H}\\
	&= \sum_{i=1}^N \sum_{j=1}^N \left< Z_i, K(X_i, X_j) Z_j \right>_\cY\\
	&= \left< Z, \mathbf{K}(X, X) \cdot Z\right >_{\cY^N}\\
	&= Z^\T \cdot \mathbf{K}(X, X) \cdot Z \\
	&= (\mathbf{K}(X, X)^{-1} \cdot Y)^\T \cdot \mathbf{K}(X, X) \cdot \mathbf{K}(X, X)^{-1} \cdot Y\\
	&= Y^\T \cdot (\mathbf{K}(X, X)^{-1})^\T \cdot Y\\
	&=  Y^\T \cdot \mathbf{K}(X, X)^{-1} \cdot Y \ .
\end{align}
In the fourth line we used the reproducing property and in the second to last line the fact $\mathbf{K}$ that $K$ is Hermitian, that is, $K(x, y)^\T = K(y, x)$.
All in all, this leaves us with an appealing and compact form for the optimal recovery loss:
\begin{equation}
	\label{eq:optimal-recovery-loss-closed-form}
	l(X, Y) = Y^\T \mathbf{K}(X, X)^{-1} Y \ .
\end{equation}
\subsection{Ridge Regression}

Aka Tikhonov regularization aka $L_2$ parameter regularization.
\begin{equation}
	\label{eq:ridge-regression-loss}
	l(X, Y) \coloneqq \inf_{f \in \mathcal{H}} \lambda \norm{f}_\mathcal{H}^2 
	+ l_\cY (f(X), Y)
\end{equation}

\begin{equation}
	\label{eq:ridge-regression-f}
	f(x) = \mathbf{K}(x, X)^\T \left(\mathbf{K}(x, X) + \lambda I\right)^{-1}Y \ .
\end{equation}

\begin{equation}
	\label{eq:ridge-regression-loss-se-closed-form}
	l(X, Y) = \lambda Y^\T (\mathbf{K}(X, X) + \lambda \mathbbm{1}_N)^{-1} Y \ .
\end{equation}
Ridge regression can be seen as a Tikhonov-regularized variant of optimal recovery.

\subsection{Equivalence of Optimization Problems}

\begin{definition}
	\label{def:equivalent-problems}
	Let $f: X \rightarrow \R$ and $g: Y \rightarrow \R$ be the target functions of two optimization problems $A$ and $B$.
	Then $A$ and $B$ are \emph{equivalent} if there exists a bijection $\varphi: X \rightarrow Y$ such that
	\begin{equation}
		x \in X \text{~is optimal for A~} \Leftrightarrow \phi(x) \in Y \text{~is optimal for A~}\ .
	\end{equation}
\end{definition}