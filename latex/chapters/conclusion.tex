\section{Conclusion}

In this thesis, chapter 3 of the paper "Do Ideas Have Shape? Plato's Theory of Forms as the Continuous Limit of Neural Networks" by Houman Owhadi has been presented.
After a brief introduction to ResNets, the concept of mechanical regression was derived from a simpliefied ResNet model.
This was done by first showing the equivalence of the ResNet formulation to a discrete stationary action principle.
Using Hamiltonian mechanics, this problem was reformulated as a discrete geodesic shooting problem, which can be parameterized by the initial momentum.
All three discrete problems have continuous counterparts, which can be obtained when the number of ResNet layers tends towards infinity or equivalently the step size towards $0$.
It has been shown that the minimal values converge by utilizing the parameterization by the optimal initial momentum.
Also, the adherence values of sequences of minimizers for the discrete problems are optimal points for the continuous problems.

In \cref{sec:algorithm}, an algorithm has been derived from the continuous geodesic shooting formulation.
The Hamiltonian system was discretized with a modified Leapfrog integrator and the optimal initial momentum was approximated via a version of gradient descent.
Parts of this algorithm were implemented and numerical experiments were conducted for a common machine learning benchmark dataset.
The experiment yielded results similar to those presented by \cite{owhadi20}, showing decreasing space deformation with increasing balancing parameter $\nu$.

In this thesis, only the first parts of the paper by \citet{owhadi20} could be discussed.
In the remaining chapters, Owhadi extends the simplified ResNet model to artificial neural networks.
This includes a regularization technique and non-linear activation functions, which play an essential role in the practical application of neural networks.
The composition of multiple idea registration models as described in \cref{sec:mechanical-regression} leads to a model of artificial neural networks.
Furthermore, \citet{owhadi20} introduces \emph{reduced equivariant multichannel kernels} which can be used to model convolutional neural networks.