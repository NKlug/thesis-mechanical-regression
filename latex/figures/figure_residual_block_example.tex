\begin{figure}
	\centering
	\begin{tikzpicture}
		[
		font=\scriptsize,
		block/.style ={rectangle, draw=black, thick, text width=15em, align=center, minimum height=1.5em}
		]
		\node[] (a) [block] {Fully Connected Layer (1024)};
		\node[below= -1.5\pgflinewidth of a] (a1) [block] {BatchNorm};
		\node[below= -1.5\pgflinewidth of a1] (b) [block] {ReLU};
		\node[below= 4mm of b] (c) [block] {Fully Connected Layer (1024)};
		\node[below= -1.5\pgflinewidth of c] (c1) [block] {BatchNorm};
		\node[below= -1.5\pgflinewidth of c1] (d) [block] {ReLU};
		\node[below= 4mm of d] (e) [block] {$\bigoplus$};
		\node[above=of a] (x) [] {};
		\draw[->, line width=1pt] (b.south) -- (c.north);
		\draw[->, line width=1pt] (d.south) -- (e.north);
		\draw[->, line width=1pt] (x) -- node [pos=0.1, right] {\small $x$} (a);
		\node[below=of e] (y) [] {};
		\draw[->, line width=1pt] (e) -- node [midway, right] {\small $f(x)$} (y);
		\node[above=.4 of a] (z) [] {};
		\node[right=12em of z] (h) [] {};
		\draw[line width=1pt] (z.center) -- (h.center);
		\draw[->, line width=1pt] (h.center) |- node [pos=.26,right] {\small $x$} (e.east);
		\draw [decorate, decoration={brace, amplitude=10pt}] ([xshift=-0.3cm]d.south west)-- ([xshift=-0.3cm]a.north west) node [black,midway, left, xshift=-10pt]{\small $g(x)$};
	\end{tikzpicture}
	\caption{Typical residual block.
	$\bigoplus$ indicates the inputs of both arrows being summed.
	The block learns the residual $g(x)$, upon which $x$ is added, resulting in $f(x)$.
	This particular block architecture was used in \cite{drover18} for 3D Human Pose Estimation.
	Similar blocks are used in image classification and object detection, where the fully connected layers are usually replaced by convolutions.
	}
	\label{fig:residual-block}
\end{figure}